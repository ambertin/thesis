% !TeX program = pdfLaTeX
\documentclass[12pt]{article}
\usepackage{amsmath}
\usepackage{graphicx,psfrag,epsf}
\usepackage{enumerate}
\usepackage{natbib}
\usepackage{textcomp}
\usepackage[hyphens]{url} % not crucial - just used below for the URL
\usepackage{hyperref}
\providecommand{\tightlist}{%
  \setlength{\itemsep}{0pt}\setlength{\parskip}{0pt}}

%\pdfminorversion=4
% NOTE: To produce blinded version, replace "0" with "1" below.
\newcommand{\blind}{0}

% DON'T change margins - should be 1 inch all around.
\addtolength{\oddsidemargin}{-.5in}%
\addtolength{\evensidemargin}{-.5in}%
\addtolength{\textwidth}{1in}%
\addtolength{\textheight}{1.3in}%
\addtolength{\topmargin}{-.8in}%

%% load any required packages here





\begin{document}


\def\spacingset#1{\renewcommand{\baselinestretch}%
{#1}\small\normalsize} \spacingset{1}


%%%%%%%%%%%%%%%%%%%%%%%%%%%%%%%%%%%%%%%%%%%%%%%%%%%%%%%%%%%%%%%%%%%%%%%%%%%%%%

\if0\blind
{
  \title{\bf Honors Thesis Proposal}

  \author{
        Audrey Bertin \\
    Statistical and Data Sciences\\
      }
  \maketitle
} \fi

\if1\blind
{
  \bigskip
  \bigskip
  \bigskip
  \begin{center}
    {\LARGE\bf Honors Thesis Proposal}
  \end{center}
  \medskip
} \fi

\bigskip
\begin{abstract}
For my honors thesis in the department of Statistical and Data Sciences,
I would like to focus on the topic of reproducibility in data science.
Reproducibility is critical to the advancement of knowledge,yet there
are not very many effective tools to address it and it does not appear
to be widely taught at US academic institutions. In this honors project,
I will discuss several main areas - 1) The need for widespread knowledge
about reproducibility. 2) An R package I am working on developing to
address this issue, along with new ideas and ways of coding I am
learning through the process. 3) How my work is benefitting, and will
continue to benefit introductory data science students and other users.
4) Further recommendations on how best to incorporate reproducibility
into data science education.
\end{abstract}

\noindent%
{\it Keywords:} reproducibility, statistical software, workflow, collaboration, teaching, curriculum recommendations
\vfill

\newpage
\spacingset{1.45} % DON'T change the spacing!

\section{The Broad Issue: Reproducibility in Data
Science}\label{the-broad-issue-reproducibility-in-data-science}

As research is becoming increasingly data-driven, and because knowledge
can be shared worldwide so rapidly, reproducibility is critical to the
advancement of scientific knowledge.

Data-based research cannot be fully \emph{reproducible} unless the
requisite code and data files produce identical results when run by
another analyst. When researchers provide the code and data used for
their work in a well-organized and reproducible format, readers are more
easily able to determine the veracity of any findings by following the
steps from raw data to conclusions.

The creators of reproducible research can more easily receive more
specific feedback (including bug fixes) on their work. Moreover, others
interested in the research topic can use the code to apply the methods
and ideas used in one project to their own work with minimal effort.

However, while the necessity of reproducibility is clear, there are
significant behavioral and technical challenges that impede its
widespread implementation, and no clear consensus on standards of what
constitutes reproducibility in published research. Not only are the
\emph{components} of reproducible research up for discussion (e.g., need
the software be open source?), but the corresponding
\emph{recommendations} for ensuring reproducibility also vary.

Much of the discussion around reproducibility is also generalized---it
is written to be applicable to users working with a variety of
statistical software programs. Since all statistical software programs
operate differently, generalized recommendations on reproducibility are
often shallow and unspecific. While they provide useful guidelines, they
can often be difficult to implement, particularly to new analysts who
are unsure how to apply such recommendations within the software
programs they are using. Thus, reproducibility recommendations tailored
to specific software programs are more likely to be adopted.

One of the most effective ways to work on improving reproducibility in
the data science community is to focus on a specific piece of software
used by data analysts. In the program in Statistical and Data Sciences
at Smith College, we use the program R. Since R is freely available
online and very popular in the data science community, it is a good
candidate for research focus. As a result, it is the primarily language
used to teach data science at many academic institions around the US.

Unfortunately, there are not very many academic papers or software
packages discussing reproducibility in R, and much of the work that does
exist is not ideal for students just beginning to learn data science.
Much of this work is narrowly tailored, with each package effectively
addressising a small component of reproducibility---file structure,
modularization of code, version control, etc. Many existing software
packages available to R users succeed at their area of focus, but at a
cost. They are often difficult to learn and operate, providing a barrier
to entry for less-experienced data analysts.

Due to the above reasons, reproducibility is often left behind and not
prioritized to the degree it should be.

\section{My Focus}\label{my-focus}

In order to address this issue, I have been working to help develop an R
package that can provide important information about the reproducibility
of a data analysis project and share recommendations for how to improve
it, all in a few short and simple lines of code. The package is written
in such a way as to be accessible to introductory data science students,
and can easily be brought into the classroom as a tool for integrating
the concept of reproducibility into data science education.

The package is still in the development process, and I will be spending
a significant portion of my thesis time improving it and expanding its
functionality.

One major focus will be delving into adding \texttt{make}-like
functionality that can analyze an R project structure and files and use
this information to generate a Makefile. This Makefile would have
information about target files and their prerequisites and would assist
with making sure that re-running an analysis is done as quickly as
possible by ensuring that only the necessary code and files that have
been updated are run when rebuilding and re-running code.

In my thesis, I would like to focus much of my discussion on this
package while covering the following four general topics:

\begin{enumerate}
\def\labelenumi{\arabic{enumi})}
\tightlist
\item
  The need for widespread knowledge about reproducibility.
\end{enumerate}

In this background section, I would like to delve into evidence that
widespread reproducibility is lacking across the sciences. Several
reports from different academic fields dive into this idea and share the
state of reproducibility in their areas of study.

I will also cover the benefits of reproducible research, many of which I
mentioned in brief in the previous section. I will also delve into the
importance of teaching and working to integrate reproducibility early on
in the education of students studying data science.

\begin{enumerate}
\def\labelenumi{\arabic{enumi})}
\setcounter{enumi}{1}
\tightlist
\item
  A discussion of my R package and the ideas I have learned through the
  package development process.
\end{enumerate}

In this section, I will describe the purpose of the R package and how it
works (covering several of the major functions).

The package has a very unique structure compared to many other available
packages. Much of the functionality of the package operates under the
hood and in the background. Unlike most packages, it does not have
functions that are used when coding, but rather functions that analyze
users' coding and project building practices. Many of the functions
utilize features of R programming not covered in any courses at Smith
College, which required a significant amount of outside research and
troubleshooting to put into use.

I would like to highlight several of the unique functionalities of the
package that taught me new ways of programming in R. These include:

\begin{itemize}
\item
  File path and directory manipulation
\item
  Function shimming/masking
\item
  Environment variables
\item
  Utilizing hidden files
\item
  Leveraging the dots (\ldots{}) for additional functionality
\item
  Understanding dependencies, keeping track of file update histories,
  and using Makefiles
\end{itemize}

As package development continues throughout the course of this honors
project, the list above will likely expand as new issues are tackled.

\begin{enumerate}
\def\labelenumi{\arabic{enumi})}
\setcounter{enumi}{2}
\tightlist
\item
  How my work has benefitted and will continue to benefit introductory
  data science students and other users.
\end{enumerate}

In this section, I will discuss my work as a statistics tutor,
discussing some of the challenges that seem to have regularly and
demonstrating how my package works to address these issues.

I will also include some preliminary reviews and discussion of how
functional the software is to users who have never seen it before.

Looking back at the bigger picture, I will then talk about the
software's wider applications to reproducibility in general (not just
for introductory students) and its unique place in the world of the
tools available to data scientists.

\begin{enumerate}
\def\labelenumi{\arabic{enumi})}
\setcounter{enumi}{3}
\tightlist
\item
  Further recommendations on how best to incorporate reproducibility
  into data science curricula, including a look at approaches currently
  being pioneered at some universities.
\end{enumerate}

In this section, I will work on defining the ideal way to integrate
reproducibility into the data science curriculum. I will refer to the
guidelines on teaching data science from the 2017 Annual Review of
Statistics and Its Applications, as well as to a variety of other
sources discussing easy methods for implementing different aspects of
reproducibility.

I will also look into some of the new reproducibility curriculum
additions that are beginning to appear at colleges across the country,
including one course offered at the University of Washington that has
reproducibility as its sole focus, as well as recommendations on how to
teach reproducibility as shared by several leaders in the field.

I will focus on how my work fits into the discussion of teaching
reproducibility and how it might be integrated into data science
education in a way that combines effectively with other available tools
such as RMarkdown and GitHub.

\section{Relevant Preparation}\label{relevant-preparation}

I have spent the last year studying reproducibility as part of research
I have been conducting with Prof.~Ben Baumer in the Statistical and Data
Sciences department. That research has focused on creating a publicly
available piece of software that can be downloaded by people all around
the world. In the process of sharing that work, I have already begun the
process of researching the need for reproducibility and available
resources, so I already have a relatively well developed background of
knowledge on reproducibility coming into this honors project. I have
become familiar with several reproducibility experts through my
research, whose work I can delve into further to gather insights on
effective ways to develop the idea data science curriculum.

As a statistics tutor, I also have a good sense of which types of tools
and which aspects of the curriculum work best for students of all
different levels, and can use this knowledge to help inform the
recommendations in this project. I also have made connections with data
scientists and students of all different levels, who I can reach out to
in order to build a diverse pool of testers for my package.

Finally, I have completed a significant number of data analysis and
research projects in my time at Smith. I am very familiar with
strategies for finding/summarizing sources and collecting/analyzing
data, as well as with sharing my findings. I have had experience with
writing lengthy (35+ page) papers before, and have developed excellent
time management skills through that process that will allow me to
successfully complete this project.

\begin{center}\rule{0.5\linewidth}{\linethickness}\end{center}

Thank you for taking the time to consider my honors thesis proposal. I
have had a wonderful time partipating in my reproducibility research
over the last year, and look forward to this opportunity to expand on it
further in a meaningful way.

Audrey Bertin '21

\bibliographystyle{agsm}
\bibliography{bibliography.bib,pkgs.bib}

\end{document}
