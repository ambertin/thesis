% This is the Reed College LaTeX thesis template. Most of the work
% for the document class was done by Sam Noble (SN), as well as this
% template. Later comments etc. by Ben Salzberg (BTS). Additional
% restructuring and APA support by Jess Youngberg (JY).
% Your comments and suggestions are more than welcome; please email
% them to cus@reed.edu
%
% See https://www.reed.edu/cis/help/LaTeX/index.html for help. There are a
% great bunch of help pages there, with notes on
% getting started, bibtex, etc. Go there and read it if you're not
% already familiar with LaTeX.
%
% Any line that starts with a percent symbol is a comment.
% They won't show up in the document, and are useful for notes
% to yourself and explaining commands.
% Commenting also removes a line from the document;
% very handy for troubleshooting problems. -BTS

% As far as I know, this follows the requirements laid out in
% the 2002-2003 Senior Handbook. Ask a librarian to check the
% document before binding. -SN

%%
%% Preamble
%%
% \documentclass{<something>} must begin each LaTeX document
\documentclass[12pt,twoside]{reedthesis}
% Packages are extensions to the basic LaTeX functions. Whatever you
% want to typeset, there is probably a package out there for it.
% Chemistry (chemtex), screenplays, you name it.
% Check out CTAN to see: https://www.ctan.org/
%%
\usepackage{graphicx,latexsym}
\usepackage{amsmath}
\usepackage{amssymb,amsthm}
\usepackage{longtable,booktabs,setspace}
\usepackage{chemarr} %% Useful for one reaction arrow, useless if you're not a chem major
\usepackage[hyphens]{url}
% Added by CII
\usepackage{hyperref}
\usepackage{lmodern}
\usepackage{float}
\floatplacement{figure}{H}
% End of CII addition
\usepackage{rotating}

% Next line commented out by CII
%%% \usepackage{natbib}
% Comment out the natbib line above and uncomment the following two lines to use the new
% biblatex-chicago style, for Chicago A. Also make some changes at the end where the
% bibliography is included.
%\usepackage{biblatex-chicago}
%\bibliography{thesis}


% Added by CII (Thanks, Hadley!)
% Use ref for internal links
\renewcommand{\hyperref}[2][???]{\autoref{#1}}
\def\chapterautorefname{Chapter}
\def\sectionautorefname{Section}
\def\subsectionautorefname{Subsection}
% End of CII addition

% Added by CII
\usepackage{caption}
\captionsetup{width=5in}
% End of CII addition

% \usepackage{times} % other fonts are available like times, bookman, charter, palatino

% Syntax highlighting #22
  \usepackage{color}
  \usepackage{fancyvrb}
  \newcommand{\VerbBar}{|}
  \newcommand{\VERB}{\Verb[commandchars=\\\{\}]}
  \DefineVerbatimEnvironment{Highlighting}{Verbatim}{commandchars=\\\{\}}
  % Add ',fontsize=\small' for more characters per line
  \usepackage{framed}
  \definecolor{shadecolor}{RGB}{248,248,248}
  \newenvironment{Shaded}{\begin{snugshade}}{\end{snugshade}}
  \newcommand{\KeywordTok}[1]{\textcolor[rgb]{0.13,0.29,0.53}{\textbf{#1}}}
  \newcommand{\DataTypeTok}[1]{\textcolor[rgb]{0.13,0.29,0.53}{#1}}
  \newcommand{\DecValTok}[1]{\textcolor[rgb]{0.00,0.00,0.81}{#1}}
  \newcommand{\BaseNTok}[1]{\textcolor[rgb]{0.00,0.00,0.81}{#1}}
  \newcommand{\FloatTok}[1]{\textcolor[rgb]{0.00,0.00,0.81}{#1}}
  \newcommand{\ConstantTok}[1]{\textcolor[rgb]{0.00,0.00,0.00}{#1}}
  \newcommand{\CharTok}[1]{\textcolor[rgb]{0.31,0.60,0.02}{#1}}
  \newcommand{\SpecialCharTok}[1]{\textcolor[rgb]{0.00,0.00,0.00}{#1}}
  \newcommand{\StringTok}[1]{\textcolor[rgb]{0.31,0.60,0.02}{#1}}
  \newcommand{\VerbatimStringTok}[1]{\textcolor[rgb]{0.31,0.60,0.02}{#1}}
  \newcommand{\SpecialStringTok}[1]{\textcolor[rgb]{0.31,0.60,0.02}{#1}}
  \newcommand{\ImportTok}[1]{#1}
  \newcommand{\CommentTok}[1]{\textcolor[rgb]{0.56,0.35,0.01}{\textit{#1}}}
  \newcommand{\DocumentationTok}[1]{\textcolor[rgb]{0.56,0.35,0.01}{\textbf{\textit{#1}}}}
  \newcommand{\AnnotationTok}[1]{\textcolor[rgb]{0.56,0.35,0.01}{\textbf{\textit{#1}}}}
  \newcommand{\CommentVarTok}[1]{\textcolor[rgb]{0.56,0.35,0.01}{\textbf{\textit{#1}}}}
  \newcommand{\OtherTok}[1]{\textcolor[rgb]{0.56,0.35,0.01}{#1}}
  \newcommand{\FunctionTok}[1]{\textcolor[rgb]{0.00,0.00,0.00}{#1}}
  \newcommand{\VariableTok}[1]{\textcolor[rgb]{0.00,0.00,0.00}{#1}}
  \newcommand{\ControlFlowTok}[1]{\textcolor[rgb]{0.13,0.29,0.53}{\textbf{#1}}}
  \newcommand{\OperatorTok}[1]{\textcolor[rgb]{0.81,0.36,0.00}{\textbf{#1}}}
  \newcommand{\BuiltInTok}[1]{#1}
  \newcommand{\ExtensionTok}[1]{#1}
  \newcommand{\PreprocessorTok}[1]{\textcolor[rgb]{0.56,0.35,0.01}{\textit{#1}}}
  \newcommand{\AttributeTok}[1]{\textcolor[rgb]{0.77,0.63,0.00}{#1}}
  \newcommand{\RegionMarkerTok}[1]{#1}
  \newcommand{\InformationTok}[1]{\textcolor[rgb]{0.56,0.35,0.01}{\textbf{\textit{#1}}}}
  \newcommand{\WarningTok}[1]{\textcolor[rgb]{0.56,0.35,0.01}{\textbf{\textit{#1}}}}
  \newcommand{\AlertTok}[1]{\textcolor[rgb]{0.94,0.16,0.16}{#1}}
  \newcommand{\ErrorTok}[1]{\textcolor[rgb]{0.64,0.00,0.00}{\textbf{#1}}}
  \newcommand{\NormalTok}[1]{#1}

% To pass between YAML and LaTeX the dollar signs are added by CII
\title{My Final College Paper}
\author{Audrey M. Bertin}
% The month and year that you submit your FINAL draft TO THE LIBRARY (May or December)
\date{May 2021}
\division{Statistical and Data Sciences}
\advisor{Benjamin S. Baumer}
\institution{Smith College}
\degree{Bachelor of Arts}
%If you have two advisors for some reason, you can use the following
% Uncommented out by CII
% End of CII addition

%%% Remember to use the correct department!
\department{Statistical and Data Sciences}
% if you're writing a thesis in an interdisciplinary major,
% uncomment the line below and change the text as appropriate.
% check the Senior Handbook if unsure.
%\thedivisionof{The Established Interdisciplinary Committee for}
% if you want the approval page to say "Approved for the Committee",
% uncomment the next line
%\approvedforthe{Committee}

% Added by CII
%%% Copied from knitr
%% maxwidth is the original width if it's less than linewidth
%% otherwise use linewidth (to make sure the graphics do not exceed the margin)
\makeatletter
\def\maxwidth{ %
  \ifdim\Gin@nat@width>\linewidth
    \linewidth
  \else
    \Gin@nat@width
  \fi
}
\makeatother

%Added by @MyKo101, code provided by @GerbrichFerdinands

\renewcommand{\contentsname}{Table of Contents}
% End of CII addition

\setlength{\parskip}{0pt}

% Added by CII

\providecommand{\tightlist}{%
  \setlength{\itemsep}{0pt}\setlength{\parskip}{0pt}}

\Acknowledgements{
I want to thank a few people.
}

\Dedication{
You can have a dedication here if you wish.
}

\Preface{
This is an example of a thesis setup to use the reed thesis document
class (for LaTeX) and the R bookdown package, in general.
}

\Abstract{
The preface pretty much says it all. \par
Second paragraph of abstract starts here.
}

% End of CII addition
%%
%% End Preamble
%%
%
\begin{document}

% Everything below added by CII
  \maketitle

\frontmatter % this stuff will be roman-numbered
\pagestyle{empty} % this removes page numbers from the frontmatter
  \begin{acknowledgements}
    I want to thank a few people.
  \end{acknowledgements}
  \begin{preface}
    This is an example of a thesis setup to use the reed thesis document
    class (for LaTeX) and the R bookdown package, in general.
  \end{preface}
  \hypersetup{linkcolor=black}
  \setcounter{tocdepth}{2}
  \tableofcontents


  \begin{abstract}
    The preface pretty much says it all. \par
    Second paragraph of abstract starts here.
  \end{abstract}
  \begin{dedication}
    You can have a dedication here if you wish.
  \end{dedication}
\mainmatter % here the regular arabic numbering starts
\pagestyle{fancyplain} % turns page numbering back on

\chapter*{Introduction}\label{introduction}
\addcontentsline{toc}{chapter}{Introduction}

Potential sources:

\url{https://arxiv.org/abs/1401.3269}

\url{https://academic.oup.com/isp/article-abstract/17/4/392/2528285}

\url{https://dl.acm.org/doi/abs/10.1145/3186266?casa_token=3yv8mooiZXYAAAAA:AUWshgsmm-4ulr7qYK2vm3a6EdJneFLgn3nxplGeaZpT7hgFcIRkRA7edGHdjg_pPvs5p-GoHHFo}

\url{https://dl.acm.org/doi/abs/10.1145/3027385.3029445?casa_token=rgNbcFWIA1AAAAAA:cZRPAy1KYewKfheFe74GBDwKzF9Q8X0xMau0AtBVkYSTYrd7apJEnwDY_T8oqh1cZQED1gHjjtq4}

\url{https://berkeleysciencereview.com/2014/06/reproducible-collaborative-data-science/}

\url{https://guides.lib.uw.edu/research/reproducibility/teaching}

\url{https://escholarship.org/uc/item/90b2f5xh}

\url{https://www.mitpressjournals.org/doi/full/10.1162/dint_a_00053}

\url{https://www.pnas.org/content/115/11/2584}

\url{https://www.pnas.org/content/115/11/2561}

\chapter{An Introduction to Reproducibility}\label{reproducibility}

\section{What is reproducibility?}\label{what-is-reproducibility}

Data-based research is considered fully reproducible when the requisite
code and data files produce identical results when run by another
analyst. As research is becoming increasingly data-driven, and because
knowledge can be shared worldwide so rapidly, reproducibility is
critical to the advancement of scientific knowledge. Academics around
the world have recognized this, and publications and discussions
addressing reproducibility appear to have increased in the last several
years (Eisner (2018); Fidler \& Wilcox (2018); Gosselin (2020); McArthur
(2019); Wallach, Boyack, \& Ioannidis (2018)).

• Brief history of reproducibility • Reproducibility vs replicability

\section{Why is reproducibility
important?}\label{why-is-reproducibility-important}

Reproducible research has a wide variety of benefits. When researchers
provide the code and data used for their work in a well-organized and
reproducible format, readers are more easily able to determine the
veracity of any findings by following the steps from raw data to
conclusions. The creators of reproducible research can also more easily
receive more specific feedback (including bug fixes) on their work.
Moreover, others interested in the research topic can use the code to
apply the methods and ideas used in one project to their own work with
minimal effort.

\section{The Components of Reproducible
Research}\label{the-components-of-reproducible-research}

\subsection{Literature}\label{literature}

Publications on reproducibility can be found in all areas of scientific
research. However, as Goodman, Fanelli, \& Ioannidis (2016) argue, the
language and conceptual framework of research reproducibility varies
significantly across the sciences, and there are no clear standards on
reproducibility agreed upon by the scientific community as a whole. We
consider recommendations from a variety of fields and determine the key
aspects of reproducibility faced by scientists in different disciplines.

Kitzes, Turek, \& Deniz (2017) present a collection of case studies on
reproducibility practices from across the data-intensive sciences,
illustrating a variety of recommendations and techniques for achieving
reproducibility. Although their work does not come to a consensus on the
exact standards of reproducibility that should be followed, several
common trends and principles emerge from their case studies: 1) use
clear separation, labeling, and documentation, 2) automate processes
when possible, and 3) design the data analysis workflow as a sequence of
small steps glued together, with outputs from one step serving as inputs
into the next. This is a common suggestion within the computing
community, originating as part of the Unix philosophy (Gancarz (2003)).

Cooper et al. (2017) focus on data analysis in \texttt{R} and identify a
similar list of important reproducibility components, reinforcing the
need for clearly labeled, well-documented, and well-separated files. In
addition, they recommend publishing a list of dependencies and using
version control. Broman (2019) reiterates the need for clear naming and
file separation while sharing several additional suggestions: keep the
project contained in one directory, use relative paths, and include a
\texttt{README}.

The reproducibility recommendations from R OpenSci, a non-profit
initiative founded in 2011 to make scientific data retrieval
reproducible, share similar principles to those discussed previously.
They focus on a need for a well-developed file system, with no
extraneous files and clear labeling. They also reiterate the need to
note dependencies and use automation when possible, while making clear a
suggestion not present in the previously-discussed literature: the need
to use seeds, which allow for the saving and restoring of the random
number generator state, when running code involving randomness (Martinez
et al. (2018)).

When considered in combination, these sources provide a well-rounded
picture of the components important to research reproducibility. Using
this literature as a guideline, we identify several key features of
reproducible work. These recommendations are a matter of opinion---due
to the lack of agreement on which components of reproducibility are most
important, we select those that are mentioned most often, as well as
some that are mentioned less but that we view as important.
\begin{enumerate}
\def\labelenumi{\arabic{enumi}.}
\tightlist
\item
  A well-designed file structure:
\end{enumerate}
\begin{itemize}
\tightlist
\item
  Separate folders for different file types.
\item
  No extraneous files.
\item
  Minimal clutter.
\end{itemize}
\begin{enumerate}
\def\labelenumi{\arabic{enumi}.}
\setcounter{enumi}{1}
\tightlist
\item
  Good documentation:
\end{enumerate}
\begin{itemize}
\tightlist
\item
  Files are clearly named, preferably in a way where the order in which
  they should be run is clear.
\item
  A README is present.
\item
  Dependencies are noted.
\end{itemize}
\begin{enumerate}
\def\labelenumi{\arabic{enumi}.}
\setcounter{enumi}{2}
\tightlist
\item
  Reproducible file paths:
\end{enumerate}
\begin{itemize}
\tightlist
\item
  No absolute paths, or paths leading to locations outside of a
  project's directory, are used in code---only portable (relative)
  paths.
\end{itemize}
\begin{enumerate}
\def\labelenumi{\arabic{enumi}.}
\setcounter{enumi}{3}
\tightlist
\item
  Randomness is accounted for:
\end{enumerate}
\begin{itemize}
\tightlist
\item
  If randomness is used in code, a seed must also be set.
\end{itemize}
\begin{enumerate}
\def\labelenumi{\arabic{enumi}.}
\setcounter{enumi}{4}
\tightlist
\item
  Readable, styled code:
\end{enumerate}
\begin{itemize}
\tightlist
\item
  Code should be written in a coherent style. Code that conforms to a
  style guide or is written in a consistent dialect is easier to read
  (Hermans \& Aldewereld (2017)). We believe that the \texttt{tidyverse}
  provides the most accessible dialect of \texttt{R}.
\end{itemize}
Much of the available literature focuses on file structure,
organization, and naming, and \texttt{fertile}'s features are consistent
with this. Marwick, Boettiger, \& Mullen (2018)\} provide the framework
for file structure that \texttt{fertile} is based on: a structure
similar to that of an \texttt{R} package (R-Core-Team (2020), Wickham
(2015)), with an \texttt{R} folder, as well as \texttt{data},
\texttt{data-raw}, \texttt{inst}, and \texttt{vignettes}.

\section{Current Focus on Reproducibility in
Academia}\label{current-focus-on-reproducibility-in-academia}

\chapter{Addressing the Challenges of
Reproducibility}\label{my-solution}

\section{Review Of Previous Work}\label{review-of-previous-work}

\subsection{R Packages and Other
Software}\label{r-packages-and-other-software}

Much of this work is highly generalized, written to be applicable to
users working with a variety of statistical software programs. Because
all statistical software programs operate differently, these
recommendations are inherently vague and difficult to implement,
particularly to new analysts who are relatively unfamiliar with their
software. Focused attempts to address reproducibility in specific
certain software programs are more likely to be successful. We focus on
\texttt{R}, due to its open-source nature, accessibility, and popularity
as a tool for statistical analysis.

A small body of \texttt{R} packages focuses on research reproducibility.
\texttt{rrtools} (Marwick (2019)) addresses some of the issues discussed
in Marwick et al. (2018) by creating a basic \texttt{R} package
structure for a data analysis project and implementing a basic
\texttt{testthat::check} functionality. The \texttt{orderly} (FitzJohn
et al. (2020)) package also focuses on file structure, requiring the
user to declare a desired project structure (typically a step-by-step
structure, where outputs from one step are inputs into the next) at the
beginning and then creating the files necessary to achieve that
structure. \texttt{workflowr}'s (Blischak, Carbonetto, \& Stephens
(2019)) functionality is based around version control and making code
easily available online. It works to generate a website containing
time-stamped, versioned, and documented results. \texttt{checkers}
(Ross, DeCicco, \& Randhawa (2018)) allows you to create custom checks
that examine different aspects of reproducibility. \texttt{packrat}
(Ushey, McPherson, Cheng, Atkins, \& Allaire (2018)) is focused on
dependencies, creating a packaged folder containing a project as well as
all of its dependencies so that projects dependent on lesser-used
packages can be easily shared across computers. \texttt{drake} (OpenSci
(2020)) analyzes workflows, skips steps where results are up to date,
and provides evidence that results match the underlying code and data.
Lastly, the \texttt{reproducible} (McIntire \& Chubaty (2020)) package
focuses on the concept of caching: saving information so that projects
can be run faster each time they are re-completed from the start.

Many of these packages are narrow, with each effectively addressing a
small component of reproducibility: file structure, modularization of
code, version control, etc. These packages often succeed in their area
of focus, but at the cost of accessibility to a wider audience. Their
functions are often quite complex to use, and many steps must be
completed to achieve the required reproducibility goal. This cumbersome
nature means that most reproducibility packages currently available are
not easily accessible to users near the beginning of their \texttt{R}
journey, nor particularly useful to those looking for quick and easy
reproducibility checks. A more effective way of realizing widespread
reproducibility is to make the process for doing so simple enough that
it takes little to no conscious effort to implement. You want users to
``fall into a hole'' (we paraphrase Hadley Wickham) of good practice.

\texttt{Continuous\ integration} tools provide more general approaches
to automated checking, which can enhance reproducibility with minimal
code. For example, \texttt{wercker}---a command line tool that leverages
Docker---enables users to test whether their projects will successfully
compile when run on a variety of operating systems without access to the
user's local hard drive (Oracle Corporation (2019)).
\texttt{GitHub\ Actions} is integrated into GitHub and can be configured
to do similar checks on projects hosted in repositories.
\texttt{Travis\ CI} and \texttt{Circle\ CI} are popular continuous
integration tools that can also be used to check \texttt{R} code.

However, while these tools can be useful, they are generalized so as to
be useful to the widest audience. As a result, their checks are not
designed to be \texttt{R}-specific, which makes them sub-optimal for
users looking to address reproducibility issues involving features
specific to the \texttt{R} programming language, such as package
installation and seed setting.

\section{Identifying Gaps In Existing
Solutions}\label{identifying-gaps-in-existing-solutions}

\section{\texorpdfstring{My Contribution: \texttt{fertile}, An R Package
Creating Optimal Conditions For
Reproducibility}{My Contribution: fertile, An R Package Creating Optimal Conditions For Reproducibility}}\label{my-contribution-fertile-an-r-package-creating-optimal-conditions-for-reproducibility}

\subsection{Package Overview}\label{package-overview}

\texttt{fertile} attempts to address these gaps in existing software by
providing a simple, easy-to-learn reproducibility package that, rather
than focusing intensely on a specific area, provides some information
about a wide variety of aspects influencing reproducibility.
\texttt{fertile} is flexible, offering benefits to users at any stage in
the data analysis workflow, and provides \texttt{R}-specific features,
which address certain aspects of reproducibility that can be missed by
external project development software.

\texttt{fertile} is designed to be used on data analyses organized as
\texttt{R} Projects (i.e.~directories containing an \texttt{.Rproj}
file). Once an \texttt{R} Project is created, \texttt{fertile} provides
benefits throughout the data analysis process, both during development
as well as after the fact. \texttt{fertile} achieves this by operating
in two modes: proactively (to prevent reproducibility mistakes from
happening in the first place), and retroactively (analyzing code that
has already been written for potential problems).

\subsection{Proactive Use}\label{proactive-use}

Proactively, the package identifies potential mistakes as they are made
by the user and outputs an informative message as well as a recommended
solution. For example, \texttt{fertile} catches when a user passes a
potentially problematic file path---such as an absolute path, or a path
that points to a location outside of the project directory---to a
variety of common input/output functions operating on many different
file types.
\begin{Shaded}
\begin{Highlighting}[]
\KeywordTok{library}\NormalTok{(fertile)}
\KeywordTok{file.exists}\NormalTok{(}\StringTok{"~/Desktop/my_data.csv"}\NormalTok{)}
\end{Highlighting}
\end{Shaded}
\begin{verbatim}
[1] TRUE
\end{verbatim}
\begin{Shaded}
\begin{Highlighting}[]
\KeywordTok{read.csv}\NormalTok{(}\StringTok{"~/Desktop/my_data.csv"}\NormalTok{)}
\end{Highlighting}
\end{Shaded}
\begin{verbatim}
Error: Detected absolute paths
\end{verbatim}
\begin{Shaded}
\begin{Highlighting}[]
\KeywordTok{read.csv}\NormalTok{(}\StringTok{"../../../Desktop/my_data.csv"}\NormalTok{)}
\end{Highlighting}
\end{Shaded}
\begin{verbatim}
Error: Detected paths that lead outside the project directory
\end{verbatim}
\texttt{fertile} is even more aggressive with functions (like
\texttt{setwd()}) that are almost certain to break reproducibility,
causing them to throw errors that prevent their execution and providing
recommendations for better alternatives.
\begin{Shaded}
\begin{Highlighting}[]
\KeywordTok{setwd}\NormalTok{(}\StringTok{"~/Desktop"}\NormalTok{)}
\end{Highlighting}
\end{Shaded}
\begin{verbatim}
Error: setwd() is likely to break reproducibility. Use here::here() instead.
\end{verbatim}
These proactive warning features are activated immediately after
attaching the \texttt{fertile} package and require no additional effort
by the user.

\subsection{Retroactive Use}\label{retroactive-use}

Retroactively, \texttt{fertile} analyzes potential obstacles to
reproducibility in an RStudio Project (i.e., a directory that contains
an \texttt{.Rproj} file). The package considers several different
aspects of the project which may influence reproducibility, including
the directory structure, file paths, and whether randomness is used
thoughtfully.

The end products of these analyses are reproducibility reports
summarizing a project's adherence to reproducibility standards and
recommending remedies for where the project falls short. For example,
\texttt{fertile} might identify the use of randomness in code and
recommend setting a seed if one is not present.

Users can access the majority of \texttt{fertile}'s retroactive features
through two primary functions, \texttt{proj\_check()} and
\texttt{proj\_analyze()}.

The \texttt{proj\_check()} function runs fifteen different
reproducibility tests, noting which ones passed, which ones failed, the
reason for failure, a recommended solution, and a guide to where to look
for help. These tests include: looking for a clear build chain, checking
to make sure the root level of the project is clear of clutter,
confirming that there are no files present that are not being directly
used by or created by the code, and looking for uses of randomness that
do not have a call to \texttt{set.seed()} present. A full list is
provided below:
\begin{Shaded}
\begin{Highlighting}[]
\KeywordTok{list_checks}\NormalTok{()}
\end{Highlighting}
\end{Shaded}
\begin{verbatim}
-- The available checks in `fertile` are as follows: ----------------------
\end{verbatim}
\begin{verbatim}
 [1] "has_tidy_media"          "has_tidy_images"        
 [3] "has_tidy_code"           "has_tidy_raw_data"      
 [5] "has_tidy_data"           "has_tidy_scripts"       
 [7] "has_readme"              "has_no_lint"            
 [9] "has_proj_root"           "has_no_nested_proj_root"
[11] "has_only_used_files"     "has_clear_build_chain"  
[13] "has_no_absolute_paths"   "has_only_portable_paths"
[15] "has_no_randomness"      
\end{verbatim}
Subsets of the fifteen tests can be invoked using the
\texttt{tidyselect} helper functions (Henry \& Wickham (2020)) in
combination with the more limited \texttt{proj\_check\_some()} function.
\begin{Shaded}
\begin{Highlighting}[]
\NormalTok{proj_dir <-}\StringTok{ "project_miceps"}
\end{Highlighting}
\end{Shaded}
\begin{Shaded}
\begin{Highlighting}[]
\KeywordTok{proj_check_some}\NormalTok{(proj_dir, }\KeywordTok{contains}\NormalTok{(}\StringTok{"paths"}\NormalTok{))}
\end{Highlighting}
\end{Shaded}
\begin{verbatim}
-- Compiling... ------------------------------------- fertile 0.0.0.9027 --
\end{verbatim}
\begin{verbatim}
-- Rendering R scripts... --------------------------- fertile 0.0.0.9027 --
\end{verbatim}
\begin{verbatim}
-- Running reproducibility checks ------------------- fertile 0.0.0.9027 --
\end{verbatim}
\begin{verbatim}
v Checking for no absolute paths
\end{verbatim}
\begin{verbatim}
v Checking for only portable paths
\end{verbatim}
\begin{verbatim}
-- Summary of fertile checks ------------------------ fertile 0.0.0.9027 --
\end{verbatim}
\begin{verbatim}
v Reproducibility checks passed: 2
\end{verbatim}
Each test can also be run individually by calling the function matching
its check name.

The \texttt{proj\_analyze()} function creates a report documenting the
structure of a data analysis project. This report contains information
about all packages referenced in code, the files present in the
directory and their types, suggestions for moving files to create a more
organized structure, and a list of reproducibility-breaking file paths
used in code.
\begin{Shaded}
\begin{Highlighting}[]
\KeywordTok{proj_analyze}\NormalTok{(proj_dir)}
\end{Highlighting}
\end{Shaded}
\begin{verbatim}
-- Analysis of reproducibility for project_miceps --- fertile 0.0.0.9027 --
\end{verbatim}
\begin{verbatim}
--   Packages referenced in source code ------------- fertile 0.0.0.9027 --
\end{verbatim}
\begin{verbatim}
# A tibble: 9 x 3
  package       N used_in                    
  <chr>     <int> <chr>                      
1 broom         1 project_miceps/analysis.Rmd
2 dplyr         1 project_miceps/analysis.Rmd
3 ggplot2       1 project_miceps/analysis.Rmd
4 purrr         1 project_miceps/analysis.Rmd
5 readr         1 project_miceps/analysis.Rmd
6 rmarkdown     1 project_miceps/analysis.Rmd
7 skimr         1 project_miceps/analysis.Rmd
8 stargazer     1 project_miceps/analysis.Rmd
9 tidyr         1 project_miceps/analysis.Rmd
\end{verbatim}
\begin{verbatim}
--   Files present in directory --------------------- fertile 0.0.0.9027 --
\end{verbatim}
\begin{verbatim}
                     file   ext    size
1 Estrogen_Receptors.docx  docx  10.97K
2      citrate_v_time.png   png 187.79K
3     proteins_v_time.png   png 378.51K
4   Blot_data_updated.csv   csv  14.43K
5      CS_data_redone.csv   csv   7.39K
6                mice.csv   csv  14.33K
7               README.md    md      39
8            miceps.Rproj Rproj     204
9            analysis.Rmd   Rmd   4.94K
                                                                     mime
1 application/vnd.openxmlformats-officedocument.wordprocessingml.document
2                                                               image/png
3                                                               image/png
4                                                                text/csv
5                                                                text/csv
6                                                                text/csv
7                                                           text/markdown
8                                                            text/rstudio
9                                                         text/x-markdown
\end{verbatim}
\begin{verbatim}
--   Suggestions for moving files ------------------- fertile 0.0.0.9027 --
\end{verbatim}
\begin{verbatim}
                 path_rel    dir_rel
1   Blot_data_updated.csv   data-raw
2      CS_data_redone.csv   data-raw
3 Estrogen_Receptors.docx inst/other
4            analysis.Rmd  vignettes
5      citrate_v_time.png inst/image
6                mice.csv   data-raw
7     proteins_v_time.png inst/image
                                                                                               cmd
1     file_move('project_miceps/Blot_data_updated.csv', fs::dir_create('project_miceps/data-raw'))
2        file_move('project_miceps/CS_data_redone.csv', fs::dir_create('project_miceps/data-raw'))
3 file_move('project_miceps/Estrogen_Receptors.docx', fs::dir_create('project_miceps/inst/other'))
4             file_move('project_miceps/analysis.Rmd', fs::dir_create('project_miceps/vignettes'))
5      file_move('project_miceps/citrate_v_time.png', fs::dir_create('project_miceps/inst/image'))
6                  file_move('project_miceps/mice.csv', fs::dir_create('project_miceps/data-raw'))
7     file_move('project_miceps/proteins_v_time.png', fs::dir_create('project_miceps/inst/image'))
\end{verbatim}
\begin{verbatim}
--   Problematic paths logged ----------------------- fertile 0.0.0.9027 --
\end{verbatim}
\begin{verbatim}
NULL
\end{verbatim}
\subsection{Logging}\label{logging}

\texttt{fertile} also contains logging functionality, which records
commands run in the console that have the potential to affect
reproducibility, enabling users to look at their past history at any
time. The package focuses mostly on package loading and file opening,
noting which function was used, the path or package it referenced, and
the timestamp at which that event happened. Users can access the log
recording their commands at any time via the \texttt{log\_report()}
function:
\begin{Shaded}
\begin{Highlighting}[]
\KeywordTok{log_report}\NormalTok{()}
\end{Highlighting}
\end{Shaded}
\begin{verbatim}
# A tibble: 6 x 4
  path        path_abs                    func     timestamp          
  <chr>       <chr>                       <chr>    <dttm>             
1 package:re~ <NA>                        base::r~ 2020-09-03 16:59:05
2 package:th~ <NA>                        base::r~ 2020-09-03 16:59:05
3 package:th~ <NA>                        base::l~ 2020-09-03 16:59:05
4 package:pu~ <NA>                        base::l~ 2020-09-04 14:57:00
5 package:fo~ <NA>                        base::l~ 2020-09-04 14:57:00
6 project_mi~ /Users/audreybertin/Docume~ readr::~ 2020-09-04 14:57:00
\end{verbatim}
The log, if not managed, can grow very long over time. For users who do
not desire such functionality, \texttt{log\_clear()} provides a way to
erase the log and start over.

\subsection{Utility Functions}\label{utility-functions}

\texttt{fertile} also provides several useful utility functions that may
assist with the process of data analysis.

\subsection{File Path Management}\label{file-path-management}

The \texttt{check\_path()} function analyzes a vector of paths (or a
single path) to determine whether there are any absolute paths or paths
that lead outside the project directory.
\begin{Shaded}
\begin{Highlighting}[]
\CommentTok{# Path inside the directory}
\KeywordTok{check_path}\NormalTok{(}\StringTok{"project_miceps"}\NormalTok{)}
\end{Highlighting}
\end{Shaded}
\begin{verbatim}
# A tibble: 0 x 3
# ... with 3 variables: path <chr>, problem <chr>, solution <chr>
\end{verbatim}
\begin{Shaded}
\begin{Highlighting}[]
\CommentTok{# Absolute path (current working directory)}
\KeywordTok{check_path}\NormalTok{(}\KeywordTok{getwd}\NormalTok{())}
\end{Highlighting}
\end{Shaded}
\begin{verbatim}
Error: Detected absolute paths
\end{verbatim}
\begin{Shaded}
\begin{Highlighting}[]
\CommentTok{# Path outside the directory}
\KeywordTok{check_path}\NormalTok{(}\StringTok{"../fertile.Rmd"}\NormalTok{)}
\end{Highlighting}
\end{Shaded}
\begin{verbatim}
Error: Detected paths that lead outside the project directory
\end{verbatim}
\subsection{File Types}\label{file-types}

There are several functions that can be used to check the type of a
file:
\begin{Shaded}
\begin{Highlighting}[]
\KeywordTok{is_data_file}\NormalTok{(fs}\OperatorTok{::}\KeywordTok{path}\NormalTok{(proj_dir, }\StringTok{"mice.csv"}\NormalTok{))}
\end{Highlighting}
\end{Shaded}
\begin{verbatim}
[1] TRUE
\end{verbatim}
\begin{Shaded}
\begin{Highlighting}[]
\KeywordTok{is_image_file}\NormalTok{(fs}\OperatorTok{::}\KeywordTok{path}\NormalTok{(proj_dir, }\StringTok{"proteins_v_time.png"}\NormalTok{))}
\end{Highlighting}
\end{Shaded}
\begin{verbatim}
[1] TRUE
\end{verbatim}
\begin{Shaded}
\begin{Highlighting}[]
\KeywordTok{is_text_file}\NormalTok{(fs}\OperatorTok{::}\KeywordTok{path}\NormalTok{(proj_dir, }\StringTok{"README.md"}\NormalTok{))}
\end{Highlighting}
\end{Shaded}
\begin{verbatim}
[1] TRUE
\end{verbatim}
\begin{Shaded}
\begin{Highlighting}[]
\KeywordTok{is_r_file}\NormalTok{(fs}\OperatorTok{::}\KeywordTok{path}\NormalTok{(proj_dir, }\StringTok{"analysis.Rmd"}\NormalTok{))}
\end{Highlighting}
\end{Shaded}
\begin{verbatim}
[1] TRUE
\end{verbatim}
\subsection{Temporary Directories}\label{temporary-directories}

The \texttt{sandbox()} function allows the user to make a copy of their
project in a temporary directory. This can be useful for ensuring that
projects run properly when access to the local file system is removed.
\begin{Shaded}
\begin{Highlighting}[]
\NormalTok{proj_dir}
\end{Highlighting}
\end{Shaded}
\begin{verbatim}
[1] "project_miceps"
\end{verbatim}
\begin{Shaded}
\begin{Highlighting}[]
\NormalTok{fs}\OperatorTok{::}\KeywordTok{dir_ls}\NormalTok{(proj_dir) }\OperatorTok\StringTok{ }\KeywordTok{head}\NormalTok{(}\DecValTok{3}\NormalTok{)}
\end{Highlighting}
\end{Shaded}
\begin{verbatim}
project_miceps/Blot_data_updated.csv
project_miceps/CS_data_redone.csv
project_miceps/Estrogen_Receptors.docx
\end{verbatim}
\begin{Shaded}
\begin{Highlighting}[]
\NormalTok{temp_dir <-}\StringTok{ }\KeywordTok{sandbox}\NormalTok{(proj_dir)}
\NormalTok{temp_dir}
\end{Highlighting}
\end{Shaded}
\begin{verbatim}
/var/folders/v6/f62qz88s0sd5n3yqw9d8sb300000gn/T/Rtmp02CNqs/project_miceps
\end{verbatim}
\begin{Shaded}
\begin{Highlighting}[]
\NormalTok{fs}\OperatorTok{::}\KeywordTok{dir_ls}\NormalTok{(temp_dir) }\OperatorTok\StringTok{ }\KeywordTok{head}\NormalTok{(}\DecValTok{3}\NormalTok{)}
\end{Highlighting}
\end{Shaded}
\begin{verbatim}
/var/folders/v6/f62qz88s0sd5n3yqw9d8sb300000gn/T/Rtmp02CNqs/project_miceps/Blot_data_updated.csv
/var/folders/v6/f62qz88s0sd5n3yqw9d8sb300000gn/T/Rtmp02CNqs/project_miceps/CS_data_redone.csv
/var/folders/v6/f62qz88s0sd5n3yqw9d8sb300000gn/T/Rtmp02CNqs/project_miceps/Estrogen_Receptors.docx
\end{verbatim}
\subsection{Managing Project
Dependencies}\label{managing-project-dependencies}

One of the challenges with ensuring that work is reproducible is the
issue of dependencies. Many data analysis projects reference a variety
of \texttt{R} packages in their code. When such projects are shared with
other users who may not have the required packages downloaded, it can
cause errors that prevent the project from running properly.

The \texttt{proj\_pkg\_script()}\} function assists with this issue by
making it simple and fast to download dependencies. When run on an
\texttt{R} project directory, the function creates a \texttt{.R} script
file that contains the code needed to install all of the packages
referenced in the project, differentiating between packages located on
CRAN and those located on GitHub.
\begin{Shaded}
\begin{Highlighting}[]
\NormalTok{install_script <-}\StringTok{ }\KeywordTok{proj_pkg_script}\NormalTok{(proj_dir)}
\KeywordTok{cat}\NormalTok{(}\KeywordTok{readChar}\NormalTok{(install_script, }\FloatTok{1e5}\NormalTok{))}
\end{Highlighting}
\end{Shaded}
\begin{verbatim}
# Run this script to install the required packages for this R project.
# Packages hosted on CRAN...
install.packages(c( 'broom', 'dplyr', 'ggplot2', 'purrr', 'readr', 'rmarkdown', 'skimr', 'stargazer', 'tidyr' ))
# Packages hosted on GitHub...
\end{verbatim}
\section{\texorpdfstring{How \texttt{fertile}
Works}{How fertile Works}}\label{how-fertile-works}

Much of the functionality in \texttt{fertile} is achieved by writing
\texttt{shims} \textbf{link to wikipedia page here}. \texttt{fertile}'s
shimmed functions intercept the user's commands and perform various
logging and checking tasks before executing the desired function. Our
process is:
\begin{enumerate}
\def\labelenumi{\arabic{enumi}.}
\item
  Identify an \texttt{R} function that is likely to be involved in
  operations that may break reproducibility. Popular functions
  associated with only one package (e.g., \texttt{read\_csv()} from
  \texttt{readr}) are ideal candidates.
\item
  Create a function in \texttt{fertile} with the same name that takes
  the same arguments (and always the dots \texttt{...}).
\item
  Write this new function so that it:
\end{enumerate}
\begin{enumerate}
\def\labelenumi{\alph{enumi})}
\tightlist
\item
  captures any arguments,
\item
  logs the name of the function called,
\item
  performs any checks on these arguments, and
\item
  calls the original function with the original arguments. Except where
  warranted, the execution looks the same to the user as if they were
  calling the original function.
\end{enumerate}
Most shims are quite simple and look something like what is shown below
for \texttt{read\_csv()}.
\begin{Shaded}
\begin{Highlighting}[]
\NormalTok{fertile}\OperatorTok{::}\NormalTok{read_csv}
\end{Highlighting}
\end{Shaded}
\begin{verbatim}
function(file, ...) {
  if (interactive_log_on()) {
    log_push(file, "readr::read_csv")
    check_path_safe(file)
    readr::read_csv(file, ...)
  }
}
<bytecode: 0x7fa30fa51ea0>
<environment: namespace:fertile>
\end{verbatim}
\texttt{fertile} shims many common functions, including those that read
in a variety of data types, write data, and load packages. This works
both proactively and retroactively, as the shimmed functions written in
\texttt{fertile} are activated both when the user is coding
interactively and when a file containing code is rendered.

In order to ensure that the \texttt{fertile} versions of functions
(``shims'') always supersede (``mask'') their original namesakes when
called, \texttt{fertile} uses its own shims of the \texttt{library} and
\texttt{require} functions to manipulate the \texttt{R} \texttt{search}
path so that it is always located in the first position. In the
\texttt{fertile} version of \texttt{library()}, we detach
\texttt{fertile} from the search path, load the requested package, and
then re-attach \texttt{fertile}. This ensures that when a user executes
a command, \texttt{R} will check \texttt{fertile} for a matching
function before considering other packages. While it is possible that
this shifty behavior could lead to unintended consequences, our goal is
to catch a good deal of problems before they become problematic. Users
can easily disable \texttt{fertile} by detaching it, or not loading it
in the first place.

\section{\texorpdfstring{\texttt{fertile} in Practice: Experimental
Results From Smith College Student
Use}{fertile in Practice: Experimental Results From Smith College Student Use}}\label{fertile-in-practice-experimental-results-from-smith-college-student-use}

\texttt{fertile} is designed to: 1) be simple enough that users with
minimal \texttt{R} experience can use the package without issue, 2)
increase the reproducibility of work produced by its users, and 3)
educate its users on why their work is or is not reproducible and
provide guidance on how to address any problems.

To test \texttt{fertile}'s effectiveness, we began an initial randomized
control trial of the package on an introductory undergraduate data
science course at Smith College in Spring 2020 \textbf{ADD FOOTNOTE}
(This study was approved by Smith College IRB, Protocol \#19-032).

The experiment was structured as follows:

1.Students are given a form at the start of the semester asking whether
they consent to participate in a study on data science education. In
order to successfully consent, they must provide their system username,
collected through the command \texttt{Sys.getenv("LOGNAME")}. To
maintain privacy the results are then transformed into a hexadecimal
string via the \texttt{md5()} hashing function.
\begin{enumerate}
\def\labelenumi{\arabic{enumi}.}
\setcounter{enumi}{1}
\item
  These hexadecimal strings are then randomly assigned into equally
  sized groups, one experimental group that receives the features of
  \texttt{fertile} and one group that receives a control.
\item
  The students are then asked to download a package called
  \texttt{sds192} (the course number and prefix), which was created for
  the purpose of this trial. It leverages an \texttt{.onAttach()}
  function to scan the \texttt{R} environment and collect the username
  of the user who is loading the package and run it through the same
  hashing algorithm as used previously. It then identifies whether that
  user belongs to the experimental or the control group. Depending on
  the group they are in, they receive a different version of the
  package.
\item
  The experimental group receives the basic \texttt{sds192} package,
  which consists of some data sets and \texttt{R} Markdown templates
  necessary for completing homework assignments and projects in the
  class, but also has \texttt{fertile} installed and loaded silently in
  the background. The package's proactive features are enabled, and
  therefore users will receive warning messages when they use absolute
  or non-portable paths or attempt to change their working directory.
  The control group receives only the basic \texttt{sds192} package,
  including its data sets and \texttt{R} Markdown templates. All
  students from both groups then use their version of the package
  throughout the semester on a variety of projects.
\item
  Both groups are given a short quiz on different components of
  reproducibility that are intended to be taught by \texttt{fertile} at
  both the beginning and end of the semester. Their scores are then
  compared to see whether one group learned more than the other group or
  whether their scores were essentially equivalent. Additionally, for
  every homework assignment submitted, the professor takes note of
  whether or not the project compiles successfully.
\end{enumerate}
Based on the results, we hope to determine whether \texttt{fertile} was
successful at achieving its intended goals. A lack of notable difference
between the \emph{experimental} and \emph{control} groups in terms of
the number of code-related questions asked throughout the semester would
indicate that \texttt{fertile} achieved its goal of simplicity. A higher
average for the \emph{experimental} group in terms of the number of
homework assignments that compiled successfully would indicate that
\texttt{fertile} was successful in increasing reproducibility. A greater
increase over the semester in the reproducibility quiz scores for
students in the \emph{experimental} group compared with the
\emph{control} group would indicate that \texttt{fertile} achieved its
goal of educating users on reproducibility. Success according to these
metrics would provide evidence showing \texttt{fertile}'s benefit as
tool to help educators introduce reproducibility concepts in the
classroom.

\chapter{Incorporating Reproducibility Tools Into The Greater Data
Science Community}\label{applications}

\section{\texorpdfstring{Potential Applications of
\texttt{fertile}}{Potential Applications of fertile}}\label{potential-applications-of-fertile}

\subsection{In Journal Review}\label{in-journal-review}

\subsection{By Beginning Data
Scientists}\label{by-beginning-data-scientists}

\subsection{By Advanced Data
Scientists}\label{by-advanced-data-scientists}

\subsection{For Teaching
Reproducibily}\label{for-teaching-reproducibily}

\section{\texorpdfstring{Integration Of \texttt{fertile} And Other
Reproducibility Tools in Data Science
Education}{Integration Of fertile And Other Reproducibility Tools in Data Science Education}}\label{integration-of-fertile-and-other-reproducibility-tools-in-data-science-education}

\chapter*{Conclusion}\label{conclusion}
\addcontentsline{toc}{chapter}{Conclusion}

\texttt{fertile} is an \texttt{R} package that lowers barriers to
reproducible data analysis projects in \texttt{R}, providing a wide
array of checks and suggestions addressing many different aspects of
project reproducibility, including file organization, file path usage,
documentation, and dependencies. \texttt{fertile} is meant to be
educational, providing informative error messages that indicate why
users' mistakes are problematic and sharing recommendations on how to
fix them. The package is designed in this way so as to promote a greater
understanding of reproducibility concepts in its users, with the goal of
increasing the overall awareness and understanding of reproducibility in
the \texttt{R} community.

The package has very low barriers to entry, making it accessible to
users with various levels of background knowledge. Unlike many other
\texttt{R} packages focused on reproducibility that are currently
available, the features of \texttt{fertile} can be accessed almost
effortlessly. Many of the retroactive features can be accessed in only
two lines of code requiring minimal arguments and some of the proactive
features can be accessed with no additional effort beyond loading the
package. This, in combination with the fact that \texttt{fertile} does
not focus on one specific area of reproducibility, instead covering
(albeit in less detail) a wide variety of topics, means that
\texttt{fertile} makes it easy for data analysts of all skill levels to
quickly gain a better understanding of the reproducibility of the work.

In the moment, it often feels easiest to take a shortcut---to use an
absolute path or change a working directory. However, when considering
the long term path of a project, spending the extra time to improve
reproducibility is worthwhile. \texttt{fertile}'s user-friendly features
can help data analysts avoid these harmful shortcuts with minimal
effort.

\appendix

\chapter{The First Appendix}\label{the-first-appendix}

This first appendix includes all of the R chunks of code that were
hidden throughout the document (using the \texttt{include\ =\ FALSE}
chunk tag) to help with readibility and/or setup.

\textbf{In the main Rmd file}
\begin{Shaded}
\begin{Highlighting}[]
\CommentTok{# This chunk ensures that the thesisdown package is}
\CommentTok{# installed and loaded. This thesisdown package includes}
\CommentTok{# the template files for the thesis.}
\ControlFlowTok{if}\NormalTok{ (}\OperatorTok{!}\KeywordTok{require}\NormalTok{(remotes)) \{}
  \ControlFlowTok{if}\NormalTok{ (params}\OperatorTok{$}\StringTok{`}\DataTypeTok{Install needed packages for \{thesisdown\}}\StringTok{`}\NormalTok{) \{}
    \KeywordTok{install.packages}\NormalTok{(}\StringTok{"remotes"}\NormalTok{, }\DataTypeTok{repos =} \StringTok{"https://cran.rstudio.com"}\NormalTok{)}
\NormalTok{  \} }\ControlFlowTok{else}\NormalTok{ \{}
    \KeywordTok{stop}\NormalTok{(}
      \KeywordTok{paste}\NormalTok{(}\StringTok{'You need to run install.packages("remotes")",}
\StringTok{            "first in the Console.'}\NormalTok{)}
\NormalTok{    )}
\NormalTok{  \}}
\NormalTok{\}}
\ControlFlowTok{if}\NormalTok{ (}\OperatorTok{!}\KeywordTok{require}\NormalTok{(thesisdown)) \{}
  \ControlFlowTok{if}\NormalTok{ (params}\OperatorTok{$}\StringTok{`}\DataTypeTok{Install needed packages for \{thesisdown\}}\StringTok{`}\NormalTok{) \{}
\NormalTok{    remotes}\OperatorTok{::}\KeywordTok{install_github}\NormalTok{(}\StringTok{"ismayc/thesisdown"}\NormalTok{)}
\NormalTok{  \} }\ControlFlowTok{else}\NormalTok{ \{}
    \KeywordTok{stop}\NormalTok{(}
      \KeywordTok{paste}\NormalTok{(}
        \StringTok{"You need to run"}\NormalTok{,}
        \StringTok{'remotes::install_github("ismayc/thesisdown")'}\NormalTok{,}
        \StringTok{"first in the Console."}
\NormalTok{      )}
\NormalTok{    )}
\NormalTok{  \}}
\NormalTok{\}}
\KeywordTok{library}\NormalTok{(thesisdown)}
\CommentTok{# Set how wide the R output will go}
\KeywordTok{options}\NormalTok{(}\DataTypeTok{width =} \DecValTok{70}\NormalTok{)}
\end{Highlighting}
\end{Shaded}
\textbf{In Chapter \ref{ref-labels}:}

\chapter{The Second Appendix, for
Fun}\label{the-second-appendix-for-fun}

\backmatter

\chapter*{References}\label{references}
\addcontentsline{toc}{chapter}{References}

\markboth{References}{References}

\noindent

\setlength{\parindent}{-0.20in} \setlength{\leftskip}{0.20in}
\setlength{\parskip}{8pt}

\hypertarget{refs}{}
\hypertarget{ref-R-workflowr}{}
Blischak, J., Carbonetto, P., \& Stephens, M. (2019). Workflowr: A
framework for reproducible and collaborative data science. Retrieved
from \url{https://CRAN.R-project.org/package=workflowr}

\hypertarget{ref-broman}{}
Broman, K. (2019). Initial steps toward reproducible research: Organize
your data and code. \emph{Sitewide ATOM}. Retrieved from
\url{https://kbroman.org/steps2rr/pages/organize.html}

\hypertarget{ref-cooper2017guide}{}
Cooper, N., Hsing, P.-Y., Croucher, M., Graham, L., James, T.,
Krystalli, A., \& Michonneau, F. (2017). A guide to reproducible code in
ecology and evolution. \emph{British Ecological Society}. Retrieved from
\url{https://www.britishecologicalsociety.org/wp-content/uploads/2017/12/guide-to-reproducible-code.pdf}

\hypertarget{ref-eisner-reproducibility}{}
Eisner, D. A. (2018). Reproducibility of science: Fraud, impact factors
and carelessness. \emph{Journal of Molecular and Cellular Cardiology},
\emph{114}, 364--368.
\url{http://doi.org/https://doi.org/10.1016/j.yjmcc.2017.10.009}

\hypertarget{ref-sep-scientific-reproducibility}{}
Fidler, F., \& Wilcox, J. (2018). Reproducibility of scientific results.
In E. N. Zalta (Ed.), \emph{The stanford encyclopedia of philosophy}
(Winter 2018).
\url{https://plato.stanford.edu/archives/win2018/entries/scientific-reproducibility/};
Metaphysics Research Lab, Stanford University.

\hypertarget{ref-R-orderly}{}
FitzJohn, R., Ashton, R., Hill, A., Eden, M., Hinsley, W., Russell, E.,
\& Thompson, J. (2020). Orderly: Lightweight reproducible reporting.
Retrieved from \url{https://CRAN.R-project.org/package=orderly}

\hypertarget{ref-unix}{}
Gancarz, M. (2003). \emph{Linux and the unix philosophy} (2nd ed.).
Woburn, MA: Digital Press.

\hypertarget{ref-Goodman341ps12}{}
Goodman, S. N., Fanelli, D., \& Ioannidis, J. P. A. (2016). What does
research reproducibility mean? \emph{Science Translational Medicine},
\emph{8}(341), 1--6. \url{http://doi.org/10.1126/scitranslmed.aaf5027}

\hypertarget{ref-bioessays-gosselin}{}
Gosselin, R.-D. (2020). Statistical analysis must improve to address the
reproducibility crisis: The access to transparent statistics (acts) call
to action. \emph{BioEssays}, \emph{42}(1), 1900189.
\url{http://doi.org/10.1002/bies.201900189}

\hypertarget{ref-R-tidyselect}{}
Henry, L., \& Wickham, H. (2020). Tidyselect: Select from a set of
strings. Retrieved from
\url{https://CRAN.R-project.org/package=tidyselect}

\hypertarget{ref-hermans2017programming}{}
Hermans, F., \& Aldewereld, M. (2017). Programming is writing is
programming. In \emph{Companion to the first international conference on
the art, science and engineering of programming} (pp. 1--8).

\hypertarget{ref-kitzes2017practice}{}
Kitzes, J., Turek, D., \& Deniz, F. (2017). \emph{The practice of
reproducible research: Case studies and lessons from the data-intensive
sciences}. Berkeley, CA: University of California Press. Retrieved from
\url{https://www.practicereproducibleresearch.org}

\hypertarget{ref-r-opensci}{}
Martinez, C., Hollister, J., Marwick, B., Szöcs, E., Zeitlin, S.,
Kinoshita, B. P., \ldots{} Meinke, B. (2018). Reproducibility in
Science: A Guide to enhancing reproducibility in scientific results and
writing. Retrieved from
\url{http://ropensci.github.io/reproducibility-guide/}

\hypertarget{ref-R-rrtools}{}
Marwick, B. (2019). Rrtools: Creates a reproducible research compendium.
Retrieved from \url{https://github.com/benmarwick/rrtools}

\hypertarget{ref-marwick2018packaging}{}
Marwick, B., Boettiger, C., \& Mullen, L. (2018). Packaging data
analytical work reproducibly using R (and friends). \emph{The American
Statistician}, \emph{72}(1), 80--88.
\url{http://doi.org/doi.org/10.1080/00031305.2017.1375986}

\hypertarget{ref-engineering-reproducibility}{}
McArthur, S. L. (2019). Repeatability, reproducibility, and
replicability: Tackling the 3R challenge in biointerface science and
engineering. \emph{Biointerphases}, \emph{14}(2), 1--2.
\url{http://doi.org/10.1116/1.5093621}

\hypertarget{ref-R-reproducible}{}
McIntire, E. J. B., \& Chubaty, A. M. (2020). Reproducible: A set of
tools that enhance reproducibility beyond package management. Retrieved
from \url{https://CRAN.R-project.org/package=reproducible}

\hypertarget{ref-R-drake}{}
OpenSci, R. (2020). Drake: A pipeline toolkit for reproducible
computation at scale. Retrieved from
\url{https://cran.r-project.org/package=drake}

\hypertarget{ref-wercker}{}
Oracle Corporation. (2019). Wercker. Retrieved from
\url{https://github.com/wercker/wercker}

\hypertarget{ref-coreteam-extensions}{}
R-Core-Team. (2020). Writing r extensions. \emph{R Foundation for
Statistical Computing}. Retrieved from
\url{http://cran.stat.unipd.it/doc/manuals/r-release/R-exts.pdf}

\hypertarget{ref-R-checkers}{}
Ross, N., DeCicco, L., \& Randhawa, N. (2018). Checkers: Automated
checking of best practices for research compendia. Retrieved from
\url{https://github.com/ropenscilabs/checkers/blob/master/DESCRIPTIONr}

\hypertarget{ref-R-packrat}{}
Ushey, K., McPherson, J., Cheng, J., Atkins, A., \& Allaire, J. (2018).
Packrat: A dependency management system for projects and their r package
dependencies. Retrieved from
\url{https://CRAN.R-project.org/package=packrat}

\hypertarget{ref-plos-biology}{}
Wallach, J. D., Boyack, K. W., \& Ioannidis, J. P. A. (2018).
Reproducible research practices, transparency, and open access data in
the biomedical literature, 2015-2017. \emph{PLOS Biology},
\emph{16}(11), 1--20. \url{http://doi.org/10.1371/journal.pbio.2006930}

\hypertarget{ref-hadley-packages}{}
Wickham, H. (2015). \emph{R packages} (1st ed.). Sebastopol, CA:
O'Reilly Media, Inc.


% Index?

\end{document}
