% This is the Reed College LaTeX thesis template. Most of the work
% for the document class was done by Sam Noble (SN), as well as this
% template. Later comments etc. by Ben Salzberg (BTS). Additional
% restructuring and APA support by Jess Youngberg (JY).
% Your comments and suggestions are more than welcome; please email
% them to cus@reed.edu
%
% See https://www.reed.edu/cis/help/LaTeX/index.html for help. There are a
% great bunch of help pages there, with notes on
% getting started, bibtex, etc. Go there and read it if you're not
% already familiar with LaTeX.
%
% Any line that starts with a percent symbol is a comment.
% They won't show up in the document, and are useful for notes
% to yourself and explaining commands.
% Commenting also removes a line from the document;
% very handy for troubleshooting problems. -BTS

% As far as I know, this follows the requirements laid out in
% the 2002-2003 Senior Handbook. Ask a librarian to check the
% document before binding. -SN

%%
%% Preamble
%%
% \documentclass{<something>} must begin each LaTeX document
\documentclass[12pt,twoside]{reedthesis}
% Packages are extensions to the basic LaTeX functions. Whatever you
% want to typeset, there is probably a package out there for it.
% Chemistry (chemtex), screenplays, you name it.
% Check out CTAN to see: https://www.ctan.org/
%%
\usepackage{graphicx,latexsym}
\usepackage{amsmath}
\usepackage{amssymb,amsthm}
\usepackage{longtable,booktabs,setspace}
\usepackage{chemarr} %% Useful for one reaction arrow, useless if you're not a chem major
\usepackage[hyphens]{url}
% Added by CII
\usepackage{hyperref}
\usepackage{lmodern}
\usepackage{float}
\floatplacement{figure}{H}
% End of CII addition
\usepackage{rotating}

% Next line commented out by CII
%%% \usepackage{natbib}
% Comment out the natbib line above and uncomment the following two lines to use the new
% biblatex-chicago style, for Chicago A. Also make some changes at the end where the
% bibliography is included.
%\usepackage{biblatex-chicago}
%\bibliography{thesis}


% Added by CII (Thanks, Hadley!)
% Use ref for internal links
\renewcommand{\hyperref}[2][???]{\autoref{#1}}
\def\chapterautorefname{Chapter}
\def\sectionautorefname{Section}
\def\subsectionautorefname{Subsection}
% End of CII addition

% Added by CII
\usepackage{caption}
\captionsetup{width=5in}
% End of CII addition

% \usepackage{times} % other fonts are available like times, bookman, charter, palatino

% Syntax highlighting #22

% To pass between YAML and LaTeX the dollar signs are added by CII
\title{My Final College Paper}
\author{Audrey M. Bertin}
% The month and year that you submit your FINAL draft TO THE LIBRARY (May or December)
\date{May 2021}
\division{Statistical and Data Sciences}
\advisor{Benjamin S. Baumer}
\institution{Smith College}
\degree{Bachelor of Arts}
%If you have two advisors for some reason, you can use the following
% Uncommented out by CII
% End of CII addition

%%% Remember to use the correct department!
\department{Statistical and Data Sciences}
% if you're writing a thesis in an interdisciplinary major,
% uncomment the line below and change the text as appropriate.
% check the Senior Handbook if unsure.
%\thedivisionof{The Established Interdisciplinary Committee for}
% if you want the approval page to say "Approved for the Committee",
% uncomment the next line
%\approvedforthe{Committee}

% Added by CII
%%% Copied from knitr
%% maxwidth is the original width if it's less than linewidth
%% otherwise use linewidth (to make sure the graphics do not exceed the margin)
\makeatletter
\def\maxwidth{ %
  \ifdim\Gin@nat@width>\linewidth
    \linewidth
  \else
    \Gin@nat@width
  \fi
}
\makeatother

%Added by @MyKo101, code provided by @GerbrichFerdinands

\renewcommand{\contentsname}{Table of Contents}
% End of CII addition

\setlength{\parskip}{0pt}

% Added by CII

\providecommand{\tightlist}{%
  \setlength{\itemsep}{0pt}\setlength{\parskip}{0pt}}

\Acknowledgements{
I want to thank a few people.
}

\Dedication{
You can have a dedication here if you wish.
}

\Preface{
This is an example of a thesis setup to use the reed thesis document
class (for LaTeX) and the R bookdown package, in general.
}

\Abstract{
The preface pretty much says it all. \par
Second paragraph of abstract starts here.
}

% End of CII addition
%%
%% End Preamble
%%
%
\begin{document}

% Everything below added by CII
  \maketitle

\frontmatter % this stuff will be roman-numbered
\pagestyle{empty} % this removes page numbers from the frontmatter
  \begin{acknowledgements}
    I want to thank a few people.
  \end{acknowledgements}
  \begin{preface}
    This is an example of a thesis setup to use the reed thesis document
    class (for LaTeX) and the R bookdown package, in general.
  \end{preface}
  \hypersetup{linkcolor=black}
  \setcounter{tocdepth}{2}
  \tableofcontents

  \listoftables

  \listoffigures
  \begin{abstract}
    The preface pretty much says it all. \par
    Second paragraph of abstract starts here.
  \end{abstract}
  \begin{dedication}
    You can have a dedication here if you wish.
  \end{dedication}
\mainmatter % here the regular arabic numbering starts
\pagestyle{fancyplain} % turns page numbering back on

\chapter{If you have more two advisors, un-silence line
7}\label{if-you-have-more-two-advisors-un-silence-line-7}

Placeholder

\chapter{An Introduction to Reproducibility}\label{reproducibility}

\section{What is reproducibility?}\label{what-is-reproducibility}

\section{Why is reproducibility
important?}\label{why-is-reproducibility-important}

\section{Current Discussion on Reproducibility in
Academia}\label{current-discussion-on-reproducibility-in-academia}

\chapter{Addressing the Challenges of
Reproducibility}\label{my-solution}

\section{Review Of Previous Work}\label{review-of-previous-work}

\subsection{Literature}\label{literature}

Publications on reproducibility can be found in all areas of scientific
research. However, as Goodman, Fanelli, \& Ioannidis (2016) argue, the
language and conceptual framework of research reproducibility varies
significantly across the sciences, and there are no clear standards on
reproducibility agreed upon by the scientific community as a whole. We
consider recommendations from a variety of fields and determine the key
aspects of reproducibility faced by scientists in different disciplines.

Kitzes, Turek, \& Deniz (2017) present a collection of case studies on
reproducibility practices from across the data-intensive sciences,
illustrating a variety of recommendations and techniques for achieving
reproducibility. Although their work does not come to a consensus on the
exact standards of reproducibility that should be followed, several
common trends and principles emerge from their case studies: 1) use
clear separation, labeling, and documentation, 2) automate processes
when possible, and 3) design the data analysis workflow as a sequence of
small steps glued together, with outputs from one step serving as inputs
into the next. This is a common suggestion within the computing
community, originating as part of the Unix philosophy (Gancarz (2003)).

Cooper et al. (2017) focus on data analysis in \texttt{R} and identify a
similar list of important reproducibility components, reinforcing the
need for clearly labeled, well-documented, and well-separated files. In
addition, they recommend publishing a list of dependencies and using
version control. Broman (2019) reiterates the need for clear naming and
file separation while sharing several additional suggestions: keep the
project contained in one directory, use relative paths, and include a
\texttt{README}.

The reproducibility recommendations from R OpenSci, a non-profit
initiative founded in 2011 to make scientific data retrieval
reproducible, share similar principles to those discussed previously.
They focus on a need for a well-developed file system, with no
extraneous files and clear labeling. They also reiterate the need to
note dependencies and use automation when possible, while making clear a
suggestion not present in the previously-discussed literature: the need
to use seeds, which allow for the saving and restoring of the random
number generator state, when running code involving randomness (Martinez
et al. (2018)).

When considered in combination, these sources provide a well-rounded
picture of the components important to research reproducibility. Using
this literature as a guideline, we identify several key features of
reproducible work. These recommendations are a matter of opinion---due
to the lack of agreement on which components of reproducibility are most
important, we select those that are mentioned most often, as well as
some that are mentioned less but that we view as important.
\begin{enumerate}
\def\labelenumi{\arabic{enumi}.}
\tightlist
\item
  A well-designed file structure:
\end{enumerate}
\begin{itemize}
\tightlist
\item
  Separate folders for different file types.
\item
  No extraneous files.
\item
  Minimal clutter.
\end{itemize}
\begin{enumerate}
\def\labelenumi{\arabic{enumi}.}
\setcounter{enumi}{1}
\tightlist
\item
  Good documentation:
\end{enumerate}
\begin{itemize}
\tightlist
\item
  Files are clearly named, preferably in a way where the order in which
  they should be run is clear.
\item
  A README is present.
\item
  Dependencies are noted.
\end{itemize}
\begin{enumerate}
\def\labelenumi{\arabic{enumi}.}
\setcounter{enumi}{2}
\tightlist
\item
  Reproducible file paths:
\end{enumerate}
\begin{itemize}
\tightlist
\item
  No absolute paths, or paths leading to locations outside of a
  project's directory, are used in code---only portable (relative)
  paths.
\end{itemize}
\begin{enumerate}
\def\labelenumi{\arabic{enumi}.}
\setcounter{enumi}{3}
\tightlist
\item
  Randomness is accounted for:
\end{enumerate}
\begin{itemize}
\tightlist
\item
  If randomness is used in code, a seed must also be set.
\end{itemize}
\begin{enumerate}
\def\labelenumi{\arabic{enumi}.}
\setcounter{enumi}{4}
\tightlist
\item
  Readable, styled code:
\end{enumerate}
\begin{itemize}
\tightlist
\item
  Code should be written in a coherent style. Code that conforms to a
  style guide or is written in a consistent dialect is easier to read
  (Hermans \& Aldewereld (2017)). We believe that the \texttt{tidyverse}
  provides the most accessible dialect of \texttt{R}.
\end{itemize}
Much of the available literature focuses on file structure,
organization, and naming, and \texttt{fertile}'s features are consistent
with this. Marwick, Boettiger, \& Mullen (2018)\} provide the framework
for file structure that \texttt{fertile} is based on: a structure
similar to that of an \texttt{R} package (R-Core-Team (2020), Wickham
(2015)), with an \texttt{R} folder, as well as \texttt{data},
\texttt{data-raw}, \texttt{inst}, and \texttt{vignettes}.

\subsection{R Packages and Other
Software}\label{r-packages-and-other-software}

Much of this work is highly generalized, written to be applicable to
users working with a variety of statistical software programs. Because
all statistical software programs operate differently, these
recommendations are inherently vague and difficult to implement,
particularly to new analysts who are relatively unfamiliar with their
software. Focused attempts to address reproducibility in specific
certain software programs are more likely to be successful. We focus on
\texttt{R}, due to its open-source nature, accessibility, and popularity
as a tool for statistical analysis.

A small body of \texttt{R} packages focuses on research reproducibility.
\texttt{rrtools} (Marwick (2019)) addresses some of the issues discussed
in Marwick et al. (2018) by creating a basic \texttt{R} package
structure for a data analysis project and implementing a basic
\texttt{testthat::check} functionality. The \texttt{orderly} (FitzJohn
et al. (2020)) package also focuses on file structure, requiring the
user to declare a desired project structure (typically a step-by-step
structure, where outputs from one step are inputs into the next) at the
beginning and then creating the files necessary to achieve that
structure. \texttt{workflowr}'s (Blischak, Carbonetto, \& Stephens
(2019)) functionality is based around version control and making code
easily available online. It works to generate a website containing
time-stamped, versioned, and documented results. \texttt{checkers}
(Ross, DeCicco, \& Randhawa (2018)) allows you to create custom checks
that examine different aspects of reproducibility. \texttt{packrat}
(Ushey, McPherson, Cheng, Atkins, \& Allaire (2018)) is focused on
dependencies, creating a packaged folder containing a project as well as
all of its dependencies so that projects dependent on lesser-used
packages can be easily shared across computers. \texttt{drake} (OpenSci
(2020)) analyzes workflows, skips steps where results are up to date,
and provides evidence that results match the underlying code and data.
Lastly, the \texttt{reproducible} (McIntire \& Chubaty (2020)) package
focuses on the concept of caching: saving information so that projects
can be run faster each time they are re-completed from the start.

Many of these packages are narrow, with each effectively addressing a
small component of reproducibility: file structure, modularization of
code, version control, etc. These packages often succeed in their area
of focus, but at the cost of accessibility to a wider audience. Their
functions are often quite complex to use, and many steps must be
completed to achieve the required reproducibility goal. This cumbersome
nature means that most reproducibility packages currently available are
not easily accessible to users near the beginning of their \texttt{R}
journey, nor particularly useful to those looking for quick and easy
reproducibility checks. A more effective way of realizing widespread
reproducibility is to make the process for doing so simple enough that
it takes little to no conscious effort to implement. You want users to
``fall into a hole'' (we paraphrase Hadley Wickham) of good practice.

\texttt{Continuous\ integration} tools provide more general approaches
to automated checking, which can enhance reproducibility with minimal
code. For example, \texttt{wercker}---a command line tool that leverages
Docker---enables users to test whether their projects will successfully
compile when run on a variety of operating systems without access to the
user's local hard drive (Oracle Corporation (2019)).
\texttt{GitHub\ Actions} is integrated into GitHub and can be configured
to do similar checks on projects hosted in repositories.
\texttt{Travis\ CI} and \texttt{Circle\ CI} are popular continuous
integration tools that can also be used to check \texttt{R} code.

However, while these tools can be useful, they are generalized so as to
be useful to the widest audience. As a result, their checks are not
designed to be \texttt{R}-specific, which makes them sub-optimal for
users looking to address reproducibility issues involving features
specific to the \texttt{R} programming language, such as package
installation and seed setting.

\section{Identifying Gaps In Existing
Solutions}\label{identifying-gaps-in-existing-solutions}

\section{\texorpdfstring{My Contribution: \texttt{fertile}, An R Package
Creating Optimal Conditions For
Reproducibility}{My Contribution: fertile, An R Package Creating Optimal Conditions For Reproducibility}}\label{my-contribution-fertile-an-r-package-creating-optimal-conditions-for-reproducibility}

\section{\texorpdfstring{How \texttt{fertile}
Works}{How fertile Works}}\label{how-fertile-works}

\section{\texorpdfstring{\texttt{fertile} in Practice: Experimental
Results From Smith College Student
Use}{fertile in Practice: Experimental Results From Smith College Student Use}}\label{fertile-in-practice-experimental-results-from-smith-college-student-use}

\chapter{Incorporating Reproducibility Tools Into The Greater Data
Science Community}\label{applications}

\section{\texorpdfstring{Potential Applications of
\texttt{fertile}}{Potential Applications of fertile}}\label{potential-applications-of-fertile}

\section{\texorpdfstring{Integration Of \texttt{fertile} And Other
Reproducibility Tools in Data Science
Education}{Integration Of fertile And Other Reproducibility Tools in Data Science Education}}\label{integration-of-fertile-and-other-reproducibility-tools-in-data-science-education}

\chapter*{Conclusion}\label{conclusion}
\addcontentsline{toc}{chapter}{Conclusion}

If we don't want Conclusion to have a chapter number next to it, we can
add the \texttt{\{-\}} attribute.

\textbf{More info}

And here's some other random info: the first paragraph after a chapter
title or section head \emph{shouldn't be} indented, because indents are
to tell the reader that you're starting a new paragraph. Since that's
obvious after a chapter or section title, proper typesetting doesn't add
an indent there.

\appendix

\chapter{The First Appendix}\label{the-first-appendix}

This first appendix includes all of the R chunks of code that were
hidden throughout the document (using the \texttt{include\ =\ FALSE}
chunk tag) to help with readibility and/or setup.

\textbf{In the main Rmd file}

\textbf{In Chapter \ref{ref-labels}:}

\chapter{The Second Appendix, for
Fun}\label{the-second-appendix-for-fun}

\chapter*{References}\label{references}
\addcontentsline{toc}{chapter}{References}

Placeholder

\hypertarget{refs}{}
\hypertarget{ref-R-workflowr}{}
Blischak, J., Carbonetto, P., \& Stephens, M. (2019). Workflowr: A
framework for reproducible and collaborative data science. Retrieved
from \url{https://CRAN.R-project.org/package=workflowr}

\hypertarget{ref-broman}{}
Broman, K. (2019). Initial steps toward reproducible research: Organize
your data and code. \emph{Sitewide ATOM}. Retrieved from
\url{https://kbroman.org/steps2rr/pages/organize.html}

\hypertarget{ref-cooper2017guide}{}
Cooper, N., Hsing, P.-Y., Croucher, M., Graham, L., James, T.,
Krystalli, A., \& Michonneau, F. (2017). A guide to reproducible code in
ecology and evolution. \emph{British Ecological Society}. Retrieved from
\url{https://www.britishecologicalsociety.org/wp-content/uploads/2017/12/guide-to-reproducible-code.pdf}

\hypertarget{ref-R-orderly}{}
FitzJohn, R., Ashton, R., Hill, A., Eden, M., Hinsley, W., Russell, E.,
\& Thompson, J. (2020). Orderly: Lightweight reproducible reporting.
Retrieved from \url{https://CRAN.R-project.org/package=orderly}

\hypertarget{ref-unix}{}
Gancarz, M. (2003). \emph{Linux and the unix philosophy} (2nd ed.).
Woburn, MA: Digital Press.

\hypertarget{ref-Goodman341ps12}{}
Goodman, S. N., Fanelli, D., \& Ioannidis, J. P. A. (2016). What does
research reproducibility mean? \emph{Science Translational Medicine},
\emph{8}(341), 1--6. \url{http://doi.org/10.1126/scitranslmed.aaf5027}

\hypertarget{ref-hermans2017programming}{}
Hermans, F., \& Aldewereld, M. (2017). Programming is writing is
programming. In \emph{Companion to the first international conference on
the art, science and engineering of programming} (pp. 1--8).

\hypertarget{ref-kitzes2017practice}{}
Kitzes, J., Turek, D., \& Deniz, F. (2017). \emph{The practice of
reproducible research: Case studies and lessons from the data-intensive
sciences}. Berkeley, CA: University of California Press. Retrieved from
\url{https://www.practicereproducibleresearch.org}

\hypertarget{ref-r-opensci}{}
Martinez, C., Hollister, J., Marwick, B., Szöcs, E., Zeitlin, S.,
Kinoshita, B. P., \ldots{} Meinke, B. (2018). Reproducibility in
Science: A Guide to enhancing reproducibility in scientific results and
writing. Retrieved from
\url{http://ropensci.github.io/reproducibility-guide/}

\hypertarget{ref-R-rrtools}{}
Marwick, B. (2019). Rrtools: Creates a reproducible research compendium.
Retrieved from \url{https://github.com/benmarwick/rrtools}

\hypertarget{ref-marwick2018packaging}{}
Marwick, B., Boettiger, C., \& Mullen, L. (2018). Packaging data
analytical work reproducibly using R (and friends). \emph{The American
Statistician}, \emph{72}(1), 80--88.
\url{http://doi.org/doi.org/10.1080/00031305.2017.1375986}

\hypertarget{ref-R-reproducible}{}
McIntire, E. J. B., \& Chubaty, A. M. (2020). Reproducible: A set of
tools that enhance reproducibility beyond package management. Retrieved
from \url{https://CRAN.R-project.org/package=reproducible}

\hypertarget{ref-R-drake}{}
OpenSci, R. (2020). Drake: A pipeline toolkit for reproducible
computation at scale. Retrieved from
\url{https://cran.r-project.org/package=drake}

\hypertarget{ref-wercker}{}
Oracle Corporation. (2019). Wercker. Retrieved from
\url{https://github.com/wercker/wercker}

\hypertarget{ref-coreteam-extensions}{}
R-Core-Team. (2020). Writing r extensions. \emph{R Foundation for
Statistical Computing}. Retrieved from
\url{http://cran.stat.unipd.it/doc/manuals/r-release/R-exts.pdf}

\hypertarget{ref-R-checkers}{}
Ross, N., DeCicco, L., \& Randhawa, N. (2018). Checkers: Automated
checking of best practices for research compendia. Retrieved from
\url{https://github.com/ropenscilabs/checkers/blob/master/DESCRIPTIONr}

\hypertarget{ref-R-packrat}{}
Ushey, K., McPherson, J., Cheng, J., Atkins, A., \& Allaire, J. (2018).
Packrat: A dependency management system for projects and their r package
dependencies. Retrieved from
\url{https://CRAN.R-project.org/package=packrat}

\hypertarget{ref-hadley-packages}{}
Wickham, H. (2015). \emph{R packages} (1st ed.). Sebastopol, CA:
O'Reilly Media, Inc.


% Index?

\end{document}
