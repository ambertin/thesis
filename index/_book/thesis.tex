% This is the Reed College LaTeX thesis template. Most of the work
% for the document class was done by Sam Noble (SN), as well as this
% template. Later comments etc. by Ben Salzberg (BTS). Additional
% restructuring and APA support by Jess Youngberg (JY).
% Your comments and suggestions are more than welcome; please email
% them to cus@reed.edu
%
% See https://www.reed.edu/cis/help/LaTeX/index.html for help. There are a
% great bunch of help pages there, with notes on
% getting started, bibtex, etc. Go there and read it if you're not
% already familiar with LaTeX.
%
% Any line that starts with a percent symbol is a comment.
% They won't show up in the document, and are useful for notes
% to yourself and explaining commands.
% Commenting also removes a line from the document;
% very handy for troubleshooting problems. -BTS

% As far as I know, this follows the requirements laid out in
% the 2002-2003 Senior Handbook. Ask a librarian to check the
% document before binding. -SN

%%
%% Preamble
%%
% \documentclass{<something>} must begin each LaTeX document
\documentclass[12pt,twoside]{reedthesis}
% Packages are extensions to the basic LaTeX functions. Whatever you
% want to typeset, there is probably a package out there for it.
% Chemistry (chemtex), screenplays, you name it.
% Check out CTAN to see: https://www.ctan.org/
%%
\usepackage{graphicx,latexsym}
\usepackage{amsmath}
\usepackage{amssymb,amsthm}
\usepackage{longtable,booktabs,setspace}
\usepackage{chemarr} %% Useful for one reaction arrow, useless if you're not a chem major
\usepackage[hyphens]{url}
% Added by CII
\usepackage{hyperref}
\usepackage{lmodern}
\usepackage{float}
\floatplacement{figure}{H}
% End of CII addition
\usepackage{rotating}

% Next line commented out by CII
%%% \usepackage{natbib}
% Comment out the natbib line above and uncomment the following two lines to use the new
% biblatex-chicago style, for Chicago A. Also make some changes at the end where the
% bibliography is included.
%\usepackage{biblatex-chicago}
%\bibliography{thesis}


% Added by CII (Thanks, Hadley!)
% Use ref for internal links
\renewcommand{\hyperref}[2][???]{\autoref{#1}}
\def\chapterautorefname{Chapter}
\def\sectionautorefname{Section}
\def\subsectionautorefname{Subsection}
% End of CII addition

% Added by CII
\usepackage{caption}
\captionsetup{width=5in}
% End of CII addition

% \usepackage{times} % other fonts are available like times, bookman, charter, palatino

% Syntax highlighting #22
  \usepackage{color}
  \usepackage{fancyvrb}
  \newcommand{\VerbBar}{|}
  \newcommand{\VERB}{\Verb[commandchars=\\\{\}]}
  \DefineVerbatimEnvironment{Highlighting}{Verbatim}{commandchars=\\\{\}}
  % Add ',fontsize=\small' for more characters per line
  \usepackage{framed}
  \definecolor{shadecolor}{RGB}{248,248,248}
  \newenvironment{Shaded}{\begin{snugshade}}{\end{snugshade}}
  \newcommand{\KeywordTok}[1]{\textcolor[rgb]{0.13,0.29,0.53}{\textbf{#1}}}
  \newcommand{\DataTypeTok}[1]{\textcolor[rgb]{0.13,0.29,0.53}{#1}}
  \newcommand{\DecValTok}[1]{\textcolor[rgb]{0.00,0.00,0.81}{#1}}
  \newcommand{\BaseNTok}[1]{\textcolor[rgb]{0.00,0.00,0.81}{#1}}
  \newcommand{\FloatTok}[1]{\textcolor[rgb]{0.00,0.00,0.81}{#1}}
  \newcommand{\ConstantTok}[1]{\textcolor[rgb]{0.00,0.00,0.00}{#1}}
  \newcommand{\CharTok}[1]{\textcolor[rgb]{0.31,0.60,0.02}{#1}}
  \newcommand{\SpecialCharTok}[1]{\textcolor[rgb]{0.00,0.00,0.00}{#1}}
  \newcommand{\StringTok}[1]{\textcolor[rgb]{0.31,0.60,0.02}{#1}}
  \newcommand{\VerbatimStringTok}[1]{\textcolor[rgb]{0.31,0.60,0.02}{#1}}
  \newcommand{\SpecialStringTok}[1]{\textcolor[rgb]{0.31,0.60,0.02}{#1}}
  \newcommand{\ImportTok}[1]{#1}
  \newcommand{\CommentTok}[1]{\textcolor[rgb]{0.56,0.35,0.01}{\textit{#1}}}
  \newcommand{\DocumentationTok}[1]{\textcolor[rgb]{0.56,0.35,0.01}{\textbf{\textit{#1}}}}
  \newcommand{\AnnotationTok}[1]{\textcolor[rgb]{0.56,0.35,0.01}{\textbf{\textit{#1}}}}
  \newcommand{\CommentVarTok}[1]{\textcolor[rgb]{0.56,0.35,0.01}{\textbf{\textit{#1}}}}
  \newcommand{\OtherTok}[1]{\textcolor[rgb]{0.56,0.35,0.01}{#1}}
  \newcommand{\FunctionTok}[1]{\textcolor[rgb]{0.00,0.00,0.00}{#1}}
  \newcommand{\VariableTok}[1]{\textcolor[rgb]{0.00,0.00,0.00}{#1}}
  \newcommand{\ControlFlowTok}[1]{\textcolor[rgb]{0.13,0.29,0.53}{\textbf{#1}}}
  \newcommand{\OperatorTok}[1]{\textcolor[rgb]{0.81,0.36,0.00}{\textbf{#1}}}
  \newcommand{\BuiltInTok}[1]{#1}
  \newcommand{\ExtensionTok}[1]{#1}
  \newcommand{\PreprocessorTok}[1]{\textcolor[rgb]{0.56,0.35,0.01}{\textit{#1}}}
  \newcommand{\AttributeTok}[1]{\textcolor[rgb]{0.77,0.63,0.00}{#1}}
  \newcommand{\RegionMarkerTok}[1]{#1}
  \newcommand{\InformationTok}[1]{\textcolor[rgb]{0.56,0.35,0.01}{\textbf{\textit{#1}}}}
  \newcommand{\WarningTok}[1]{\textcolor[rgb]{0.56,0.35,0.01}{\textbf{\textit{#1}}}}
  \newcommand{\AlertTok}[1]{\textcolor[rgb]{0.94,0.16,0.16}{#1}}
  \newcommand{\ErrorTok}[1]{\textcolor[rgb]{0.64,0.00,0.00}{\textbf{#1}}}
  \newcommand{\NormalTok}[1]{#1}

% To pass between YAML and LaTeX the dollar signs are added by CII
\title{My Final College Paper}
\author{Audrey M. Bertin}
% The month and year that you submit your FINAL draft TO THE LIBRARY (May or December)
\date{May 2021}
\division{Statistical and Data Sciences}
\advisor{Benjamin S. Baumer}
\institution{Smith College}
\degree{Bachelor of Arts}
%If you have two advisors for some reason, you can use the following
% Uncommented out by CII
% End of CII addition

%%% Remember to use the correct department!
\department{Statistical and Data Sciences}
% if you're writing a thesis in an interdisciplinary major,
% uncomment the line below and change the text as appropriate.
% check the Senior Handbook if unsure.
%\thedivisionof{The Established Interdisciplinary Committee for}
% if you want the approval page to say "Approved for the Committee",
% uncomment the next line
%\approvedforthe{Committee}

% Added by CII
%%% Copied from knitr
%% maxwidth is the original width if it's less than linewidth
%% otherwise use linewidth (to make sure the graphics do not exceed the margin)
\makeatletter
\def\maxwidth{ %
  \ifdim\Gin@nat@width>\linewidth
    \linewidth
  \else
    \Gin@nat@width
  \fi
}
\makeatother

%Added by @MyKo101, code provided by @GerbrichFerdinands

\renewcommand{\contentsname}{Table of Contents}
% End of CII addition

\setlength{\parskip}{0pt}

% Added by CII

\providecommand{\tightlist}{%
  \setlength{\itemsep}{0pt}\setlength{\parskip}{0pt}}

\Acknowledgements{
I want to thank a few people.
}

\Dedication{
You can have a dedication here if you wish.
}

\Preface{
This is an example of a thesis setup to use the reed thesis document
class (for LaTeX) and the R bookdown package, in general.
}

\Abstract{
The preface pretty much says it all. \par
Second paragraph of abstract starts here.
}

% End of CII addition
%%
%% End Preamble
%%
%
\begin{document}

% Everything below added by CII
  \maketitle

\frontmatter % this stuff will be roman-numbered
\pagestyle{empty} % this removes page numbers from the frontmatter
  \begin{acknowledgements}
    I want to thank a few people.
  \end{acknowledgements}
  \begin{preface}
    This is an example of a thesis setup to use the reed thesis document
    class (for LaTeX) and the R bookdown package, in general.
  \end{preface}
  \hypersetup{linkcolor=black}
  \setcounter{tocdepth}{2}
  \tableofcontents

  \listoftables

  \listoffigures
  \begin{abstract}
    The preface pretty much says it all. \par
    Second paragraph of abstract starts here.
  \end{abstract}
  \begin{dedication}
    You can have a dedication here if you wish.
  \end{dedication}
\mainmatter % here the regular arabic numbering starts
\pagestyle{fancyplain} % turns page numbering back on

\chapter*{Introduction}\label{introduction}
\addcontentsline{toc}{chapter}{Introduction}

Welcome to the \emph{R Markdown} thesis template. This template is based
on (and in many places copied directly from) the Reed College LaTeX
template, but hopefully it will provide a nicer interface for those that
have never used TeX or LaTeX before. Using \emph{R Markdown} will also
allow you to easily keep track of your analyses in \textbf{R} chunks of
code, with the resulting plots and output included as well. The hope is
this \emph{R Markdown} template gets you in the habit of doing
reproducible research, which benefits you long-term as a researcher, but
also will greatly help anyone that is trying to reproduce or build onto
your results down the road.

Hopefully, you won't have much of a learning period to go through and
you will reap the benefits of a nicely formatted thesis. The use of
LaTeX in combination with \emph{Markdown} is more consistent than the
output of a word processor, much less prone to corruption or crashing,
and the resulting file is smaller than a Word file. While you may have
never had problems using Word in the past, your thesis is likely going
to be about twice as large and complex as anything you've written
before, taxing Word's capabilities. After working with \emph{Markdown}
and \textbf{R} together for a few weeks, we are confident this will be
your reporting style of choice going forward.

\textbf{Why use it?}

\emph{R Markdown} creates a simple and straightforward way to interface
with the beauty of LaTeX. Packages have been written in \textbf{R} to
work directly with LaTeX to produce nicely formatting tables and
paragraphs. In addition to creating a user friendly interface to LaTeX,
\emph{R Markdown} also allows you to read in your data, to analyze it
and to visualize it using \textbf{R} functions, and also to provide the
documentation and commentary on the results of your project. Further, it
allows for \textbf{R} results to be passed inline to the commentary of
your results. You'll see more on this later.

\textbf{Who should use it?}

Anyone who needs to use data analysis, math, tables, a lot of figures,
complex cross-references, or who just cares about the final appearance
of their document should use \emph{R Markdown}. Of particular use should
be anyone in the sciences, but the user-friendly nature of
\emph{Markdown} and its ability to keep track of and easily include
figures, automatically generate a table of contents, index, references,
table of figures, etc. should make it of great benefit to nearly anyone
writing a thesis project.

\textbf{For additional help with bookdown}

Please visit \href{https://bookdown.org/yihui/bookdown/}{the free online
bookdown reference guide}.

\chapter{An Introduction to Reproducibility}\label{reproducibility}

\section{What is reproducibility?}\label{what-is-reproducibility}

\section{Why is reproducibility
important?}\label{why-is-reproducibility-important}

\section{Current Discussion on Reproducibility in
Academia}\label{current-discussion-on-reproducibility-in-academia}

\chapter{Addressing the Challenges of
Reproducibility}\label{my-solution}

\section{Review Of Previous Work}\label{review-of-previous-work}

\section{Identifying Gaps In Existing
Solutions}\label{identifying-gaps-in-existing-solutions}

\section{\texorpdfstring{My Contribution: \texttt{fertile}, An R Package
Creating Optimal Conditions For
Reproducibility}{My Contribution: fertile, An R Package Creating Optimal Conditions For Reproducibility}}\label{my-contribution-fertile-an-r-package-creating-optimal-conditions-for-reproducibility}

\section{\texorpdfstring{How \texttt{fertile}
Works}{How fertile Works}}\label{how-fertile-works}

\section{\texorpdfstring{\texttt{fertile} in Practice: Experimental
Results From Smith College Student
Use}{fertile in Practice: Experimental Results From Smith College Student Use}}\label{fertile-in-practice-experimental-results-from-smith-college-student-use}

\chapter{Incorporating Reproducibility Tools Into The Greater Data
Science Community}\label{applications}

\section{\texorpdfstring{Potential Applications of
\texttt{fertile}}{Potential Applications of fertile}}\label{potential-applications-of-fertile}

\section{\texorpdfstring{Integration Of \texttt{fertile} And Other
Reproducibility Tools in Data Science
Education}{Integration Of fertile And Other Reproducibility Tools in Data Science Education}}\label{integration-of-fertile-and-other-reproducibility-tools-in-data-science-education}

\chapter*{Conclusion}\label{conclusion}
\addcontentsline{toc}{chapter}{Conclusion}

If we don't want Conclusion to have a chapter number next to it, we can
add the \texttt{\{-\}} attribute.

\textbf{More info}

And here's some other random info: the first paragraph after a chapter
title or section head \emph{shouldn't be} indented, because indents are
to tell the reader that you're starting a new paragraph. Since that's
obvious after a chapter or section title, proper typesetting doesn't add
an indent there.

\appendix

\chapter{The First Appendix}\label{the-first-appendix}

This first appendix includes all of the R chunks of code that were
hidden throughout the document (using the \texttt{include\ =\ FALSE}
chunk tag) to help with readibility and/or setup.

\textbf{In the main Rmd file}
\begin{Shaded}
\begin{Highlighting}[]
\CommentTok{# This chunk ensures that the thesisdown package is}
\CommentTok{# installed and loaded. This thesisdown package includes}
\CommentTok{# the template files for the thesis.}
\ControlFlowTok{if}\NormalTok{ (}\OperatorTok{!}\KeywordTok{require}\NormalTok{(remotes)) \{}
  \ControlFlowTok{if}\NormalTok{ (params}\OperatorTok{$}\StringTok{`}\DataTypeTok{Install needed packages for \{thesisdown\}}\StringTok{`}\NormalTok{) \{}
    \KeywordTok{install.packages}\NormalTok{(}\StringTok{"remotes"}\NormalTok{, }\DataTypeTok{repos =} \StringTok{"https://cran.rstudio.com"}\NormalTok{)}
\NormalTok{  \} }\ControlFlowTok{else}\NormalTok{ \{}
    \KeywordTok{stop}\NormalTok{(}
      \KeywordTok{paste}\NormalTok{(}\StringTok{'You need to run install.packages("remotes")",}
\StringTok{            "first in the Console.'}\NormalTok{)}
\NormalTok{    )}
\NormalTok{  \}}
\NormalTok{\}}
\ControlFlowTok{if}\NormalTok{ (}\OperatorTok{!}\KeywordTok{require}\NormalTok{(thesisdown)) \{}
  \ControlFlowTok{if}\NormalTok{ (params}\OperatorTok{$}\StringTok{`}\DataTypeTok{Install needed packages for \{thesisdown\}}\StringTok{`}\NormalTok{) \{}
\NormalTok{    remotes}\OperatorTok{::}\KeywordTok{install_github}\NormalTok{(}\StringTok{"ismayc/thesisdown"}\NormalTok{)}
\NormalTok{  \} }\ControlFlowTok{else}\NormalTok{ \{}
    \KeywordTok{stop}\NormalTok{(}
      \KeywordTok{paste}\NormalTok{(}
        \StringTok{"You need to run"}\NormalTok{,}
        \StringTok{'remotes::install_github("ismayc/thesisdown")'}\NormalTok{,}
        \StringTok{"first in the Console."}
\NormalTok{      )}
\NormalTok{    )}
\NormalTok{  \}}
\NormalTok{\}}
\KeywordTok{library}\NormalTok{(thesisdown)}
\CommentTok{# Set how wide the R output will go}
\KeywordTok{options}\NormalTok{(}\DataTypeTok{width =} \DecValTok{70}\NormalTok{)}
\end{Highlighting}
\end{Shaded}
\textbf{In Chapter \ref{ref-labels}:}

\chapter{The Second Appendix, for
Fun}\label{the-second-appendix-for-fun}

\backmatter

\chapter*{References}\label{references}
\addcontentsline{toc}{chapter}{References}

\markboth{References}{References}

\noindent

\setlength{\parindent}{-0.20in} \setlength{\leftskip}{0.20in}
\setlength{\parskip}{8pt}

\hypertarget{refs}{}
\hypertarget{ref-angel2000}{}
Angel, E. (2000). \emph{Interactive computer graphics : A top-down
approach with opengl}. Boston, MA: Addison Wesley Longman.

\hypertarget{ref-angel2001}{}
Angel, E. (2001a). \emph{Batch-file computer graphics : A bottom-up
approach with quicktime}. Boston, MA: Wesley Addison Longman.

\hypertarget{ref-angel2002a}{}
Angel, E. (2001b). \emph{Test second book by angel}. Boston, MA: Wesley
Addison Longman.


% Index?

\end{document}
