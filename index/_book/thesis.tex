% This is the Reed College LaTeX thesis template. Most of the work
% for the document class was done by Sam Noble (SN), as well as this
% template. Later comments etc. by Ben Salzberg (BTS). Additional
% restructuring and APA support by Jess Youngberg (JY).
% Your comments and suggestions are more than welcome; please email
% them to cus@reed.edu
%
% See https://www.reed.edu/cis/help/LaTeX/index.html for help. There are a
% great bunch of help pages there, with notes on
% getting started, bibtex, etc. Go there and read it if you're not
% already familiar with LaTeX.
%
% Any line that starts with a percent symbol is a comment.
% They won't show up in the document, and are useful for notes
% to yourself and explaining commands.
% Commenting also removes a line from the document;
% very handy for troubleshooting problems. -BTS

% As far as I know, this follows the requirements laid out in
% the 2002-2003 Senior Handbook. Ask a librarian to check the
% document before binding. -SN

%%
%% Preamble
%%
% \documentclass{<something>} must begin each LaTeX document
\documentclass[12pt,twoside]{reedthesis}
% Packages are extensions to the basic LaTeX functions. Whatever you
% want to typeset, there is probably a package out there for it.
% Chemistry (chemtex), screenplays, you name it.
% Check out CTAN to see: https://www.ctan.org/
%%
\usepackage{graphicx,latexsym}
\usepackage{amsmath}
\usepackage{amssymb,amsthm}
\usepackage{longtable,booktabs,setspace}
\usepackage{chemarr} %% Useful for one reaction arrow, useless if you're not a chem major
\usepackage[hyphens]{url}
% Added by CII
\usepackage{hyperref}
\usepackage{lmodern}
\usepackage{float}
\floatplacement{figure}{H}
% End of CII addition
\usepackage{rotating}

\setlength{\headheight}{18pt}%

% Next line commented out by CII
%%% \usepackage{natbib}
% Comment out the natbib line above and uncomment the following two lines to use the new
% biblatex-chicago style, for Chicago A. Also make some changes at the end where the
% bibliography is included.
%\usepackage{biblatex-chicago}
%\bibliography{thesis}


% Added by CII (Thanks, Hadley!)
% Use ref for internal links
\renewcommand{\hyperref}[2][???]{\autoref{#1}}
\def\chapterautorefname{Chapter}
\def\sectionautorefname{Section}
\def\subsectionautorefname{Subsection}
% End of CII addition

% Added by CII
\usepackage{caption}
\captionsetup{width=5in}
% End of CII addition

% \usepackage{times} % other fonts are available like times, bookman, charter, palatino

% Syntax highlighting #22
  \usepackage{color}
  \usepackage{fancyvrb}
  \newcommand{\VerbBar}{|}
  \newcommand{\VERB}{\Verb[commandchars=\\\{\}]}
  \DefineVerbatimEnvironment{Highlighting}{Verbatim}{commandchars=\\\{\}}
  % Add ',fontsize=\small' for more characters per line
  \usepackage{framed}
  \definecolor{shadecolor}{RGB}{248,248,248}
  \newenvironment{Shaded}{\begin{snugshade}}{\end{snugshade}}
  \newcommand{\KeywordTok}[1]{\textcolor[rgb]{0.13,0.29,0.53}{\textbf{#1}}}
  \newcommand{\DataTypeTok}[1]{\textcolor[rgb]{0.13,0.29,0.53}{#1}}
  \newcommand{\DecValTok}[1]{\textcolor[rgb]{0.00,0.00,0.81}{#1}}
  \newcommand{\BaseNTok}[1]{\textcolor[rgb]{0.00,0.00,0.81}{#1}}
  \newcommand{\FloatTok}[1]{\textcolor[rgb]{0.00,0.00,0.81}{#1}}
  \newcommand{\ConstantTok}[1]{\textcolor[rgb]{0.00,0.00,0.00}{#1}}
  \newcommand{\CharTok}[1]{\textcolor[rgb]{0.31,0.60,0.02}{#1}}
  \newcommand{\SpecialCharTok}[1]{\textcolor[rgb]{0.00,0.00,0.00}{#1}}
  \newcommand{\StringTok}[1]{\textcolor[rgb]{0.31,0.60,0.02}{#1}}
  \newcommand{\VerbatimStringTok}[1]{\textcolor[rgb]{0.31,0.60,0.02}{#1}}
  \newcommand{\SpecialStringTok}[1]{\textcolor[rgb]{0.31,0.60,0.02}{#1}}
  \newcommand{\ImportTok}[1]{#1}
  \newcommand{\CommentTok}[1]{\textcolor[rgb]{0.56,0.35,0.01}{\textit{#1}}}
  \newcommand{\DocumentationTok}[1]{\textcolor[rgb]{0.56,0.35,0.01}{\textbf{\textit{#1}}}}
  \newcommand{\AnnotationTok}[1]{\textcolor[rgb]{0.56,0.35,0.01}{\textbf{\textit{#1}}}}
  \newcommand{\CommentVarTok}[1]{\textcolor[rgb]{0.56,0.35,0.01}{\textbf{\textit{#1}}}}
  \newcommand{\OtherTok}[1]{\textcolor[rgb]{0.56,0.35,0.01}{#1}}
  \newcommand{\FunctionTok}[1]{\textcolor[rgb]{0.00,0.00,0.00}{#1}}
  \newcommand{\VariableTok}[1]{\textcolor[rgb]{0.00,0.00,0.00}{#1}}
  \newcommand{\ControlFlowTok}[1]{\textcolor[rgb]{0.13,0.29,0.53}{\textbf{#1}}}
  \newcommand{\OperatorTok}[1]{\textcolor[rgb]{0.81,0.36,0.00}{\textbf{#1}}}
  \newcommand{\BuiltInTok}[1]{#1}
  \newcommand{\ExtensionTok}[1]{#1}
  \newcommand{\PreprocessorTok}[1]{\textcolor[rgb]{0.56,0.35,0.01}{\textit{#1}}}
  \newcommand{\AttributeTok}[1]{\textcolor[rgb]{0.77,0.63,0.00}{#1}}
  \newcommand{\RegionMarkerTok}[1]{#1}
  \newcommand{\InformationTok}[1]{\textcolor[rgb]{0.56,0.35,0.01}{\textbf{\textit{#1}}}}
  \newcommand{\WarningTok}[1]{\textcolor[rgb]{0.56,0.35,0.01}{\textbf{\textit{#1}}}}
  \newcommand{\AlertTok}[1]{\textcolor[rgb]{0.94,0.16,0.16}{#1}}
  \newcommand{\ErrorTok}[1]{\textcolor[rgb]{0.64,0.00,0.00}{\textbf{#1}}}
  \newcommand{\NormalTok}[1]{#1}

% To pass between YAML and LaTeX the dollar signs are added by CII
\title{Addressing The Scientific Reproducibility Crisis Through Educational
Software Integration}
\author{Audrey M. Bertin}
% The month and year that you submit your FINAL draft TO THE LIBRARY (May or December)
\date{May 2021}
\division{Statistical and Data Sciences}
\advisor{Benjamin S. Baumer}
\institution{Smith College}
\degree{Bachelor of Arts}
%If you have two advisors for some reason, you can use the following
% Uncommented out by CII
% End of CII addition

%%% Remember to use the correct department!
\department{Statistical and Data Sciences}
% if you're writing a thesis in an interdisciplinary major,
% uncomment the line below and change the text as appropriate.
% check the Senior Handbook if unsure.
%\thedivisionof{The Established Interdisciplinary Committee for}
% if you want the approval page to say "Approved for the Committee",
% uncomment the next line
%\approvedforthe{Committee}

% Added by CII
%%% Copied from knitr
%% maxwidth is the original width if it's less than linewidth
%% otherwise use linewidth (to make sure the graphics do not exceed the margin)
\makeatletter
\def\maxwidth{ %
  \ifdim\Gin@nat@width>\linewidth
    \linewidth
  \else
    \Gin@nat@width
  \fi
}
\makeatother

%Added by @MyKo101, code provided by @GerbrichFerdinands

\renewcommand{\contentsname}{Table of Contents}
% End of CII addition

\setlength{\parskip}{0pt}

% Added by CII

\providecommand{\tightlist}{%
  \setlength{\itemsep}{0pt}\setlength{\parskip}{0pt}}

\Acknowledgements{
I want to thank a few people.
}

\Dedication{
You can have a dedication here if you wish.
}

\Preface{
This is an example of a thesis setup to use the reed thesis document
class (for LaTeX) and the R bookdown package, in general.
}

\Abstract{
The preface pretty much says it all. \par
Second paragraph of abstract starts here.
}

% End of CII addition
%%
%% End Preamble
%%
%
\begin{document}

% Everything below added by CII
  \maketitle

\frontmatter % this stuff will be roman-numbered
\pagestyle{empty} % this removes page numbers from the frontmatter
  \begin{acknowledgements}
    I want to thank a few people.
  \end{acknowledgements}
  \begin{preface}
    This is an example of a thesis setup to use the reed thesis document
    class (for LaTeX) and the R bookdown package, in general.
  \end{preface}
  \hypersetup{linkcolor=black}
  \setcounter{tocdepth}{2}
  \tableofcontents


  \begin{abstract}
    The preface pretty much says it all. \par
    Second paragraph of abstract starts here.
  \end{abstract}
  \begin{dedication}
    You can have a dedication here if you wish.
  \end{dedication}
\mainmatter % here the regular arabic numbering starts
\pagestyle{fancyplain} % turns page numbering back on
\begin{Shaded}
\begin{Highlighting}[]
\KeywordTok{library}\NormalTok{(knitr)}
\NormalTok{hook_output =}\StringTok{ }\NormalTok{knit_hooks}\OperatorTok{$}\KeywordTok{get}\NormalTok{(}\StringTok{'output'}\NormalTok{)}
\NormalTok{knit_hooks}\OperatorTok{$}\KeywordTok{set}\NormalTok{(}\DataTypeTok{output =} \ControlFlowTok{function}\NormalTok{(x, options) \{}
  \CommentTok{# this hook is used only when the linewidth option is not NULL}
  \ControlFlowTok{if}\NormalTok{ (}\OperatorTok{!}\KeywordTok{is.null}\NormalTok{(n <-}\StringTok{ }\NormalTok{options}\OperatorTok{$}\NormalTok{linewidth)) \{}
\NormalTok{    x =}\StringTok{ }\NormalTok{knitr}\OperatorTok{:::}\KeywordTok{split_lines}\NormalTok{(x)}
    \CommentTok{# any lines wider than n should be wrapped}
    \ControlFlowTok{if}\NormalTok{ (}\KeywordTok{any}\NormalTok{(}\KeywordTok{nchar}\NormalTok{(x) }\OperatorTok{>}\StringTok{ }\NormalTok{n)) x =}\StringTok{ }\KeywordTok{strwrap}\NormalTok{(x, }\DataTypeTok{width =}\NormalTok{ n)}
\NormalTok{    x =}\StringTok{ }\KeywordTok{paste}\NormalTok{(x, }\DataTypeTok{collapse =} \StringTok{'}\CharTok{\textbackslash{}n}\StringTok{'}\NormalTok{)}
\NormalTok{  \}}
  \KeywordTok{hook_output}\NormalTok{(x, options)}
\NormalTok{\})}
\end{Highlighting}
\end{Shaded}
\chapter*{Introduction}\label{introduction}
\addcontentsline{toc}{chapter}{Introduction}

Potential sources:

\url{https://arxiv.org/abs/1401.3269}

\url{https://academic.oup.com/isp/article-abstract/17/4/392/2528285}

\url{https://berkeleysciencereview.com/2014/06/reproducible-collaborative-data-science/}

\url{https://guides.lib.uw.edu/research/reproducibility/teaching}

\url{https://escholarship.org/uc/item/90b2f5xh}

\url{https://www.mitpressjournals.org/doi/full/10.1162/dint_a_00053}

\chapter{An Introduction to Reproducibility}\label{reproducibility}

\section{What Is Reproducibility?}\label{what-is-reproducibility}

In the field of data science, research is considered fully
\emph{reproducible} when the requisite code and data files produce
identical results when run by another analyst, or more generally, when a
researcher can ``duplicate the results of a prior study using the same
materials as were used by the original investigator'' (Bollen et al.
(2015)).

This term was first coined in 1992 by computer scientist Jon Claerbout,
who associated it with a ``software platform and set of procedures that
permit the reader of a paper to see the entire processing trail from the
raw data and code to figures and tables'' (Claerbout \& Karrenbach
(1992)).

Since its inception, the concept of reproducibility has been applied
across many different data-intensive fields, including epidemiology,
computational biology, economics, clinical trials, and, now, the more
general domain of statistical and data sciences (Goodman, Fanelli, \&
Ioannidis (2016)).

Reproducible research has a wide variety of benefits in the scientific
community. When researchers provide the code and data used for their
work in a well-organized and reproducible format, readers are more
easily able to determine the veracity of any findings by following the
steps from raw data to conclusions. The creators of reproducible
research can also more easily receive more specific feedback (including
bug fixes) on their work. Moreover, others interested in the research
topic can use the code to apply the methods and ideas used in one
project to their own work with minimal effort.

Although often confused, the concept of \emph{reproducibility} is
distinct from the similar idea of \emph{replicability}: the ability of a
researcher to duplicate the results of a study when following the
original procedure but collecting new data. Replicability has
larger-scale implications than reproducibilty; the findings of research
studies can not be accepted unless a variety of other researchers come
to the same conclusions through independent work.
\begin{center}\includegraphics[width=1\linewidth]{figure/versus} \end{center}

Reproducibility and replicability are both necessary to the advancement
of scientific research, but they vary significantly in terms of their
difficulty to achieve. Reproducibility, in theory, is somewhat simple to
attain in data analyses--because code is inherently non-random
(excepting applications involving random number generation) and data
remain consistent, variability is highly restricted. The achievement of
replicability, on the other hand, is a much more complex challenge,
involving significantly more variablility and requiring high quality
data, effective study design, and incredibly robust hypotheses.

\section{The Reproducibility Crisis}\label{the-reproducibility-crisis}

Despite the relative simplicity of achieving reproducibility, a
significant proportion of the work produced in the scientific community
fails to meet reproducibility standards. 52\% of respondents in a 2016
Nature survey believed that science was going through a ``crisis'' of
reproducibility. Additionally, the vast majority of researchers across
all fields studied reported having been unable to reproduce another
researcher's results, while approximately half reported having been
unable to reproduce their own (Baker (2016)). Other studies paint an
even bleaker picture: a 2015 study found that over 50\% of studies
psychology failed reproducibility tests and research from 2012 found
that figure closer to 90\% in the field of cancer biology (Baker (2015),
Begley \& Ellis (2012)).

In the past several years, this ``crisis'' of reproducibility has risen
toward the forefront of scientific discussion. Without reproducibility,
the scientific community cannot properly verify study results.This makes
it difficult to identify which information should be believed and which
should not and increases the likelihood that studies sharing misleading
information will be dispersed. The rise of data-driven technologies,
alongside our newly founded ability to instantly share knowledge
worldwide, has made reproducibility increasingly critical to the
advancement of scientific understanding, necessitating the development
of solutions for addressing the issue.

Academics have recognized this, and publications on the topic appear to
have increased siginficantly in the last several years (Eisner (2018);
Fidler \& Wilcox (2018); Gosselin (2020); McArthur (2019); Wallach,
Boyack, \& Ioannidis (2018)).

\section{The Components of Reproducible
Research}\label{the-components-of-reproducible-research}

In order to see why there is an issue with reproducibility and gain a
sense of how to solve it, it is important to first understand the
components of reproducibility. Essentially, to answer, ``What parts does
researcher need to include, or what steps do they need to take, to be
able to declare their work reproducible?''

Publications attempting to answer this can be found in all areas of
scientific research. However, as Goodman et al. (2016) argue, the
language and conceptual framework used to describe research
reproducibility varies significantly across the sciences, and there are
no clear standards on reproducibility agreed upon by the scientific
community as a whole.

At a minimum, according to Goodman et al. (2016), achieving
reproducibility requires the sharing of data (raw or processed),
relevant metadata, code, and related software. However, according to
other authors, the full achievement of reproducibility may require
additional components.

Kitzes, Turek, \& Deniz (2017) present a collection of case studies on
reproducibility practices from across the data-intensive sciences,
illustrating a variety of recommendations and techniques for achieving
reproducibility. Although their work does not come to a consensus on the
exact standards of reproducibility that should be followed, several
common trends and principles emerge from their case studies that extend
beyond the minimum recommendations of Goodman et al. (2016):
\begin{enumerate}
\def\labelenumi{\arabic{enumi})}
\tightlist
\item
  use clear separation, labeling, and documentation in provided code,
\item
  automate processes when possible, and
\item
  design the data analysis workflow as a sequence of small steps glued
  together, with outputs from one step serving as inputs into the next.
  This is a common suggestion within the computing community,
  originating as part of the Unix philosophy (Gancarz (2003)).
\end{enumerate}
Cooper et al. (2017) focus on data analysis completed in \texttt{R} and
identify a similar list of important reproducibility components,
reinforcing the need for clearly labeled, well-documented, and
well-separated files. In addition, they recommend publishing a list of
software dependencies and using version control to track project changes
over time.

Broman (2019) reiterates the need for clear naming and file separation
while sharing several additional suggestions: keep the project contained
in one directory, use relative paths when accessing the file system, and
include a \texttt{README} file describing the project.

The reproducibility recommendations from R OpenSci, a non-profit
initiative founded in 2011 to make scientific data retrieval
reproducible, share similar principles to those discussed previously.
They focus on a need for a well-developed file system, with no
extraneous files and clear labeling. They also reiterate the need to
note dependencies and use automation when possible, while making clear a
suggestion not present in the previously-discussed literature: the need
to use seeds, which allow for the saving and restoring of the random
number generator state, when running code involving randomness (Martinez
et al. (2018)).

Although these recommendations differ from one another, when considered
in combination they provide a well-rounded picture of the components
important to research reproducibility across the scientific community:
\begin{enumerate}
\def\labelenumi{\arabic{enumi}.}
\tightlist
\item
  The basic project components are made accessible to the public:
\end{enumerate}
\begin{itemize}
\tightlist
\item
  Data (raw and/or processed)
\item
  Metadata
\item
  Code
\item
  Related Software
\end{itemize}
\begin{enumerate}
\def\labelenumi{\arabic{enumi}.}
\setcounter{enumi}{1}
\tightlist
\item
  The file structure of project is well-organized:
\end{enumerate}
\begin{itemize}
\tightlist
\item
  Separate folders for different file types.
\item
  No extraneous files.
\item
  Minimal clutter.
\end{itemize}
\begin{enumerate}
\def\labelenumi{\arabic{enumi}.}
\setcounter{enumi}{2}
\tightlist
\item
  The project is documented well:
\end{enumerate}
\begin{itemize}
\tightlist
\item
  Files are clearly named, preferably in a way where the order in which
  they should be run is clear.
\item
  A README is present.
\item
  Code contains comments.
\item
  Software dependencies are noted.
\end{itemize}
\begin{enumerate}
\def\labelenumi{\arabic{enumi}.}
\setcounter{enumi}{3}
\tightlist
\item
  File paths used in code are not system- or user-dependent:
\end{enumerate}
\begin{itemize}
\tightlist
\item
  No absolute paths.
\item
  No paths leading to locations outside of a project's directory.
\item
  Only relative paths, pointing to locations within a project's
  directory, are permitted.
\end{itemize}
\begin{enumerate}
\def\labelenumi{\arabic{enumi}.}
\setcounter{enumi}{4}
\tightlist
\item
  Randomness is accounted for:
\end{enumerate}
\begin{itemize}
\tightlist
\item
  If randomness is used in code, a seed must also be set.
\end{itemize}
\begin{enumerate}
\def\labelenumi{\arabic{enumi}.}
\setcounter{enumi}{5}
\tightlist
\item
  Code is readable and consistently styled:
\end{enumerate}
\begin{itemize}
\tightlist
\item
  Though not mentioned in the sources described previously, it is also
  important that code be written in a coherent style. This is because
  code that conforms to a style guide or is written in a consistent
  dialect is easier to read, simplifying the process of following a
  researcher's work from beginning to end (Hermans \& Aldewereld
  (2017)).
\end{itemize}
\section{Current Attempts to Address Reproducibility in Scientific
Publishing}\label{current-attempts-to-address-reproducibility-in-scientific-publishing}

In an attempt to increase reproducibility, leaders from academic
journals around the world have taken steps to create new standards and
requirements for submitted articles. These standards attempt to address
the components of reproducibility listed previously, requesting that
authors provide certain materials necessary for reproducibing their work
when they submit an article. However, these standards are highly
inconsistent, varying significantly both across and within disciplines,
and many only cover one or two of the six primary components, if any at
all.

To illustrate this point, we will consider several case studies from
journals publishing research on a variety of scientific fields.

\subsection{Case Studies Across The
Sciences}\label{case-studies-across-the-sciences}

The journal whose requirements appear to align most closely with those
components defined previously in Section 3 is the \emph{American Journal
of Political Science} (AJPS). In 2012, the AJPS became the first
political science journal to require authors to make their data openly
accessible online, and the publication has instituted stricter
requirements since. AJPS now requires that authors submit the following
alongside their papers (American Journal of Political Science (2016)).
\begin{itemize}
\tightlist
\item
  The dataset analyzed in the paper and information about its source. If
  the dataset has been processed, instructions for manipulating the raw
  data to achieve the final data must also be shared.
\item
  Detailed, clear code necessary for reproducing all of the tables and
  figures in the paper.
\item
  Documentation, including a README and codebook.
\item
  Information about the software used to conduct the analysis, including
  the specific versions and packages used.
\end{itemize}
These standards are quite thorough and contain mandates for the
inclusion of the vast majority of components necessary for complete
reproducibility. Most journals, however, do not come close to meeting
such high standards in their reproducibility statements.

For example, in the biomedical sciences, a group of editors
reproesenting over 30 major journals met in 2014 to address
reproducibility in their field, coming to a consensus on a set of
principles they wanted to uphold (National Institutes of Health (2014)).
Listed below are those relating specifically to the use of data and
statistical methods:
\begin{enumerate}
\def\labelenumi{\arabic{enumi})}
\item
  Journals in the biomedical sciences should have a mechanism to check
  the statistical accuracy of submissions.
\item
  Journals should have no (or generous) limit on methods section length.
\item
  Journals should use a checklist to ensure the reporting of key
  information, including:
\end{enumerate}
\begin{itemize}
\tightlist
\item
  The article meets nomenclature/reporting standards of the biomedical
  field.
\item
  Investigators report how often each experiment was performed and
  whether results were substantiated by repetition under a range of
  conditions.
\item
  Statistics must be fully reported in the paper (including test used,
  value of N, definition of center, dispersion and precision measures).
\item
  Authors must state whether samples were randomized and how.
\item
  Authors must state whether the experiment was blinded.
\item
  Authors must clearly state the criteria used for exclusion of any data
  or subjects and must include all results, even those that do not
  support the main findings.
\end{itemize}
\begin{enumerate}
\def\labelenumi{\arabic{enumi})}
\setcounter{enumi}{3}
\item
  All datasets used in analysis must be made available on request and
  should be hosted on public repositories when possible. If not
  possible, data values should be presented in the paper or
  supplementary information.
\item
  Software sharing should be encouraged. At the minimum, authors should
  provide a statement describing if software is available and how to
  obtain it.
\end{enumerate}
Even though these principles seem well-developed on the surface, they
fail to meet even the basic requirements defined by Goodman et al.
(2016) previously. Several of the principles are purely recommendations;
there is no requirement that code be shared, nor metadata. Additionally,
software requirements are quite loose, requiring no information about
dependencies or software version.

We see a similar issue even in journals designed specifically for the
purpose of improving scientific reproducibility. \emph{Experimental
Results}, a publication created by Cambridge University Press to address
some of the reproducibility and open access issues in academia, also
falls short of meeting high standards. The journal, which showcases
articles from a variety of scientific decisions disciplines, states in
their transparency and openness policy:
\begin{quote}
Whenever possible authors should make evidence and resources that
underpin published findings, such as data, code, and other materials,
available to readers without undue barriers to access.
\end{quote}
The inclusion of code and data are only recommended and no definition of
what ``other materials'' may mean is provided. No components of
reproducibility extending beyond those required at a minimum are even
considered (Cambridge University Press (2020)).

The \emph{American Economic Review}, the first of the top economics
journals to require the inclusion of data alongside publications, has
stronger guidelines than several of those mentioned previously, though
not as strong as the \emph{American Journal of Political Science}. Their
Data and Code Availability Policy states the following (American
Economic Association (2020)):
\begin{quote}
It is the policy of the American Economic Association to publish papers
only if the data and code used in the analysis are clearly and precisely
documented, and access to the data and code is clearly and precisely
documented and is non-exclusive to the authors.
\end{quote}
These requirements are quite strict, prohibiting exceptions for papers
using data or code not available to the public in the way that many
other journals claiming to promote reproducibility do.

\subsection{Case Studies In The Statistical And Data
Sciences}\label{case-studies-in-the-statistical-and-data-sciences}

When considering reproducibility policy, the field of Statistical and
Data Sciences performs relatively well. The majority of highly ranked
journals in the field contain statements on reproducibility. Some of
these are quite robust, surpassing the requirements of many of the other
journals discussed previously, while others are lacking.

The \emph{Journal of the American Statistical Association} stands out as
having relatively robust requirements. The publication's guideliens
require that data be made publicly available at the time of publication
except for reasons of security or confidentiality. It is strongly
recommended that code be deposited in open repositories. If data is used
in a process form, the provided code should include the necessary
cleaning/preparation steps. Data must be in an easily understood form
and a data dictionary should be included. Code should also be in a form
that can be used and understood by others, including consistent and
readable syntax and comments. Workflows involving more than one script
should also contain a master script, Makefile, or other mechanism that
makes it clear what each component does, in what order to run them, and
what the inputs and outputs to each area (American Statistical
Association (2020)).

The \emph{Journal of Statistical Software} also has strong guidelines,
though less thorough. Authors must provide \emph{commented} source code
for their software; all figures, tables, and output must be exactly
reproducible on at least one platform, and random number generation must
be controlled; and replication materials (typically in the form of a
script) must be provided (Journal of Statistical Software (2020)).

The expectations of the \emph{Journal of Computational and Graphical
Statistics} are notably weaker, requiring only that authors ``submit
code and datasets as online supplements to the manuscript,'' with
exceptions for security or confidentiality, but providing no further
detail (Journal of Computational and Graphical Statistics (2020)). The
\emph{R Journal} has the same requirements, but with no exceptions on
the data provision policy, stating that authors should ``not use such
datasets as examples'' (R Journal Editors (2020)).

Perhaps the weakest reproducibility policies come from \emph{The
American Statistician} and the \emph{Annals of Statistics}. The former
appears to have no requirements, stating only that it ``strongly
encourages authors to submit datasets, code, other programs, and/or
appendices that are directly relevant to their submitted articles,''
while the latter appears to have no statement on reproducibility at all
(The American Statistician (2020))s.
\begin{center}\includegraphics[width=1\linewidth]{figure/stats-journals} \end{center}

\subsection{The Bigger Picture}\label{the-bigger-picture}

The journals mentioned here are just some of the many academic
publishers with reproducibility policies. While they provide a sense of
the specific wording and requirements of some policies, they do not
necessarily serve as a representative sample of all academic publishing.
It is important to also consider the bigger picture, exploring the state
of reproducibility policy in academic publishing as a whole.

Given the scale of the academic publishing network and the sheer number
of journals around the world, this is not necessarily an easy task.

In order to simplify this process, academics at the Center for Open
Science (COS) attempted to create a metric, called the TOP Factor. The
TOP Factor reports the steps that a journal is taking to implement open
science practices. It has been calculated for a wide variety of
journals, though the COS is still far from scoring all of the
publications that are currently available.

The TOP Factor is calculated as follows. Publications are scored on a
variety of categories associated with open science and reproducibility.
For each category, they receive a score between 0 (poor) and 3
(excellent) based on the degree to which they emphasize each category in
their submission/publication policies. A journal's final score, which
can range from 0 to 30, is the sum of the individual scores in each of
the categories.
\begin{center}\includegraphics[width=1\linewidth]{figure/top-1} \end{center}
\begin{center}\includegraphics[width=1\linewidth]{figure/top-2} \end{center}

When looking at the overall distribution of TOP Factor scores, we see a
relatively grim picture: Around 50\% of journals score as low as 0-5
overall, while only just over 5\% score more than 15, just half of the
maximum possible score. Over 40 journals failed to score a single point
(Woolston (2020)).

Although it is clear that some journals have relatively strong
reproducibility and openness policies, that is clearly not the norm. And
many that do appear to have policies lacking in robustness, including
exceptions for data privacy and security concerns or phrasing guidelines
as recommendations rather than reqquirements. The field of data science
stands out among the rest, with the majority of top journals having
relatively robust policies.

\subsection{Assessing the Success of Academic Reproducibility
Policies}\label{assessing-the-success-of-academic-reproducibility-policies}

We have seen that, although not necessarily the standard, some journals
across the sciences have enacted reproducibility policies. The simple
implementation of a policy, however, does not ensure that its goals will
be achieved. Reproducibility can only be addressed when both authors
\emph{and} journal reviewers actively implement publishing standards in
practice. Without participation and dedication from all involved,
reproducibility guidelines serve more as a theoretical goal than a
practical achievement.

It is important to ask, then, whether academic reproducibility standards
\emph{actually} result in a greater number of reproducible publications.

Let us consider the case of the journal \emph{Science}. \emph{Science}
instituted a reproducibility policy in 2011 and has maintained it ever
since. In its original form, their policy stated the following:
\begin{quote}
All data necessary to understand, assess, and extend the conclusions of
the manuscript must be available to any reader of Science. All computer
codes involved in the creation or analysis of data must also be
available to any reader of Science. After publication, all reasonable
requests for data and materials must be fulfilled. Any restrictions on
the availability of data, codes, or materials\ldots{}must be disclosed
to the editors upon submission\ldots{}
\end{quote}
This policy is similar to many of the others considered previously,
requiring the publishing of code and data with exceptions permitted when
necessary.

Stodden, Seiler, \& Ma (2018a) tested the efficacy of this policy in
practice, emailing corresponding authors of 204 articles published in
the year after \emph{Science} first implmented its policy to request the
data and code associated with their articles. The researchers only
received (at least some of) the requested material from 36\% of authors.
This low rates were due to several factors:
\begin{itemize}
\tightlist
\item
  26\% of authors did not respond to email contact.
\item
  11\% of authors were unwilling to provide the data or code without
  further information regarding the researchers' intentions.
\item
  11\% asked the researchers to contact someone else and that person did
  not respond.
\item
  7\% refused to share data and/or code.
\item
  3\% directed the researchers back to their paper's supplmental
  information section.
\item
  3\% of authors made a promise to follow up and then did not follow
  through.
\item
  3\% of emails bounced.
\item
  2\% gave reasons why they could not share for ethical reasons, size
  limitations, or some other reason.
\end{itemize}
Of the 56 papers they deemed likely reproducible, the authors randomly
selected 22 and were able to replicate the results for all but 1, which
failed due to its reliance on software that was no longer available.

Hardwicke et al. (2018) conducted a study on the journal
\emph{Cognition}, where researchers compared the reproducibility of
published work both before and after the journal instituted an open data
policy, which required that authors make relevant research data publicly
available prior to publication of an article.

The researchers found a considerable increase in the proportion of data
available statements (in constrast to `data not available' statements,
which could be present due to privacy or security concerns) since the
implementation of the policy. Pre-open data policy, only 25\% of
articles had data available, while that number was a much higher 78\%
after the policy was put in place.

While the institution of an open data policy appears to have been
associated with a significant increase in the percentage of studies with
data available, further research indicates that the policy was perhaps
not as effective as intended. Many of the datasets were usable in
theory, but not in practice. Only 62\% of the articles with data
available statements had truly reusable datasets--in this case, meaning
that the data were accessible, complete, and understandable. Though this
is an increase from the pre-policy period, which saw 49\% of articles
with data availability statements as reusable in practice, it is still
far from ideal.

Combining these two data points indicates that \emph{less than half} of
articles published after the open data policy was instituted actually
contained truly usable data.

In this small sample of cases, we see that purely having a
reproducibility statement does not necessarily mean that all, or even a
majority, of published work will truly be reproducible.

\section{Limitations on Achieving Reproducibility in Scientific
Publishing}\label{limitations-on-achieving-reproducibility-in-scientific-publishing}

There are several reasons for this apparent divide between journal
reproducibility standards and the true proportion of submitted articles
that are truly reproducible. Some of these are challenges faced by the
article authors, while others are faced by the journal editors.

\subsection{Challenges for Authors}\label{challenges-for-authors}

Stodden, Seiler, \& Ma (2018b) conducted a survey asking over 7,700
researchers about one of the key characteristics of reproducibility --
open data -- and gathered information about the reasons why authors
found difficulties in making their data available to the public

The main challenges listed by respondents were as follows:
\begin{itemize}
\tightlist
\item
  46\% identified ``Organizing data in a presentable and useful way'' to
  be difficult.
\item
  37\% had been ``Unsure about copyright and licensing.''
\item
  33\% had problems with ``Not knowing which repository to use.''
\item
  26\% cited a ``Lack of time to deposit data.''
\item
  19\% found the ``Costs of sharing data'' to be high.
\end{itemize}
The relative frequency of these issues varied across several
characteristics, including author seniority, subject area, and
geographical location, though authors in all categories faced some
issues.

Beyond technical challenges, other reasons may lead authors to not place
their focus on reproducibility. For example, some researchers might fear
damage to their reputation if a reproduction attempt fails after they
have provided the necessary materials.

\textbf{ADD SOURCE}: LUPIA, ARTHUR, AND COLIN ELMAN. (2014) Openness in
Political Science: Data Access and Research Transparency. PS, Political
Science \& Politics 47 (1): 19--42.

Given the relatively high frequency of concern over achieving
reproducibility, it follows that researchers will not make the necessary
effort to do so if journal guidelines provide a way out. Policies that
\emph{recommend} the inclusion of data or that allow exceptions to open
data for certain reasons are likely to be associated with a lower
proportion of reproducible articles than those that make open data
mandatory.

\subsection{Challenges for Journals}\label{challenges-for-journals}

In addition to the challenges faced on the part of the authors, journal
reviewers face their own difficulties in ensuring reproducibility.

In order to make sure that all submitted articles comply with
reproducibility guidelines, reviewers must go through them one by one
and reproduce all of the results by hand using the provided materials.

This is an incredibly intensive process, as we will see in the example
of the \emph{American Journal for Political Science} (AJPS), whose
reproducibility policy was discussed prevously in Chapter 1.4.1.

Jacoby, Lafferty-Hess, \& Christian (2017) describe the AJPS process in
detail:

Acceptance of an article for publication in the AJPS is contingent on
successful reproducibility of any empirical analyses reported in the
article.

After an article is submitted, staff from a third party vendor hired by
AJPS go through the provided materials to ensure that they can be
preserved, understood, and used by others. They then run all of the
analyses in the article using the code, instructions, and data provided
by the authors and compare their results to the submitted articles.
Authors are then given an opportunity to resolve any issues that come
up. This process is repeated until reproducibility is ensured.

Although providing a significant benefit to the scientific community,
this thorough process is associated with high costs.

The verification process slows down the journal review process
significantly, adding a median 53 days to the publication workflow, as
many submitted articles require one or more rounds of resubmission (the
average number of resubmissions is 1.7). It is also quite labor
intensive, taking an average of 8 person-hours per manuscript to
reproduce the analyses and prepare the materials for public release and
adding significant monetary cost to AJPS.

Journals are often reluctant to take on such an intensive task due to
the drastically increased burden it places on reviewers and on the
publication's financial resources. This is particularly true given that
the number of submitted articles per year has been increasing over time
(Leopold (2015)). Every additional submission increases the burden of
achieving reproducibility, and with a large enough volume, the challenge
can quickly become seemingly impossible to manage reasonably.

As a result, journals often encourage reviewers to consider authors'
compliance with data sharing policies, but do not formally require that
they ensure it as a criterion for acceptance (Hrynaszkiewicz (2020)).

\section{Attempts to Address These
Limitations}\label{attempts-to-address-these-limitations}

The previous discussion makes clear that, although reproducibility is
critically important to scientific progress and academic journals are
taking steps to encourage it, the scientific community is far from
achieving the desired level of widespread reproducibility. In large
part, this appears due to the challenge and complexity of actually
achieving reproducibility. Those attempting to improve the
reproducibility of work can face issues with concerns over legality of
sharing data, large commitments of time or money, challenges finding a
good repository, and organizing all of the many components of their work
in an understandable way, among other things.

Additionally, science faces the additional challenge that many
publishers do not emphasize reproducibility at all, providing many
opportunities for all authors except those personally dedicated to
producing reproducible work to leave reproducibility by the wayside.
Many journals have no reproducibility requirements, and those that do
often do not take the necessary steps to ensure that they are actually
met.

These issues, however, are not impossible to overcome. Proponents of
reproducibility have taken action to help address them, both through
education on reproducibility and through software that helps simplify
the process of achieving it.

\subsection{Through Education}\label{through-education}

One way to address the reproducibility crisis is to educate data
analysts on the topic so that they are aware of both the concept of
reproducibility and how to achieve it in their own work. A natural place
to focus this education is early on in the data science training
pipeline as part of introductory or early-intermediate courses in
undergraduate and graduate data science porgrams.
(\url{https://arxiv.org/abs/1401.3269}) This sort of educational
integration has a variety of benefits:
\begin{itemize}
\item
  Bringing reproducibility into the discussion early on gives students
  the tools to add knowledge to their field in the best way possible
  before they actually conduct any substantive analysis on their own.
  (Bringing the Gold Standard into the Classroom: Replication in
  University Teaching - Nicole Janz). This produces many long run
  benefits, helping lessen the burden on promoting reproducibility
  placed on journals and increasing the number (and percentage) of
  researchers doing and promoting reproducible work.
\item
  If covered in detail as part of the data science curriculum,
  reproducibility will eventually come easily to students. If learned
  independently, without effective tools, reproducibility can
  challenging and even disheartening to try to understand and succeed
  at. Practicing in the classroom gives students the ability to fail
  without damaging their reputation, giving a great opportunity to truly
  learn and understand the concepts so that they feel capable of
  handling them when they begin their own research.
\item
  The application of grading to the topic provides an incentive for
  students to pay attention, learn, and absorb the information. This
  same incentive does not exist when researchers attempt to learn about
  reproducibility independently. In that situation, internal motivation,
  which may be weak in some individuals, is the only factor present to
  help promote success.
\end{itemize}
Several educators, primarily at the graduate level, have realized the
opportunity and taken steps to introduce reproducibility into their
courses.

The primary way of achieving this integration is through the assignment
of ``replication studies'' in standard methods choices. In these
assignments, students are given a published study and its supporting
materials and asked to reproduce the results. The most famous course of
this kind is Government 2001, taught by Gary King at Harvard University.
(Janz article) In King's course, students team up in small groups to
reproduce a previous study. To help ensure that their workflow is
reproducible, students are required to hand over their data and code to
another student team who then tries to reproduce their work once again.

In Thomas M. Carsey's intermediate statistics course at the University
of North Carolina at Chapel Hill, students must reproduce the findings
of a study by re-collecting the data from the original sources, then
must extend the study by building on the analysis.

Christopher Fariss of Penn State University asks his students to
replicate a research paper published in the last five years, noting that
students must describe the article and the ease in which the results
replicate.

The University of California at Berkeley has a similar course to Gary
King's Harvard course, where students each take a different piece of an
existing study to work on reproducing and have to ensure that their
piece fits with the piece of the next student.
(\url{https://berkeleysciencereview.com/2014/06/reproducible-collaborative-data-science/})

At the undergraduate level, rather than assign replication studies the
way many graduate schools tend to do, Smith College and Duke University
have both integrated reproducibility into their introductory courses
through the requirement that assignments be completed in the
\texttt{RMarkdown} code + narration format.
(\url{https://escholarship.org/uc/item/90b2f5xh}).

Another way to provide education on reproducibility is through the
creation of workshops that focus solely on the topic, rather than
through integration as just one part of a class.

For example, the University of Cambridge conducts a Replication
Workshop, where graduate students are asked replicate a paper in their
field over eight weekly sessions. When students encounter challenges,
such as authors not responding to queries for data or steps of the
analysis being poorly defined and explained, they gain a first hand
understanding of the consequences of poor transparency.

Workshops such as these are typically optional and not included as part
of the primary curriculum, however, so while they may cover the topic of
reproducibility in more detail than traditional courses, they often
reach fewer students.

In spite of all of the advantages that these educational tools provide,
``reproducibility training and assessment in data science education is
largely neglected, especially among undergraduates and Master's students
in professional schools\ldots{}, probably because the students are
usually considered to be non-research oriented.''
(\url{https://www.mitpressjournals.org/doi/pdf/10.1162/dint_a_00053})
While some examples of reproducibility education exist, they are
certainly not commonplace. However, given the increased discussion and
emphasis on reproducibility in academia over the past several years, it
is likely that this will change, particularly if methods are provided to
educators to make the integration of reproducibility into their courses
simple and relatively unburdensome.

\subsection{Through Software}\label{through-software}

Several researchers and members of the Statistical and Data Sciences
community have taken action to develop software focused on
reproducibility which removes some of the load on data analysts by
automating reproducibility processes and checking whether certain
components are achieved.

Much of this software has been written for users of the coding and data
analysis language \texttt{R}. \texttt{R} is very popular in the data
science community due to its open-source nature, accessibility,
extensive developer and user base, and statsitical analysis-specific
features.

Some of the existing software solutions are listed below:

\texttt{rrtools} (Marwick (2019)) addresses many of the issues discussed
in Marwick, Boettiger, \& Mullen (2018) by creating a basic reproducible
structure based on the \texttt{R} package format for a data analysis
project. In addition, it allows for isolation of the computer
environment using \texttt{Docker}, provides a method to capture
information about the versions of packages used in a project, contains
tools for generating a README file, and provides an option for users to
write tests to check that their functions operate as intended.

The \texttt{orderly} (FitzJohn et al. (2020)) package also focuses on
file structure, requiring the user to declare a desired project
structure (typically a step-by-step structure, where outputs from one
step are inputs into the next) at the beginning and then creating the
files necessary to achieve that structure. Its principal aim is to
automate many of the basic steps involved in writing analyses, making it
simple to:
\begin{enumerate}
\def\labelenumi{\arabic{enumi})}
\tightlist
\item
  Track all inputs into an analysis.
\item
  Store multiple versions of an analysis where it is repeated.
\item
  Track outputs of an analysis.
\item
  Create analyses that depend on the outputs of previous analyses.
\end{enumerate}
When projects have a variety of components, \texttt{orderly} makes it
easy to see inputs and outputs change with each re-run.

\texttt{workflowr}'s (Blischak, Carbonetto, \& Stephens (2019))
functionality is based around version control and making code easily
available online. It works to generate a website containing
time-stamped, versioned, and documented results. In addition, it manages
the session and package information of each analysis and controls random
number generation.

\texttt{checkers} (Ross, DeCicco, \& Randhawa (2018)) allows you to
create custom checks that examine different aspects of reproducibility.
It also contains some pre-built checks, such as seeing if users
reference packages that are less preferred to other similar ones and
ensuring that the project is under version control. \texttt{renv} (Ushey
\& RStudio (2020)) (formerly \texttt{packrat}) helps to make projects
more isolated, portable, and reproducible. It gives every project its
own private package library, makes it easy to install the packages the
project depends on if it is moved to another computer.

\texttt{drake} (OpenSci (2020)) analyzes workflows, skips steps where
results are up to date, utilizes optimized computing to complete the
rest of the steps, and provides evidence that results match the
underlying code and data.

Lastly, the \texttt{reproducible} (McIntire \& Chubaty (2020)) package
focuses on the concept of caching: saving information so that projects
can be run faster each time they are re-completed from the start.
\begin{center}\includegraphics[width=0.5\linewidth]{figure/packages} \end{center}

There have also been several \texttt{Continuous\ integration} tools
developed outside of R which can be used by those coding in almost any
language. These provide more general approaches to automated checking,
which can enhance reproducibility with minimal code.

For example, \texttt{wercker}---a command line tool that leverages
Docker---enables users to test whether their projects will successfully
compile when run on a variety of operating systems without access to the
user's local hard drive (Oracle Corporation (2019)).

\texttt{GitHub\ Actions} is integrated into GitHub and can be configured
to do similar checks on projects hosted in repositories.

\texttt{Travis\ CI} and \texttt{Circle\ CI} are popular continuous
integration tools that can also be used to check \texttt{R} code.
\begin{center}\includegraphics[width=0.8\linewidth]{figure/ci-tools} \end{center}

\chapter{\texorpdfstring{\texttt{fertile}: My Contribution To Addressing
Reproducibility}{fertile: My Contribution To Addressing Reproducibility}}\label{my-solution}

\section{Understanding The Gaps In Existing Reproducibility
Solutions}\label{understanding-the-gaps-in-existing-reproducibility-solutions}

Although the current state of reproducibility in academia is quite poor,
it is not an impossible challenge to overcome. The relative simplicity
of addressing reproducibility, particularly when compared with
replicability, makes it an ideal candidate for solution-building.
Although significant progress on addressing reproducibility on a
widespread scale is a long-term challenge, impactful forward
progress--if on a smaller scale--can be achieved in the short-term.

As we have seen, software developers, data scientists, and educators
around the world have realized this potential, taking steps to help
address the current crisis of reproducibility. Journals have put in
place guidelines for authors, statisticians have developed \texttt{R}
packages that help structure projects in a reproducible format, and
educators have begun integrate reproducibility exercises into their
courses.

However, many of these attempts to address reproducibility have
significant drawbacks associated with them. We have already explored the
issues with journal policies, both for authors and reviewers, in-depth.
In this section, we will consider the education and software solutions
and their associated challenges.

\subsection{In Education}\label{in-education}

The two primary concerns about the integration of reproducibility in
data science curricula revolve around time and difficulty.

As noted previously, the primary mode of teaching reproducibility is
through the assignment of replication studies where students must take
an existing study and go through the process of reproducing it
themselves, including contacting the author for all necessary materials,
rerunning code and analysis, and problem-solving when issues almost
certainly come up.

In addition to the time required for the professor to collect all of the
studies that students will be working on, the inclusion of such an
assignment places a significant burden on educators by taking up time
where they could be teaching other important material. Replication
studies, if done correctly, can take weeks for students to successfully
complete. The choice to give such assignments is therefore associated
with a significant opportunity cost which many professors are unwilling
to take.

Additionally, both replication studies assigned in class and replication
workshops outside of normal coursework require a working knowledge of
how to successfully complete and understand research. This makes them
inaccessible to individuals who are still in their undergraduate career
and may not yet have had an opportunity to conduct research or those who
are studying in non-research-focused technical programs.

In order to reach the widest variety of students possible, it is
necessary to develop a new method of teaching reproducibility that is
neither time consuming nor dependent on a prior understanding of the
research process.

\subsection{In Software}\label{in-software}

Previously, we considered several different types of software solutions:
packages designed for users of \texttt{R} and continuous integration
programs that can be used alongside a variety of coding languages.
Although these solutions have their advantages, they also have
significant drawbacks in terms of their ability to address
reproducibility on a widespread scale.

Many of the packages designed for \texttt{R} are incredibly narrow in
scope, with each effectively addressing a small component of
reproducibility: file structure, modularization of code, version
control, etc. They often succeed in their area of focus, but at the cost
of accessibility to a wider audience. Their functions are often quite
complex to use, and many steps must be completed to achieve the required
reproducibility goal. This cumbersome nature means that most
reproducibility packages currently available are not easily accessible
to users with minimal \texttt{R} experience, nor particularly useful to
those looking for quick and easy reproducibility checks. The significant
learning curve associated with them can also detract potential users who
may be interested in reproducibility but not willing to dedicate an
extensive amount of time to understanding the intricacies of software
operation.

Due to their generalized design, Continuous Integration tools do not
face the same issues with narrowness or complexity that \texttt{R}
packages struggle with. However, this generalizability provides its own
additional challenge. Since Continuous Integration tools are designed to
be accessible to a wide variety of users with different coding
preferences, they are not particularly customizable and lack the ability
to address features specific to certain programming languages.

Neither of these different software solutions appear to adequately
address the challenge of reproducibility. In order to be the most
effective, a piece of software must instead:
\begin{enumerate}
\def\labelenumi{\arabic{enumi})}
\tightlist
\item
  Be simple, with a small library of functions/tools that are
  straightforward to use.
\item
  Be accessible to a variety of users, with a relatively small learning
  curvey.
\item
  Be able to address a wide variety of aspects of reproducibility,
  rather than just one or two key issues.
\item
  Have features specific to a particular coding language that can
  address that language's unique challenges.
\item
  Be customizable, allowing users to choose for themselves which aspects
  of reproducibility they want to focus on.
\end{enumerate}
\section{\texorpdfstring{\texttt{fertile}, An R Package Creating Optimal
Conditions For
Reproducibility}{fertile, An R Package Creating Optimal Conditions For Reproducibility}}\label{fertile-an-r-package-creating-optimal-conditions-for-reproducibility}

What if it were possible to address the existing issues with both
educational and software reproducibility solutions simultaneously?

That is where my work comes in. In an attempt to produce meaningful
change in the field of reproducibility, I have been developing
\texttt{fertile}, a software package designed for \texttt{R} which helps
users create optimal conditions for achieving reproducibility in their
projects.

\texttt{fertile} attempts to address the gaps in existing reproducbility
solutions by combining software and education in one product. The
package provides a set of simple, easy-to-learn tools that, rather than
focus intensely on a specific area like other software programs, provide
some information about a wide variety of aspects influencing
reproducibility. It is also designed to be incredibly flexible, offering
benefits to users at any stage in the data analysis workflow and
providing users with the option to select which aspects of
reproducibility they want to focus on. \texttt{fertile} also contains
several\texttt{R}-specific features, which address certain aspects of
reproducibility that can be missed by external project development
tools.

In addition, \texttt{fertile} is designed to be educational, teaching
its users about the components of reproducibility and how to achieve
them in their work. The package provides users with detailed reports on
the aspects of reproducibility where their projects fell short,
identifying the root causes and, in many cases, providing a recommended
solution.

\texttt{fertile} is structured in such a way as to be understandable and
operable to individuals of any skill level, from students in their first
undergraduate data science course to experienced PhD statisticians. The
majority of its tools can be accessed in only a handful of functions
with minimal required arguments. This simplicity makes the process of
achieving and learning about reproducibility accessible to a wide
audience in a way that complex software programs or graduate courses
requiring an advanced knowledge of research methods do not.

Reproducibility is significantly easier to achieve when all of the tools
necessary to do so are located in one place. \texttt{fertile} provides
this optimal all-inclusive structure, addressing all six of the primary
components of reproducibility discussed in Chapter 1. We will consider
\texttt{fertile}'s treatment of each of these components in turn:

\subsection{Component 1: Accessibility of Project
Files}\label{component-1-accessibility-of-project-files}
\begin{itemize}
\tightlist
\item
  Data (raw and/or processed)
\item
  Metadata
\item
  Code
\item
  Related Software
\end{itemize}
What we \emph{have}:
\begin{itemize}
\tightlist
\item
  proj\_report() gives the files present in the directory
\end{itemize}
What we \emph{need}:
\begin{itemize}
\tightlist
\item
  can we report a checklist of the types of files we have available?
\item
  it should check for the existence of a code file and data file and
  also readme
\end{itemize}
\subsection{Component 2: Organized Project
Structure}\label{component-2-organized-project-structure}
\begin{itemize}
\tightlist
\item
  Separate folders for different file types.
\item
  No extraneous files.
\item
  Minimal clutter.
\end{itemize}
What we \emph{have}:
\begin{itemize}
\tightlist
\item
  has\_tidy\_media/raw\_data/images/code/data/scripts
\item
  has\_proj\_root/has\_no\_nested\_proj\_root
\item
  has\_only\_used\_files
\item
  proj\_move\_files (and associated suggestions in proj\_analyze())
\end{itemize}
What we \emph{need}:
\begin{itemize}
\tightlist
\item
  nothing!
\end{itemize}
\subsection{Component 3: Documentation}\label{component-3-documentation}
\begin{itemize}
\tightlist
\item
  Files are clearly named, preferably in a way where the order in which
  they should be run is clear.
\item
  A README is present.
\item
  Code contains comments.
\item
  Software dependencies are noted.
\end{itemize}
What we \emph{have}:
\begin{itemize}
\tightlist
\item
  has\_readme
\item
  has\_clear\_build\_chain
\item
  list of packages in proj\_analyze/report
\item
  proj\_pkg\_script install script generator
\end{itemize}
What we \emph{need}:
\begin{itemize}
\tightlist
\item
  possibly a makefile generator? or improvements to order checking?
\item
  a way to note the sessioninfo and package numbers of a project
\item
  check if code contains comments
\end{itemize}
\subsection{Component 4: File Paths}\label{component-4-file-paths}
\begin{itemize}
\tightlist
\item
  No absolute paths.
\item
  No paths leading to locations outside of a project's directory.
\item
  Only relative paths, pointing to locations within a project's
  directory, are permitted.
\end{itemize}
What we \emph{have}:
\begin{itemize}
\tightlist
\item
  proj\_analyze\_paths
\item
  check\_path
\end{itemize}
What we \emph{need}:
\begin{itemize}
\tightlist
\item
  Nothing!
\end{itemize}
\subsection{Component 5: Randomness}\label{component-5-randomness}
\begin{itemize}
\tightlist
\item
  If randomness is used in code, a seed must also be set.
\end{itemize}
What we \emph{have}:
\begin{itemize}
\tightlist
\item
  has\_no\_randomness
\end{itemize}
What we \emph{need}:
\begin{itemize}
\tightlist
\item
  Nothing!
\end{itemize}
\subsection{Component 6: Readability and
Style}\label{component-6-readability-and-style}
\begin{itemize}
\tightlist
\item
  Though not mentioned in the sources described previously, it is also
  important that code be written in a coherent style. This is because
  code that conforms to a style guide or is written in a consistent
  dialect is easier to read, simplifying the process of following a
  researcher's work from beginning to end (Hermans \& Aldewereld
  (2017)).
\end{itemize}
What we \emph{have}:
\begin{itemize}
\tightlist
\item
  has\_no\_lint
\end{itemize}
What we \emph{need}:
\begin{itemize}
\tightlist
\item
  nothing?? Ben's thoughts might be good here
\end{itemize}
Much of the available literature focuses on file structure,
organization, and naming, and \texttt{fertile}'s features are consistent
with this. Marwick et al. (2018) provide the framework for file
structure that \texttt{fertile} is based on: a structure similar to that
of an \texttt{R} package (R-Core-Team (2020), Wickham (2015)), with an
\texttt{R} folder, as well as \texttt{data}, \texttt{data-raw},
\texttt{inst}, and \texttt{vignettes}.

\texttt{fertile} is designed to be used on data analyses organized as
\texttt{R} Projects (i.e.~directories containing an \texttt{.Rproj}
file). Once an \texttt{R} Project is created, \texttt{fertile} provides
benefits throughout the data analysis process, both during development
as well as after the fact. \texttt{fertile} achieves this by operating
in two modes: proactively (to prevent reproducibility mistakes from
happening in the first place), and retroactively (analyzing code that
has already been written for potential problems).

\subsection{Proactive Use}\label{proactive-use}

Proactively, the package identifies potential mistakes as they are made
by the user and outputs an informative message as well as a recommended
solution. For example, \texttt{fertile} catches when a user passes a
potentially problematic file path---such as an absolute path, or a path
that points to a location outside of the project directory---to a
variety of common input/output functions operating on many different
file types.

\footnotesize
\begin{Shaded}
\begin{Highlighting}[]
\KeywordTok{library}\NormalTok{(fertile)}
\KeywordTok{file.exists}\NormalTok{(}\StringTok{"~/Desktop/my_data.csv"}\NormalTok{)}
\end{Highlighting}
\end{Shaded}
\begin{verbatim}
[1] TRUE
\end{verbatim}
\begin{Shaded}
\begin{Highlighting}[]
\KeywordTok{read.csv}\NormalTok{(}\StringTok{"~/Desktop/my_data.csv"}\NormalTok{)}
\end{Highlighting}
\end{Shaded}
\begin{verbatim}
    mpg cyl  disp  hp drat    wt  qsec vs am gear carb
1  21.0   6 160.0 110 3.90 2.620 16.46  0  1    4    4
2  21.0   6 160.0 110 3.90 2.875 17.02  0  1    4    4
3  22.8   4 108.0  93 3.85 2.320 18.61  1  1    4    1
4  21.4   6 258.0 110 3.08 3.215 19.44  1  0    3    1
5  18.7   8 360.0 175 3.15 3.440 17.02  0  0    3    2
6  18.1   6 225.0 105 2.76 3.460 20.22  1  0    3    1
7  14.3   8 360.0 245 3.21 3.570 15.84  0  0    3    4
8  24.4   4 146.7  62 3.69 3.190 20.00  1  0    4    2
9  22.8   4 140.8  95 3.92 3.150 22.90  1  0    4    2
10 19.2   6 167.6 123 3.92 3.440 18.30  1  0    4    4
11 17.8   6 167.6 123 3.92 3.440 18.90  1  0    4    4
12 16.4   8 275.8 180 3.07 4.070 17.40  0  0    3    3
13 17.3   8 275.8 180 3.07 3.730 17.60  0  0    3    3
14 15.2   8 275.8 180 3.07 3.780 18.00  0  0    3    3
15 10.4   8 472.0 205 2.93 5.250 17.98  0  0    3    4
16 10.4   8 460.0 215 3.00 5.424 17.82  0  0    3    4
17 14.7   8 440.0 230 3.23 5.345 17.42  0  0    3    4
18 32.4   4  78.7  66 4.08 2.200 19.47  1  1    4    1
19 30.4   4  75.7  52 4.93 1.615 18.52  1  1    4    2
20 33.9   4  71.1  65 4.22 1.835 19.90  1  1    4    1
21 21.5   4 120.1  97 3.70 2.465 20.01  1  0    3    1
22 15.5   8 318.0 150 2.76 3.520 16.87  0  0    3    2
23 15.2   8 304.0 150 3.15 3.435 17.30  0  0    3    2
24 13.3   8 350.0 245 3.73 3.840 15.41  0  0    3    4
25 19.2   8 400.0 175 3.08 3.845 17.05  0  0    3    2
26 27.3   4  79.0  66 4.08 1.935 18.90  1  1    4    1
27 26.0   4 120.3  91 4.43 2.140 16.70  0  1    5    2
28 30.4   4  95.1 113 3.77 1.513 16.90  1  1    5    2
29 15.8   8 351.0 264 4.22 3.170 14.50  0  1    5    4
30 19.7   6 145.0 175 3.62 2.770 15.50  0  1    5    6
31 15.0   8 301.0 335 3.54 3.570 14.60  0  1    5    8
32 21.4   4 121.0 109 4.11 2.780 18.60  1  1    4    2
\end{verbatim}
\begin{Shaded}
\begin{Highlighting}[]
\KeywordTok{read.csv}\NormalTok{(}\StringTok{"../../../Desktop/my_data.csv"}\NormalTok{)}
\end{Highlighting}
\end{Shaded}
\begin{verbatim}
    mpg cyl  disp  hp drat    wt  qsec vs am gear carb
1  21.0   6 160.0 110 3.90 2.620 16.46  0  1    4    4
2  21.0   6 160.0 110 3.90 2.875 17.02  0  1    4    4
3  22.8   4 108.0  93 3.85 2.320 18.61  1  1    4    1
4  21.4   6 258.0 110 3.08 3.215 19.44  1  0    3    1
5  18.7   8 360.0 175 3.15 3.440 17.02  0  0    3    2
6  18.1   6 225.0 105 2.76 3.460 20.22  1  0    3    1
7  14.3   8 360.0 245 3.21 3.570 15.84  0  0    3    4
8  24.4   4 146.7  62 3.69 3.190 20.00  1  0    4    2
9  22.8   4 140.8  95 3.92 3.150 22.90  1  0    4    2
10 19.2   6 167.6 123 3.92 3.440 18.30  1  0    4    4
11 17.8   6 167.6 123 3.92 3.440 18.90  1  0    4    4
12 16.4   8 275.8 180 3.07 4.070 17.40  0  0    3    3
13 17.3   8 275.8 180 3.07 3.730 17.60  0  0    3    3
14 15.2   8 275.8 180 3.07 3.780 18.00  0  0    3    3
15 10.4   8 472.0 205 2.93 5.250 17.98  0  0    3    4
16 10.4   8 460.0 215 3.00 5.424 17.82  0  0    3    4
17 14.7   8 440.0 230 3.23 5.345 17.42  0  0    3    4
18 32.4   4  78.7  66 4.08 2.200 19.47  1  1    4    1
19 30.4   4  75.7  52 4.93 1.615 18.52  1  1    4    2
20 33.9   4  71.1  65 4.22 1.835 19.90  1  1    4    1
21 21.5   4 120.1  97 3.70 2.465 20.01  1  0    3    1
22 15.5   8 318.0 150 2.76 3.520 16.87  0  0    3    2
23 15.2   8 304.0 150 3.15 3.435 17.30  0  0    3    2
24 13.3   8 350.0 245 3.73 3.840 15.41  0  0    3    4
25 19.2   8 400.0 175 3.08 3.845 17.05  0  0    3    2
26 27.3   4  79.0  66 4.08 1.935 18.90  1  1    4    1
27 26.0   4 120.3  91 4.43 2.140 16.70  0  1    5    2
28 30.4   4  95.1 113 3.77 1.513 16.90  1  1    5    2
29 15.8   8 351.0 264 4.22 3.170 14.50  0  1    5    4
30 19.7   6 145.0 175 3.62 2.770 15.50  0  1    5    6
31 15.0   8 301.0 335 3.54 3.570 14.60  0  1    5    8
32 21.4   4 121.0 109 4.11 2.780 18.60  1  1    4    2
\end{verbatim}
\normalsize

\texttt{fertile} is even more aggressive with functions (like
\texttt{setwd()}) that are almost certain to break reproducibility,
causing them to throw errors that prevent their execution and providing
recommendations for better alternatives.

\footnotesize
\begin{Shaded}
\begin{Highlighting}[]
\KeywordTok{setwd}\NormalTok{(}\StringTok{"~/Desktop"}\NormalTok{)}
\end{Highlighting}
\end{Shaded}
\begin{verbatim}
Error: setwd() is likely to break reproducibility. Use here::here() instead.
\end{verbatim}
\normalsize

These proactive warning features are activated immediately after
attaching the \texttt{fertile} package and require no additional effort
by the user.

\subsection{Retroactive Use}\label{retroactive-use}

Retroactively, \texttt{fertile} analyzes potential obstacles to
reproducibility in an RStudio Project (i.e., a directory that contains
an \texttt{.Rproj} file). The package considers several different
aspects of the project which may influence reproducibility, including
the directory structure, file paths, and whether randomness is used
thoughtfully.

The end products of these analyses are reproducibility reports
summarizing a project's adherence to reproducibility standards and
recommending remedies for where the project falls short. For example,
\texttt{fertile} might identify the use of randomness in code and
recommend setting a seed if one is not present.

Users can access the majority of \texttt{fertile}'s retroactive features
through two primary functions, \texttt{proj\_check()} and
\texttt{proj\_analyze()}.

The \texttt{proj\_check()} function runs fifteen different
reproducibility tests, noting which ones passed, which ones failed, the
reason for failure, a recommended solution, and a guide to where to look
for help. These tests include: looking for a clear build chain, checking
to make sure the root level of the project is clear of clutter,
confirming that there are no files present that are not being directly
used by or created by the code, and looking for uses of randomness that
do not have a call to \texttt{set.seed()} present. A full list is
provided below:

\footnotesize
\begin{Shaded}
\begin{Highlighting}[]
\KeywordTok{list_checks}\NormalTok{()}
\end{Highlighting}
\end{Shaded}
\begin{verbatim}
-- The available checks in `fertile` are as follows: -------------------------- fertile 0.0.0.9027 --
\end{verbatim}
\begin{verbatim}
 [1] "has_tidy_media"          "has_tidy_images"        
 [3] "has_tidy_code"           "has_tidy_raw_data"      
 [5] "has_tidy_data"           "has_tidy_scripts"       
 [7] "has_readme"              "has_no_lint"            
 [9] "has_proj_root"           "has_no_nested_proj_root"
[11] "has_only_used_files"     "has_clear_build_chain"  
[13] "has_no_absolute_paths"   "has_only_portable_paths"
[15] "has_no_randomness"      
\end{verbatim}
\normalsize

Subsets of the fifteen tests can be invoked using the
\texttt{tidyselect} helper functions (Henry \& Wickham (2020)) in
combination with the more limited \texttt{proj\_check\_some()} function.

\footnotesize
\begin{Shaded}
\begin{Highlighting}[]
\NormalTok{proj_dir <-}\StringTok{ "project_miceps"}
\end{Highlighting}
\end{Shaded}
\begin{Shaded}
\begin{Highlighting}[]
\KeywordTok{proj_check_some}\NormalTok{(proj_dir, }\KeywordTok{contains}\NormalTok{(}\StringTok{"paths"}\NormalTok{))}
\end{Highlighting}
\end{Shaded}
\begin{verbatim}
-- Compiling... --------------------------------------------------------------- fertile 0.0.0.9027 --
\end{verbatim}
\begin{verbatim}
-- Rendering R scripts... ----------------------------------------------------- fertile 0.0.0.9027 --
\end{verbatim}
\begin{verbatim}
-- Running reproducibility checks --------------------------------------------- fertile 0.0.0.9027 --
\end{verbatim}
\begin{verbatim}
v Checking for no absolute paths
\end{verbatim}
\begin{verbatim}
v Checking for only portable paths
\end{verbatim}
\begin{verbatim}
-- Summary of fertile checks -------------------------------------------------- fertile 0.0.0.9027 --
\end{verbatim}
\begin{verbatim}
v Reproducibility checks passed: 2
\end{verbatim}
\normalsize

Each test can also be run individually by calling the function matching
its check name.

The \texttt{proj\_analyze()} function creates a report documenting the
structure of a data analysis project. This report contains information
about all packages referenced in code, the files present in the
directory and their types, suggestions for moving files to create a more
organized structure, and a list of reproducibility-breaking file paths
used in code.

\footnotesize
\begin{Shaded}
\begin{Highlighting}[]
\KeywordTok{proj_analyze}\NormalTok{(proj_dir)}
\end{Highlighting}
\end{Shaded}
\begin{verbatim}
-- Analysis of reproducibility for project_miceps ----------------------------- fertile 0.0.0.9027 --
\end{verbatim}
\begin{verbatim}
--   Packages referenced in source code --------------------------------------- fertile 0.0.0.9027 --
\end{verbatim}
\begin{verbatim}
# A tibble: 9 x 3
  package       N used_in                    
  <chr>     <int> <chr>                      
1 broom         1 project_miceps/analysis.Rmd
2 dplyr         1 project_miceps/analysis.Rmd
3 ggplot2       1 project_miceps/analysis.Rmd
4 purrr         1 project_miceps/analysis.Rmd
5 readr         1 project_miceps/analysis.Rmd
6 rmarkdown     1 project_miceps/analysis.Rmd
7 skimr         1 project_miceps/analysis.Rmd
8 stargazer     1 project_miceps/analysis.Rmd
9 tidyr         1 project_miceps/analysis.Rmd
\end{verbatim}
\begin{verbatim}
--   Files present in directory ----------------------------------------------- fertile 0.0.0.9027 --
\end{verbatim}
\begin{verbatim}
# A tibble: 10 x 4
   file               ext        size mime                                      
   <fs::path>         <chr> <fs::byt> <chr>                                     
 1 Estrogen_Receptor~ docx     10.97K application/vnd.openxmlformats-officedocu~
 2 citrate_v_time.png png     188.98K image/png                                 
 3 proteins_v_time.p~ png     377.53K image/png                                 
 4 Blot_data_updated~ csv      14.43K text/csv                                  
 5 CS_data_redone.csv csv       7.39K text/csv                                  
 6 mice.csv           csv      14.33K text/csv                                  
 7 README.md          md           39 text/markdown                             
 8 miceps.Rproj       Rproj       204 text/rstudio                              
 9 analysis.Rmd       Rmd       4.94K text/x-markdown                           
10 tmp-pdfcrop-10664~ tex       3.32K text/x-tex                                
\end{verbatim}
\begin{verbatim}
--   Suggestions for moving files --------------------------------------------- fertile 0.0.0.9027 --
\end{verbatim}
\begin{verbatim}
# A tibble: 8 x 3
  path_rel           dir_rel    cmd                                             
  <fs::path>         <fs::path> <chr>                                           
1 Blot_data_updated~ data-raw   file_move('project_miceps/Blot_data_updated.csv~
2 CS_data_redone.csv data-raw   file_move('project_miceps/CS_data_redone.csv', ~
3 Estrogen_Receptor~ inst/other file_move('project_miceps/Estrogen_Receptors.do~
4 analysis.Rmd       vignettes  file_move('project_miceps/analysis.Rmd', fs::di~
5 citrate_v_time.png inst/image file_move('project_miceps/citrate_v_time.png', ~
6 mice.csv           data-raw   file_move('project_miceps/mice.csv', fs::dir_cr~
7 proteins_v_time.p~ inst/image file_move('project_miceps/proteins_v_time.png',~
8 tmp-pdfcrop-10664~ inst/text  file_move('project_miceps/tmp-pdfcrop-10664.tex~
\end{verbatim}
\begin{verbatim}
--   Problematic paths logged ------------------------------------------------- fertile 0.0.0.9027 --
\end{verbatim}
\begin{verbatim}
NULL
\end{verbatim}
\normalsize

\subsection{Logging}\label{logging}

\texttt{fertile} also contains logging functionality, which records
commands run in the console that have the potential to affect
reproducibility, enabling users to look at their past history at any
time. The package focuses mostly on package loading and file opening,
noting which function was used, the path or package it referenced, and
the timestamp at which that event happened. Users can access the log
recording their commands at any time via the \texttt{log\_report()}
function:

\footnotesize
\begin{Shaded}
\begin{Highlighting}[]
\KeywordTok{log_report}\NormalTok{()}
\end{Highlighting}
\end{Shaded}
\begin{verbatim}
# A tibble: 8 x 4
  path            path_abs                         func      timestamp          
  <chr>           <chr>                            <chr>     <dttm>             
1 package:remotes <NA>                             base::re~ 2020-09-16 19:49:31
2 package:thesis~ <NA>                             base::re~ 2020-09-16 19:49:31
3 package:thesis~ <NA>                             base::li~ 2020-09-16 19:49:31
4 ~/Desktop/my_d~ ~/Desktop/my_data.csv            utils::r~ 2020-09-16 19:51:30
5 ../../../Deskt~ /Users/audreybertin/Desktop/my_~ utils::r~ 2020-09-16 19:51:30
6 package:purrr   <NA>                             base::li~ 2020-09-16 19:51:39
7 package:forcats <NA>                             base::li~ 2020-09-16 19:51:40
8 project_miceps~ /Users/audreybertin/Documents/t~ readr::r~ 2020-09-16 19:51:40
\end{verbatim}
\normalsize

The log, if not managed, can grow very long over time. For users who do
not desire such functionality, \texttt{log\_clear()} provides a way to
erase the log and start over.

\subsection{Utility Functions}\label{utility-functions}

\texttt{fertile} also provides several useful utility functions that may
assist with the process of data analysis.

\subsection{File Path Management}\label{file-path-management}

The \texttt{check\_path()} function analyzes a vector of paths (or a
single path) to determine whether there are any absolute paths or paths
that lead outside the project directory.

\footnotesize
\begin{Shaded}
\begin{Highlighting}[]
\CommentTok{# Path inside the directory}
\KeywordTok{check_path}\NormalTok{(}\StringTok{"project_miceps"}\NormalTok{)}
\end{Highlighting}
\end{Shaded}
\begin{verbatim}
# A tibble: 0 x 3
# ... with 3 variables: path <chr>, problem <chr>, solution <chr>
\end{verbatim}
\begin{Shaded}
\begin{Highlighting}[]
\CommentTok{# Absolute path (current working directory)}
\KeywordTok{check_path}\NormalTok{(}\KeywordTok{getwd}\NormalTok{())}
\end{Highlighting}
\end{Shaded}
\begin{verbatim}
Error: Detected absolute paths
\end{verbatim}
\begin{Shaded}
\begin{Highlighting}[]
\CommentTok{# Path outside the directory}
\KeywordTok{check_path}\NormalTok{(}\StringTok{"../fertile.Rmd"}\NormalTok{)}
\end{Highlighting}
\end{Shaded}
\begin{verbatim}
Error: Detected paths that lead outside the project directory
\end{verbatim}
\normalsize

\subsection{File Types}\label{file-types}

There are several functions that can be used to check the type of a
file:

\footnotesize
\begin{Shaded}
\begin{Highlighting}[]
\KeywordTok{is_data_file}\NormalTok{(fs}\OperatorTok{::}\KeywordTok{path}\NormalTok{(proj_dir, }\StringTok{"mice.csv"}\NormalTok{))}
\end{Highlighting}
\end{Shaded}
\begin{verbatim}
[1] TRUE
\end{verbatim}
\begin{Shaded}
\begin{Highlighting}[]
\KeywordTok{is_image_file}\NormalTok{(fs}\OperatorTok{::}\KeywordTok{path}\NormalTok{(proj_dir, }\StringTok{"proteins_v_time.png"}\NormalTok{))}
\end{Highlighting}
\end{Shaded}
\begin{verbatim}
[1] TRUE
\end{verbatim}
\begin{Shaded}
\begin{Highlighting}[]
\KeywordTok{is_text_file}\NormalTok{(fs}\OperatorTok{::}\KeywordTok{path}\NormalTok{(proj_dir, }\StringTok{"README.md"}\NormalTok{))}
\end{Highlighting}
\end{Shaded}
\begin{verbatim}
[1] TRUE
\end{verbatim}
\begin{Shaded}
\begin{Highlighting}[]
\KeywordTok{is_r_file}\NormalTok{(fs}\OperatorTok{::}\KeywordTok{path}\NormalTok{(proj_dir, }\StringTok{"analysis.Rmd"}\NormalTok{))}
\end{Highlighting}
\end{Shaded}
\begin{verbatim}
[1] TRUE
\end{verbatim}
\normalsize

\subsection{Temporary Directories}\label{temporary-directories}

The \texttt{sandbox()} function allows the user to make a copy of their
project in a temporary directory. This can be useful for ensuring that
projects run properly when access to the local file system is removed.

\footnotesize
\begin{Shaded}
\begin{Highlighting}[]
\NormalTok{proj_dir}
\NormalTok{fs}\OperatorTok{::}\KeywordTok{dir_ls}\NormalTok{(proj_dir) }\OperatorTok\StringTok{ }\KeywordTok{head}\NormalTok{(}\DecValTok{3}\NormalTok{)}
\end{Highlighting}
\end{Shaded}
\begin{verbatim}
project_miceps/Blot_data_updated.csv
project_miceps/CS_data_redone.csv
project_miceps/Estrogen_Receptors.docx
\end{verbatim}
\begin{Shaded}
\begin{Highlighting}[]
\NormalTok{temp_dir <-}\StringTok{ }\KeywordTok{sandbox}\NormalTok{(proj_dir)}
\NormalTok{temp_dir}
\NormalTok{fs}\OperatorTok{::}\KeywordTok{dir_ls}\NormalTok{(temp_dir) }\OperatorTok\StringTok{ }\KeywordTok{head}\NormalTok{(}\DecValTok{3}\NormalTok{)}
\end{Highlighting}
\end{Shaded}
\begin{verbatim}
/var/folders/v6/f62qz88s0sd5n3yqw9d8sb300000gn/T/RtmpW7Yquk/project_miceps/
Blot_data_updated.csv
/var/folders/v6/f62qz88s0sd5n3yqw9d8sb300000gn/T/RtmpW7Yquk/project_miceps/
CS_data_redone.csv
/var/folders/v6/f62qz88s0sd5n3yqw9d8sb300000gn/T/RtmpW7Yquk/project_miceps/
Estrogen_Receptors.docx
\end{verbatim}
\normalsize

\subsection{Managing Project
Dependencies}\label{managing-project-dependencies}

One of the challenges with ensuring that work is reproducible is the
issue of dependencies. Many data analysis projects reference a variety
of \texttt{R} packages in their code. When such projects are shared with
other users who may not have the required packages downloaded, it can
cause errors that prevent the project from running properly.

The \texttt{proj\_pkg\_script()}\} function assists with this issue by
making it simple and fast to download dependencies. When run on an
\texttt{R} project directory, the function creates a \texttt{.R} script
file that contains the code needed to install all of the packages
referenced in the project, differentiating between packages located on
CRAN and those located on GitHub.

\footnotesize
\begin{Shaded}
\begin{Highlighting}[]
\NormalTok{install_script <-}\StringTok{ }\KeywordTok{proj_pkg_script}\NormalTok{(proj_dir)}
\KeywordTok{cat}\NormalTok{(}\KeywordTok{readChar}\NormalTok{(install_script, }\FloatTok{1e5}\NormalTok{))}
\end{Highlighting}
\end{Shaded}
\begin{verbatim}
# Run this script to install the required packages for this
R project.
# Packages hosted on CRAN...
install.packages(c( 'broom', 'dplyr', 'ggplot2', 'purrr',
'readr', 'rmarkdown', 'skimr', 'stargazer', 'tidyr' ))
# Packages hosted on GitHub...
\end{verbatim}
\normalsize

\section{\texorpdfstring{How \texttt{fertile}
Works}{How fertile Works}}\label{how-fertile-works}

Much of the functionality in \texttt{fertile} is achieved by writing
\texttt{shims} \textbf{link to wikipedia page here}. \texttt{fertile}'s
shimmed functions intercept the user's commands and perform various
logging and checking tasks before executing the desired function. Our
process is:
\begin{enumerate}
\def\labelenumi{\arabic{enumi}.}
\item
  Identify an \texttt{R} function that is likely to be involved in
  operations that may break reproducibility. Popular functions
  associated with only one package (e.g., \texttt{read\_csv()} from
  \texttt{readr}) are ideal candidates.
\item
  Create a function in \texttt{fertile} with the same name that takes
  the same arguments (and always the dots \texttt{...}).
\item
  Write this new function so that it:
\end{enumerate}
\begin{enumerate}
\def\labelenumi{\alph{enumi})}
\tightlist
\item
  captures any arguments,
\item
  logs the name of the function called,
\item
  performs any checks on these arguments, and
\item
  calls the original function with the original arguments. Except where
  warranted, the execution looks the same to the user as if they were
  calling the original function.
\end{enumerate}
Most shims are quite simple and look something like what is shown below
for \texttt{read\_csv()}.

\footnotesize
\begin{Shaded}
\begin{Highlighting}[]
\NormalTok{fertile}\OperatorTok{::}\NormalTok{read_csv}
\end{Highlighting}
\end{Shaded}
\begin{verbatim}
function(file, ...) {
  if (interactive_log_on()) {
    log_push(file, "readr::read_csv")
    check_path_safe(file)
    readr::read_csv(file, ...)
  }
}
<bytecode: 0x7ff32a9607b8>
<environment: namespace:fertile>
\end{verbatim}
\normalsize

\texttt{fertile} shims many common functions, including those that read
in a variety of data types, write data, and load packages. This works
both proactively and retroactively, as the shimmed functions written in
\texttt{fertile} are activated both when the user is coding
interactively and when a file containing code is rendered.

In order to ensure that the \texttt{fertile} versions of functions
(``shims'') always supersede (``mask'') their original namesakes when
called, \texttt{fertile} uses its own shims of the \texttt{library} and
\texttt{require} functions to manipulate the \texttt{R} \texttt{search}
path so that it is always located in the first position. In the
\texttt{fertile} version of \texttt{library()}, we detach
\texttt{fertile} from the search path, load the requested package, and
then re-attach \texttt{fertile}. This ensures that when a user executes
a command, \texttt{R} will check \texttt{fertile} for a matching
function before considering other packages. While it is possible that
this shifty behavior could lead to unintended consequences, our goal is
to catch a good deal of problems before they become problematic. Users
can easily disable \texttt{fertile} by detaching it, or not loading it
in the first place.

\section{\texorpdfstring{\texttt{fertile} in Practice: Experimental
Results From Smith College Student
Use}{fertile in Practice: Experimental Results From Smith College Student Use}}\label{fertile-in-practice-experimental-results-from-smith-college-student-use}

\texttt{fertile} is designed to: 1) be simple enough that users with
minimal \texttt{R} experience can use the package without issue, 2)
increase the reproducibility of work produced by its users, and 3)
educate its users on why their work is or is not reproducible and
provide guidance on how to address any problems.

To test \texttt{fertile}'s effectiveness, we began an initial randomized
control trial of the package on an introductory undergraduate data
science course at Smith College in Spring 2020 \textbf{ADD FOOTNOTE}
(This study was approved by Smith College IRB, Protocol \#19-032).

The experiment was structured as follows:

1.Students are given a form at the start of the semester asking whether
they consent to participate in a study on data science education. In
order to successfully consent, they must provide their system username,
collected through the command \texttt{Sys.getenv("LOGNAME")}. To
maintain privacy the results are then transformed into a hexadecimal
string via the \texttt{md5()} hashing function.
\begin{enumerate}
\def\labelenumi{\arabic{enumi}.}
\setcounter{enumi}{1}
\item
  These hexadecimal strings are then randomly assigned into equally
  sized groups, one experimental group that receives the features of
  \texttt{fertile} and one group that receives a control.
\item
  The students are then asked to download a package called
  \texttt{sds192} (the course number and prefix), which was created for
  the purpose of this trial. It leverages an \texttt{.onAttach()}
  function to scan the \texttt{R} environment and collect the username
  of the user who is loading the package and run it through the same
  hashing algorithm as used previously. It then identifies whether that
  user belongs to the experimental or the control group. Depending on
  the group they are in, they receive a different version of the
  package.
\item
  The experimental group receives the basic \texttt{sds192} package,
  which consists of some data sets and \texttt{R} Markdown templates
  necessary for completing homework assignments and projects in the
  class, but also has \texttt{fertile} installed and loaded silently in
  the background. The package's proactive features are enabled, and
  therefore users will receive warning messages when they use absolute
  or non-portable paths or attempt to change their working directory.
  The control group receives only the basic \texttt{sds192} package,
  including its data sets and \texttt{R} Markdown templates. All
  students from both groups then use their version of the package
  throughout the semester on a variety of projects.
\item
  Both groups are given a short quiz on different components of
  reproducibility that are intended to be taught by \texttt{fertile} at
  both the beginning and end of the semester. Their scores are then
  compared to see whether one group learned more than the other group or
  whether their scores were essentially equivalent. Additionally, for
  every homework assignment submitted, the professor takes note of
  whether or not the project compiles successfully.
\end{enumerate}
Based on the results, we hope to determine whether \texttt{fertile} was
successful at achieving its intended goals. A lack of notable difference
between the \emph{experimental} and \emph{control} groups in terms of
the number of code-related questions asked throughout the semester would
indicate that \texttt{fertile} achieved its goal of simplicity. A higher
average for the \emph{experimental} group in terms of the number of
homework assignments that compiled successfully would indicate that
\texttt{fertile} was successful in increasing reproducibility. A greater
increase over the semester in the reproducibility quiz scores for
students in the \emph{experimental} group compared with the
\emph{control} group would indicate that \texttt{fertile} achieved its
goal of educating users on reproducibility. Success according to these
metrics would provide evidence showing \texttt{fertile}'s benefit as
tool to help educators introduce reproducibility concepts in the
classroom.

\chapter{Incorporating Reproducibility Tools Into The Greater Data
Science Community}\label{applications}

\section{\texorpdfstring{Potential Applications of
\texttt{fertile}}{Potential Applications of fertile}}\label{potential-applications-of-fertile}

\subsection{In Journal Review}\label{in-journal-review}

\subsection{By Beginning Data
Scientists}\label{by-beginning-data-scientists}

\subsection{By Advanced Data
Scientists}\label{by-advanced-data-scientists}

\subsection{For Teaching
Reproducibility}\label{for-teaching-reproducibility}

Nicole Janz -- Brining the Gold Standard into the Classroom: Replication
in University Teaching

\section{\texorpdfstring{Integration Of \texttt{fertile} And Other
Reproducibility Tools in Data Science
Education}{Integration Of fertile And Other Reproducibility Tools in Data Science Education}}\label{integration-of-fertile-and-other-reproducibility-tools-in-data-science-education}

\chapter*{Conclusion}\label{conclusion}
\addcontentsline{toc}{chapter}{Conclusion}

\texttt{fertile} is an \texttt{R} package that lowers barriers to
reproducible data analysis projects in \texttt{R}, providing a wide
array of checks and suggestions addressing many different aspects of
project reproducibility, including file organization, file path usage,
documentation, and dependencies. \texttt{fertile} is meant to be
educational, providing informative error messages that indicate why
users' mistakes are problematic and sharing recommendations on how to
fix them. The package is designed in this way so as to promote a greater
understanding of reproducibility concepts in its users, with the goal of
increasing the overall awareness and understanding of reproducibility in
the \texttt{R} community.

The package has very low barriers to entry, making it accessible to
users with various levels of background knowledge. Unlike many other
\texttt{R} packages focused on reproducibility that are currently
available, the features of \texttt{fertile} can be accessed almost
effortlessly. Many of the retroactive features can be accessed in only
two lines of code requiring minimal arguments and some of the proactive
features can be accessed with no additional effort beyond loading the
package. This, in combination with the fact that \texttt{fertile} does
not focus on one specific area of reproducibility, instead covering
(albeit in less detail) a wide variety of topics, means that
\texttt{fertile} makes it easy for data analysts of all skill levels to
quickly gain a better understanding of the reproducibility of the work.

In the moment, it often feels easiest to take a shortcut---to use an
absolute path or change a working directory. However, when considering
the long term path of a project, spending the extra time to improve
reproducibility is worthwhile. \texttt{fertile}'s user-friendly features
can help data analysts avoid these harmful shortcuts with minimal
effort.

\appendix

\chapter{The First Appendix}\label{the-first-appendix}

This first appendix includes all of the R chunks of code that were
hidden throughout the document (using the \texttt{include\ =\ FALSE}
chunk tag) to help with readibility and/or setup.

\textbf{In the main Rmd file}
\begin{Shaded}
\begin{Highlighting}[]
\CommentTok{# This chunk ensures that the thesisdown package is}
\CommentTok{# installed and loaded. This thesisdown package includes}
\CommentTok{# the template files for the thesis.}
\ControlFlowTok{if}\NormalTok{ (}\OperatorTok{!}\KeywordTok{require}\NormalTok{(remotes)) \{}
  \ControlFlowTok{if}\NormalTok{ (params}\OperatorTok{$}\StringTok{`}\DataTypeTok{Install needed packages for \{thesisdown\}}\StringTok{`}\NormalTok{) \{}
    \KeywordTok{install.packages}\NormalTok{(}\StringTok{"remotes"}\NormalTok{, }\DataTypeTok{repos =} \StringTok{"https://cran.rstudio.com"}\NormalTok{)}
\NormalTok{  \} }\ControlFlowTok{else}\NormalTok{ \{}
    \KeywordTok{stop}\NormalTok{(}
      \KeywordTok{paste}\NormalTok{(}\StringTok{'You need to run install.packages("remotes")",}
\StringTok{            "first in the Console.'}\NormalTok{)}
\NormalTok{    )}
\NormalTok{  \}}
\NormalTok{\}}
\ControlFlowTok{if}\NormalTok{ (}\OperatorTok{!}\KeywordTok{require}\NormalTok{(thesisdown)) \{}
  \ControlFlowTok{if}\NormalTok{ (params}\OperatorTok{$}\StringTok{`}\DataTypeTok{Install needed packages for \{thesisdown\}}\StringTok{`}\NormalTok{) \{}
\NormalTok{    remotes}\OperatorTok{::}\KeywordTok{install_github}\NormalTok{(}\StringTok{"ismayc/thesisdown"}\NormalTok{)}
\NormalTok{  \} }\ControlFlowTok{else}\NormalTok{ \{}
    \KeywordTok{stop}\NormalTok{(}
      \KeywordTok{paste}\NormalTok{(}
        \StringTok{"You need to run"}\NormalTok{,}
        \StringTok{'remotes::install_github("ismayc/thesisdown")'}\NormalTok{,}
        \StringTok{"first in the Console."}
\NormalTok{      )}
\NormalTok{    )}
\NormalTok{  \}}
\NormalTok{\}}
\KeywordTok{library}\NormalTok{(thesisdown)}
\CommentTok{# Set how wide the R output will go}
\end{Highlighting}
\end{Shaded}
\chapter{The Second Appendix, for
Fun}\label{the-second-appendix-for-fun}

\backmatter

\chapter*{References}\label{references}
\addcontentsline{toc}{chapter}{References}

\markboth{References}{References}

\noindent

\setlength{\parindent}{-0.20in} \setlength{\leftskip}{0.20in}
\setlength{\parskip}{8pt}

\hypertarget{refs}{}
\hypertarget{ref-aee-policy}{}
American Economic Association. (2020). Data and code availability
policy. Retrieved from
\url{https://www.aeaweb.org/journals/data/data-code-policy}

\hypertarget{ref-ajps-guidelines}{}
American Journal of Political Science. (2016, May). Guidelines for
preparing replication files. Retrieved from
\url{https://ajps.org/wp-content/uploads/2018/05/ajps_replication-guidelines-2-1.pdf}

\hypertarget{ref-asa-guide}{}
American Statistical Association. (2020). JASA acs reproducibility
guide. Retrieved from
\url{https://jasa-acs.github.io/repro-guide/pages/author-guidelines}

\hypertarget{ref-nature-psych}{}
Baker, M. (2015). Over half of psychological studies fail
reproducibility test. \emph{Nature}. Retrieved from
\url{https://www.nature.com/news/over-half-of-psychology-studies-fail-reproducibility-test-1.18248}

\hypertarget{ref-nature-crisis}{}
Baker, M. (2016). 1,500 scientists lift the lid on reproducibility.
\emph{Nature}. Retrieved from
\url{https://www.nature.com/news/1-500-scientists-lift-the-lid-on-reproducibility-1.19970}

\hypertarget{ref-begley2012raise}{}
Begley, C. G., \& Ellis, L. M. (2012). Raise standards for preclinical
cancer research. \emph{Nature}, \emph{483}(7391), 531--533.

\hypertarget{ref-R-workflowr}{}
Blischak, J., Carbonetto, P., \& Stephens, M. (2019). Workflowr: A
framework for reproducible and collaborative data science. Retrieved
from \url{https://CRAN.R-project.org/package=workflowr}

\hypertarget{ref-arlington}{}
Bollen, K., Cacioppo, J. T., Kaplan, R. M., Krosnick, J. A., Olds, J.
L., \& Dean, H. (2015). Report of the subcommittee on replicability in
science advisory committee to the nsf sbe directorate.

\hypertarget{ref-broman}{}
Broman, K. (2019). Initial steps toward reproducible research: Organize
your data and code. \emph{Sitewide ATOM}. Retrieved from
\url{https://kbroman.org/steps2rr/pages/organize.html}

\hypertarget{ref-exp-results}{}
Cambridge University Press. (2020). Experimental results - transparency
and openness policy. Retrieved from
\url{https://www.cambridge.org/core/journals/experimental-results/information/transparency-and-openness-policy}

\hypertarget{ref-claerbout}{}
Claerbout, J. F., \& Karrenbach, M. (1992). Electronic documents give
reproducible research a new meaning. In \emph{SEG technical program
expanded abstracts 1992} (pp. 601--604). Society of Exploration
Geophysicists.

\hypertarget{ref-cooper2017guide}{}
Cooper, N., Hsing, P.-Y., Croucher, M., Graham, L., James, T.,
Krystalli, A., \& Michonneau, F. (2017). A guide to reproducible code in
ecology and evolution. \emph{British Ecological Society}. Retrieved from
\url{https://www.britishecologicalsociety.org/wp-content/uploads/2017/12/guide-to-reproducible-code.pdf}

\hypertarget{ref-eisner-reproducibility}{}
Eisner, D. A. (2018). Reproducibility of science: Fraud, impact factors
and carelessness. \emph{Journal of Molecular and Cellular Cardiology},
\emph{114}, 364--368.
\url{http://doi.org/https://doi.org/10.1016/j.yjmcc.2017.10.009}

\hypertarget{ref-sep-scientific-reproducibility}{}
Fidler, F., \& Wilcox, J. (2018). Reproducibility of scientific results.
In E. N. Zalta (Ed.), \emph{The stanford encyclopedia of philosophy}
(Winter 2018).
\url{https://plato.stanford.edu/archives/win2018/entries/scientific-reproducibility/};
Metaphysics Research Lab, Stanford University.

\hypertarget{ref-R-orderly}{}
FitzJohn, R., Ashton, R., Hill, A., Eden, M., Hinsley, W., Russell, E.,
\& Thompson, J. (2020). Orderly: Lightweight reproducible reporting.
Retrieved from \url{https://CRAN.R-project.org/package=orderly}

\hypertarget{ref-unix}{}
Gancarz, M. (2003). \emph{Linux and the unix philosophy} (2nd ed.).
Woburn, MA: Digital Press.

\hypertarget{ref-Goodman341ps12}{}
Goodman, S. N., Fanelli, D., \& Ioannidis, J. P. A. (2016). What does
research reproducibility mean? \emph{Science Translational Medicine},
\emph{8}(341), 1--6. \url{http://doi.org/10.1126/scitranslmed.aaf5027}

\hypertarget{ref-bioessays-gosselin}{}
Gosselin, R.-D. (2020). Statistical analysis must improve to address the
reproducibility crisis: The access to transparent statistics (acts) call
to action. \emph{BioEssays}, \emph{42}(1), 1900189.
\url{http://doi.org/10.1002/bies.201900189}

\hypertarget{ref-hardwicke2018data}{}
Hardwicke, T. E., Mathur, M. B., MacDonald, K., Nilsonne, G., Banks, G.
C., Kidwell, M. C., \ldots{} others. (2018). Data availability,
reusability, and analytic reproducibility: Evaluating the impact of a
mandatory open data policy at the journal cognition. \emph{Royal Society
Open Science}, \emph{5}(8), 180448. Retrieved from
\url{https://royalsocietypublishing.org/doi/full/10.1098/rsos.180448}

\hypertarget{ref-R-tidyselect}{}
Henry, L., \& Wickham, H. (2020). Tidyselect: Select from a set of
strings. Retrieved from
\url{https://CRAN.R-project.org/package=tidyselect}

\hypertarget{ref-hermans2017programming}{}
Hermans, F., \& Aldewereld, M. (2017). Programming is writing is
programming. In \emph{Companion to the first international conference on
the art, science and engineering of programming} (pp. 1--8).

\hypertarget{ref-hrynaszkiewicz2020publishers}{}
Hrynaszkiewicz, I. (2020). Publishers' responsibilities in promoting
data quality and reproducibility. \emph{Handbook of Experimental
Pharmacology}, \emph{257}, 319--348.
\url{http://doi.org/https://doi.org/10.1007/164_2019_290}

\hypertarget{ref-higher-ed}{}
Jacoby, W. G., Lafferty-Hess, S., \& Christian, T.-M. (2017). Should
journals be responsible for reproducibility? Inside Higher Ed. Retrieved
from
\url{https://www.insidehighered.com/blogs/rethinking-research/should-journals-be-responsible-reproducibility}

\hypertarget{ref-jcgs-guide}{}
Journal of Computational and Graphical Statistics. (2020). Instructions
for authors. Retrieved from
\url{https://www.tandfonline.com/action/authorSubmission?show=instructions\&journalCode=ucgs20}

\hypertarget{ref-jss-guide}{}
Journal of Statistical Software. (2020). Instructions for authors.
Retrieved from
\url{https://www.jstatsoft.org/pages/view/authors\#review-process.}

\hypertarget{ref-kitzes2017practice}{}
Kitzes, J., Turek, D., \& Deniz, F. (2017). \emph{The practice of
reproducible research: Case studies and lessons from the data-intensive
sciences}. Berkeley, CA: University of California Press. Retrieved from
\url{https://www.practicereproducibleresearch.org}

\hypertarget{ref-leopold2015increased}{}
Leopold, S. S. (2015). Editorial: Increased manuscript submissions
prompt journals to make hard choices. \emph{Clinical Orthopaedics and
Related Research}, \emph{473}(3), 753--755.
\url{http://doi.org/10.1007/s11999-014-4129-1}

\hypertarget{ref-r-opensci}{}
Martinez, C., Hollister, J., Marwick, B., Szöcs, E., Zeitlin, S.,
Kinoshita, B. P., \ldots{} Meinke, B. (2018). Reproducibility in
Science: A Guide to enhancing reproducibility in scientific results and
writing. Retrieved from
\url{http://ropensci.github.io/reproducibility-guide/}

\hypertarget{ref-R-rrtools}{}
Marwick, B. (2019). Rrtools: Creates a reproducible research compendium.
Retrieved from \url{https://github.com/benmarwick/rrtools}

\hypertarget{ref-marwick2018packaging}{}
Marwick, B., Boettiger, C., \& Mullen, L. (2018). Packaging data
analytical work reproducibly using R (and friends). \emph{The American
Statistician}, \emph{72}(1), 80--88.
\url{http://doi.org/doi.org/10.1080/00031305.2017.1375986}

\hypertarget{ref-engineering-reproducibility}{}
McArthur, S. L. (2019). Repeatability, reproducibility, and
replicability: Tackling the 3R challenge in biointerface science and
engineering. \emph{Biointerphases}, \emph{14}(2), 1--2.
\url{http://doi.org/10.1116/1.5093621}

\hypertarget{ref-R-reproducible}{}
McIntire, E. J. B., \& Chubaty, A. M. (2020). Reproducible: A set of
tools that enhance reproducibility beyond package management. Retrieved
from \url{https://CRAN.R-project.org/package=reproducible}

\hypertarget{ref-bio-principles}{}
National Institutes of Health. (2014, June). Principles and guidelines
for reporting preclinical research. Retrieved from
\url{https://www.nih.gov/research-training/rigor-reproducibility/principles-guidelines-reporting-preclinical-research}

\hypertarget{ref-R-drake}{}
OpenSci, R. (2020). Drake: A pipeline toolkit for reproducible
computation at scale. Retrieved from
\url{https://cran.r-project.org/package=drake}

\hypertarget{ref-wercker}{}
Oracle Corporation. (2019). Wercker. Retrieved from
\url{https://github.com/wercker/wercker}

\hypertarget{ref-r-journal}{}
R Journal Editors. (2020). Instructions for authors. Retrieved from
\url{https://journal.r-project.org/share/author-guide.pdf}

\hypertarget{ref-coreteam-extensions}{}
R-Core-Team. (2020). Writing r extensions. \emph{R Foundation for
Statistical Computing}. Retrieved from
\url{http://cran.stat.unipd.it/doc/manuals/r-release/R-exts.pdf}

\hypertarget{ref-R-checkers}{}
Ross, N., DeCicco, L., \& Randhawa, N. (2018). Checkers: Automated
checking of best practices for research compendia. Retrieved from
\url{https://github.com/ropenscilabs/checkers/blob/master/DESCRIPTIONr}

\hypertarget{ref-policy-effectiveness}{}
Stodden, V., Seiler, J., \& Ma, Z. (2018a). An empirical analysis of
journal policy effectiveness for computational reproducibility.
\emph{Proceedings of the National Academy of Sciences}, \emph{115}(11),
2584--2589. Retrieved from
\url{https://www.pnas.org/content/115/11/2584}

\hypertarget{ref-Stodden2584}{}
Stodden, V., Seiler, J., \& Ma, Z. (2018b). An empirical analysis of
journal policy effectiveness for computational reproducibility.
\emph{Proceedings of the National Academy of Sciences}, \emph{115}(11),
2584--2589. \url{http://doi.org/10.1073/pnas.1708290115}

\hypertarget{ref-ams-guide}{}
The American Statistician. (2020). Instructions for authors. Retrieved
from
\url{https://www.tandfonline.com/action/authorSubmission?show=instructions\&journalCode=utas20}

\hypertarget{ref-R-renv}{}
Ushey, K., \& RStudio. (2020). Renv: Project environments. Retrieved
from \url{https://cran.r-project.org/web/packages/renv/index.html}

\hypertarget{ref-plos-biology}{}
Wallach, J. D., Boyack, K. W., \& Ioannidis, J. P. A. (2018).
Reproducible research practices, transparency, and open access data in
the biomedical literature, 2015-2017. \emph{PLOS Biology},
\emph{16}(11), 1--20. \url{http://doi.org/10.1371/journal.pbio.2006930}

\hypertarget{ref-hadley-packages}{}
Wickham, H. (2015). \emph{R packages} (1st ed.). Sebastopol, CA:
O'Reilly Media, Inc.

\hypertarget{ref-top-guidelines}{}
Woolston, C. (2020). TOP factor rates journals on transparency,
openness. Nature Index. Retrieved from
\url{https://www.natureindex.com/news-blog/top-factor-rates-journals-on-transparency-openness}


% Index?

\end{document}
