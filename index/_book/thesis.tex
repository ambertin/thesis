% This is the Reed College LaTeX thesis template. Most of the work
% for the document class was done by Sam Noble (SN), as well as this
% template. Later comments etc. by Ben Salzberg (BTS). Additional
% restructuring and APA support by Jess Youngberg (JY).
% Your comments and suggestions are more than welcome; please email
% them to cus@reed.edu
%
% See https://www.reed.edu/cis/help/LaTeX/index.html for help. There are a
% great bunch of help pages there, with notes on
% getting started, bibtex, etc. Go there and read it if you're not
% already familiar with LaTeX.
%
% Any line that starts with a percent symbol is a comment.
% They won't show up in the document, and are useful for notes
% to yourself and explaining commands.
% Commenting also removes a line from the document;
% very handy for troubleshooting problems. -BTS

% As far as I know, this follows the requirements laid out in
% the 2002-2003 Senior Handbook. Ask a librarian to check the
% document before binding. -SN

%%
%% Preamble
%%
% \documentclass{<something>} must begin each LaTeX document
\documentclass[12pt,twoside]{reedthesis}
% Packages are extensions to the basic LaTeX functions. Whatever you
% want to typeset, there is probably a package out there for it.
% Chemistry (chemtex), screenplays, you name it.
% Check out CTAN to see: https://www.ctan.org/
%%
\usepackage{graphicx,latexsym}
\usepackage{amsmath}
\usepackage{amssymb,amsthm}
\usepackage{longtable,booktabs,setspace}
\usepackage{chemarr} %% Useful for one reaction arrow, useless if you're not a chem major
\usepackage[hyphens]{url}
% Added by CII
\usepackage{hyperref}
\usepackage{lmodern}
\usepackage{float}
\floatplacement{figure}{H}
% End of CII addition
\usepackage{rotating}

\setlength{\headheight}{18pt}%

% Next line commented out by CII
%%% \usepackage{natbib}
% Comment out the natbib line above and uncomment the following two lines to use the new
% biblatex-chicago style, for Chicago A. Also make some changes at the end where the
% bibliography is included.
%\usepackage{biblatex-chicago}
%\bibliography{thesis}


% Added by CII (Thanks, Hadley!)
% Use ref for internal links
\renewcommand{\hyperref}[2][???]{\autoref{#1}}
\def\chapterautorefname{Chapter}
\def\sectionautorefname{Section}
\def\subsectionautorefname{Subsection}
% End of CII addition

% Added by CII
\usepackage{caption}
\captionsetup{width=5in}
% End of CII addition

% \usepackage{times} % other fonts are available like times, bookman, charter, palatino

% Syntax highlighting #22
  \usepackage{color}
  \usepackage{fancyvrb}
  \newcommand{\VerbBar}{|}
  \newcommand{\VERB}{\Verb[commandchars=\\\{\}]}
  \DefineVerbatimEnvironment{Highlighting}{Verbatim}{commandchars=\\\{\}}
  % Add ',fontsize=\small' for more characters per line
  \usepackage{framed}
  \definecolor{shadecolor}{RGB}{248,248,248}
  \newenvironment{Shaded}{\begin{snugshade}}{\end{snugshade}}
  \newcommand{\KeywordTok}[1]{\textcolor[rgb]{0.13,0.29,0.53}{\textbf{#1}}}
  \newcommand{\DataTypeTok}[1]{\textcolor[rgb]{0.13,0.29,0.53}{#1}}
  \newcommand{\DecValTok}[1]{\textcolor[rgb]{0.00,0.00,0.81}{#1}}
  \newcommand{\BaseNTok}[1]{\textcolor[rgb]{0.00,0.00,0.81}{#1}}
  \newcommand{\FloatTok}[1]{\textcolor[rgb]{0.00,0.00,0.81}{#1}}
  \newcommand{\ConstantTok}[1]{\textcolor[rgb]{0.00,0.00,0.00}{#1}}
  \newcommand{\CharTok}[1]{\textcolor[rgb]{0.31,0.60,0.02}{#1}}
  \newcommand{\SpecialCharTok}[1]{\textcolor[rgb]{0.00,0.00,0.00}{#1}}
  \newcommand{\StringTok}[1]{\textcolor[rgb]{0.31,0.60,0.02}{#1}}
  \newcommand{\VerbatimStringTok}[1]{\textcolor[rgb]{0.31,0.60,0.02}{#1}}
  \newcommand{\SpecialStringTok}[1]{\textcolor[rgb]{0.31,0.60,0.02}{#1}}
  \newcommand{\ImportTok}[1]{#1}
  \newcommand{\CommentTok}[1]{\textcolor[rgb]{0.56,0.35,0.01}{\textit{#1}}}
  \newcommand{\DocumentationTok}[1]{\textcolor[rgb]{0.56,0.35,0.01}{\textbf{\textit{#1}}}}
  \newcommand{\AnnotationTok}[1]{\textcolor[rgb]{0.56,0.35,0.01}{\textbf{\textit{#1}}}}
  \newcommand{\CommentVarTok}[1]{\textcolor[rgb]{0.56,0.35,0.01}{\textbf{\textit{#1}}}}
  \newcommand{\OtherTok}[1]{\textcolor[rgb]{0.56,0.35,0.01}{#1}}
  \newcommand{\FunctionTok}[1]{\textcolor[rgb]{0.00,0.00,0.00}{#1}}
  \newcommand{\VariableTok}[1]{\textcolor[rgb]{0.00,0.00,0.00}{#1}}
  \newcommand{\ControlFlowTok}[1]{\textcolor[rgb]{0.13,0.29,0.53}{\textbf{#1}}}
  \newcommand{\OperatorTok}[1]{\textcolor[rgb]{0.81,0.36,0.00}{\textbf{#1}}}
  \newcommand{\BuiltInTok}[1]{#1}
  \newcommand{\ExtensionTok}[1]{#1}
  \newcommand{\PreprocessorTok}[1]{\textcolor[rgb]{0.56,0.35,0.01}{\textit{#1}}}
  \newcommand{\AttributeTok}[1]{\textcolor[rgb]{0.77,0.63,0.00}{#1}}
  \newcommand{\RegionMarkerTok}[1]{#1}
  \newcommand{\InformationTok}[1]{\textcolor[rgb]{0.56,0.35,0.01}{\textbf{\textit{#1}}}}
  \newcommand{\WarningTok}[1]{\textcolor[rgb]{0.56,0.35,0.01}{\textbf{\textit{#1}}}}
  \newcommand{\AlertTok}[1]{\textcolor[rgb]{0.94,0.16,0.16}{#1}}
  \newcommand{\ErrorTok}[1]{\textcolor[rgb]{0.64,0.00,0.00}{\textbf{#1}}}
  \newcommand{\NormalTok}[1]{#1}

% To pass between YAML and LaTeX the dollar signs are added by CII
\title{Addressing The Scientific Reproducibility Crisis Through Educational
Software Integration}
\author{Audrey M. Bertin}
% The month and year that you submit your FINAL draft TO THE LIBRARY (May or December)
\date{May 2021}
\division{Statistical and Data Sciences}
\advisor{Benjamin S. Baumer}
\institution{Smith College}
\degree{Bachelor of Arts}
%If you have two advisors for some reason, you can use the following
% Uncommented out by CII
% End of CII addition

%%% Remember to use the correct department!
\department{Statistical and Data Sciences}
% if you're writing a thesis in an interdisciplinary major,
% uncomment the line below and change the text as appropriate.
% check the Senior Handbook if unsure.
%\thedivisionof{The Established Interdisciplinary Committee for}
% if you want the approval page to say "Approved for the Committee",
% uncomment the next line
%\approvedforthe{Committee}

% Added by CII
%%% Copied from knitr
%% maxwidth is the original width if it's less than linewidth
%% otherwise use linewidth (to make sure the graphics do not exceed the margin)
\makeatletter
\def\maxwidth{ %
  \ifdim\Gin@nat@width>\linewidth
    \linewidth
  \else
    \Gin@nat@width
  \fi
}
\makeatother

%Added by @MyKo101, code provided by @GerbrichFerdinands

\renewcommand{\contentsname}{Table of Contents}
% End of CII addition

\setlength{\parskip}{0pt}

% Added by CII

\providecommand{\tightlist}{%
  \setlength{\itemsep}{0pt}\setlength{\parskip}{0pt}}

\Acknowledgements{
I want to thank a few people.
}

\Dedication{
You can have a dedication here if you wish.
}

\Preface{
This is an example of a thesis setup to use the reed thesis document
class (for LaTeX) and the R bookdown package, in general.
}

\Abstract{
The preface pretty much says it all. \par
Second paragraph of abstract starts here.
}

% End of CII addition
%%
%% End Preamble
%%
%
\begin{document}

% Everything below added by CII
  \maketitle

\frontmatter % this stuff will be roman-numbered
\pagestyle{empty} % this removes page numbers from the frontmatter
  \begin{acknowledgements}
    I want to thank a few people.
  \end{acknowledgements}
  \begin{preface}
    This is an example of a thesis setup to use the reed thesis document
    class (for LaTeX) and the R bookdown package, in general.
  \end{preface}
  \hypersetup{linkcolor=black}
  \setcounter{tocdepth}{2}
  \tableofcontents


  \begin{abstract}
    The preface pretty much says it all. \par
    Second paragraph of abstract starts here.
  \end{abstract}
  \begin{dedication}
    You can have a dedication here if you wish.
  \end{dedication}
\mainmatter % here the regular arabic numbering starts
\pagestyle{fancyplain} % turns page numbering back on

\chapter{This will automatically install the \{remotes\} package and
\{thesisdown\}}\label{this-will-automatically-install-the-remotes-package-and-thesisdown}

Placeholder

\chapter{An Introduction to Reproducibility}\label{reproducibility}

Placeholder

\section{What Is Reproducibility?}\label{what-is-reproducibility}

\section{The Reproducibility Crisis}\label{the-reproducibility-crisis}

\section{The Components of Reproducible
Research}\label{the-components-of-reproducible-research}

\section{Current Attempts to Address Reproducibility in Scientific
Publishing}\label{current-attempts-to-address-reproducibility-in-scientific-publishing}

\subsection{Case Studies Across The
Sciences}\label{case-studies-across-the-sciences}

\subsection{Case Studies In The Statistical And Data
Sciences}\label{case-studies-in-the-statistical-and-data-sciences}

\subsection{The Bigger Picture}\label{the-bigger-picture}

\subsection{Assessing the Success of Academic Reproducibility
Policies}\label{assessing-the-success-of-academic-reproducibility-policies}

\section{Limitations on Achieving Reproducibility in Scientific
Publishing}\label{limitations-on-achieving-reproducibility-in-scientific-publishing}

\subsection{Challenges for Authors}\label{challenges-for-authors}

\subsection{Challenges for Journals}\label{challenges-for-journals}

\section{Attempts to Address These
Limitations}\label{attempts-to-address-these-limitations}

\subsection{Through Education}\label{through-education}

\subsection{Through Software}\label{through-software}

\chapter{\texorpdfstring{\texttt{fertile}: My Contribution To Addressing
Reproducibility}{fertile: My Contribution To Addressing Reproducibility}}\label{my-solution}

\section{Understanding The Gaps In Existing Reproducibility
Solutions}\label{understanding-the-gaps-in-existing-reproducibility-solutions}

Although the current state of reproducibility in academia is quite poor,
it is not an impossible challenge to overcome. The relative simplicity
of addressing reproducibility, particularly when compared with
replicability, makes it an ideal candidate for solution-building.
Although significant progress on addressing reproducibility on a
widespread scale is a long-term challenge, impactful forward
progress--if on a smaller scale--can be achieved in the short-term.

As we have seen, software developers, data scientists, and educators
around the world have realized this potential, taking steps to help
address the current crisis of reproducibility. Journals have put in
place guidelines for authors, statisticians have developed \texttt{R}
packages that help structure projects in a reproducible format, and
educators have begun integrate reproducibility exercises into their
courses.

However, many of these attempts to address reproducibility have
significant drawbacks associated with them. We have already explored the
issues with journal policies, both for authors and reviewers, in-depth.
In this section, we will consider the education and software solutions
and their associated challenges.

\subsection{In Education}\label{in-education}

\subsection{In Software}\label{in-software}

Many of these packages are narrow, with each effectively addressing a
small component of reproducibility: file structure, modularization of
code, version control, etc. These packages often succeed in their area
of focus, but at the cost of accessibility to a wider audience. Their
functions are often quite complex to use, and many steps must be
completed to achieve the required reproducibility goal. This cumbersome
nature means that most reproducibility packages currently available are
not easily accessible to users near the beginning of their \texttt{R}
journey, nor particularly useful to those looking for quick and easy
reproducibility checks.

A more effective way of realizing widespread reproducibility is to make
the process for doing so simple enough that it takes little to no
conscious effort to implement. You want users to ``fall into a hole''
(we paraphrase Hadley Wickham) of good practice.

However, while these tools can be useful, they are generalized so as to
be useful to the widest audience. As a result, their checks are not
designed to be \texttt{R}-specific, which makes them sub-optimal for
users looking to address reproducibility issues involving features
specific to the \texttt{R} programming language, such as package
installation and seed setting.

\section{\texorpdfstring{\texttt{fertile}, An R Package Creating Optimal
Conditions For
Reproducibility}{fertile, An R Package Creating Optimal Conditions For Reproducibility}}\label{fertile-an-r-package-creating-optimal-conditions-for-reproducibility}

\subsection{Package Overview}\label{package-overview}

\texttt{fertile} attempts to address these gaps in existing software by
providing a simple, easy-to-learn reproducibility package that, rather
than focusing intensely on a specific area, provides some information
about a wide variety of aspects influencing reproducibility.
\texttt{fertile} is flexible, offering benefits to users at any stage in
the data analysis workflow, and provides \texttt{R}-specific features,
which address certain aspects of reproducibility that can be missed by
external project development software.

\texttt{fertile} is designed to be used on data analyses organized as
\texttt{R} Projects (i.e.~directories containing an \texttt{.Rproj}
file). Once an \texttt{R} Project is created, \texttt{fertile} provides
benefits throughout the data analysis process, both during development
as well as after the fact. \texttt{fertile} achieves this by operating
in two modes: proactively (to prevent reproducibility mistakes from
happening in the first place), and retroactively (analyzing code that
has already been written for potential problems).

Much of the available literature focuses on file structure,
organization, and naming, and \texttt{fertile}'s features are consistent
with this. Marwick, Boettiger, \& Mullen (2018) provide the framework
for file structure that \texttt{fertile} is based on: a structure
similar to that of an \texttt{R} package (R-Core-Team (2020), Wickham
(2015)), with an \texttt{R} folder, as well as \texttt{data},
\texttt{data-raw}, \texttt{inst}, and \texttt{vignettes}.

\subsection{Proactive Use}\label{proactive-use}

Proactively, the package identifies potential mistakes as they are made
by the user and outputs an informative message as well as a recommended
solution. For example, \texttt{fertile} catches when a user passes a
potentially problematic file path---such as an absolute path, or a path
that points to a location outside of the project directory---to a
variety of common input/output functions operating on many different
file types.
\begin{Shaded}
\begin{Highlighting}[]
\KeywordTok{library}\NormalTok{(fertile)}
\KeywordTok{file.exists}\NormalTok{(}\StringTok{"~/Desktop/my_data.csv"}\NormalTok{)}
\end{Highlighting}
\end{Shaded}
\begin{verbatim}
[1] TRUE
\end{verbatim}
\begin{Shaded}
\begin{Highlighting}[]
\KeywordTok{read.csv}\NormalTok{(}\StringTok{"~/Desktop/my_data.csv"}\NormalTok{)}
\end{Highlighting}
\end{Shaded}
\begin{verbatim}
Error: Detected absolute paths
\end{verbatim}
\begin{Shaded}
\begin{Highlighting}[]
\KeywordTok{read.csv}\NormalTok{(}\StringTok{"../../../Desktop/my_data.csv"}\NormalTok{)}
\end{Highlighting}
\end{Shaded}
\begin{verbatim}
Error: Detected paths that lead outside the project directory
\end{verbatim}
\texttt{fertile} is even more aggressive with functions (like
\texttt{setwd()}) that are almost certain to break reproducibility,
causing them to throw errors that prevent their execution and providing
recommendations for better alternatives.
\begin{Shaded}
\begin{Highlighting}[]
\KeywordTok{setwd}\NormalTok{(}\StringTok{"~/Desktop"}\NormalTok{)}
\end{Highlighting}
\end{Shaded}
\begin{verbatim}
Error: setwd() is likely to break reproducibility. Use here::here() instead.
\end{verbatim}
These proactive warning features are activated immediately after
attaching the \texttt{fertile} package and require no additional effort
by the user.

\subsection{Retroactive Use}\label{retroactive-use}

Retroactively, \texttt{fertile} analyzes potential obstacles to
reproducibility in an RStudio Project (i.e., a directory that contains
an \texttt{.Rproj} file). The package considers several different
aspects of the project which may influence reproducibility, including
the directory structure, file paths, and whether randomness is used
thoughtfully.

The end products of these analyses are reproducibility reports
summarizing a project's adherence to reproducibility standards and
recommending remedies for where the project falls short. For example,
\texttt{fertile} might identify the use of randomness in code and
recommend setting a seed if one is not present.

Users can access the majority of \texttt{fertile}'s retroactive features
through two primary functions, \texttt{proj\_check()} and
\texttt{proj\_analyze()}.

The \texttt{proj\_check()} function runs fifteen different
reproducibility tests, noting which ones passed, which ones failed, the
reason for failure, a recommended solution, and a guide to where to look
for help. These tests include: looking for a clear build chain, checking
to make sure the root level of the project is clear of clutter,
confirming that there are no files present that are not being directly
used by or created by the code, and looking for uses of randomness that
do not have a call to \texttt{set.seed()} present. A full list is
provided below:
\begin{Shaded}
\begin{Highlighting}[]
\KeywordTok{list_checks}\NormalTok{()}
\end{Highlighting}
\end{Shaded}
\begin{verbatim}
-- The available checks in `fertile` are as follows: ------------- fertile 0.0.0.9027 --
\end{verbatim}
\begin{verbatim}
 [1] "has_tidy_media"          "has_tidy_images"        
 [3] "has_tidy_code"           "has_tidy_raw_data"      
 [5] "has_tidy_data"           "has_tidy_scripts"       
 [7] "has_readme"              "has_no_lint"            
 [9] "has_proj_root"           "has_no_nested_proj_root"
[11] "has_only_used_files"     "has_clear_build_chain"  
[13] "has_no_absolute_paths"   "has_only_portable_paths"
[15] "has_no_randomness"      
\end{verbatim}
Subsets of the fifteen tests can be invoked using the
\texttt{tidyselect} helper functions (Henry \& Wickham (2020)) in
combination with the more limited \texttt{proj\_check\_some()} function.
\begin{Shaded}
\begin{Highlighting}[]
\NormalTok{proj_dir <-}\StringTok{ "project_miceps"}
\end{Highlighting}
\end{Shaded}
\begin{Shaded}
\begin{Highlighting}[]
\KeywordTok{proj_check_some}\NormalTok{(proj_dir, }\KeywordTok{contains}\NormalTok{(}\StringTok{"paths"}\NormalTok{))}
\end{Highlighting}
\end{Shaded}
\begin{verbatim}
-- Compiling... -------------------------------------------------- fertile 0.0.0.9027 --
\end{verbatim}
\begin{verbatim}
-- Rendering R scripts... ---------------------------------------- fertile 0.0.0.9027 --
\end{verbatim}
\begin{verbatim}
-- Running reproducibility checks -------------------------------- fertile 0.0.0.9027 --
\end{verbatim}
\begin{verbatim}
v Checking for no absolute paths
\end{verbatim}
\begin{verbatim}
v Checking for only portable paths
\end{verbatim}
\begin{verbatim}
-- Summary of fertile checks ------------------------------------- fertile 0.0.0.9027 --
\end{verbatim}
\begin{verbatim}
v Reproducibility checks passed: 2
\end{verbatim}
Each test can also be run individually by calling the function matching
its check name.

The \texttt{proj\_analyze()} function creates a report documenting the
structure of a data analysis project. This report contains information
about all packages referenced in code, the files present in the
directory and their types, suggestions for moving files to create a more
organized structure, and a list of reproducibility-breaking file paths
used in code.
\begin{Shaded}
\begin{Highlighting}[]
\KeywordTok{proj_analyze}\NormalTok{(proj_dir)}
\end{Highlighting}
\end{Shaded}
\begin{verbatim}
-- Analysis of reproducibility for project_miceps ---------------- fertile 0.0.0.9027 --
\end{verbatim}
\begin{verbatim}
--   Packages referenced in source code -------------------------- fertile 0.0.0.9027 --
\end{verbatim}
\begin{verbatim}
# A tibble: 9 x 3
  package       N used_in                    
  <chr>     <int> <chr>                      
1 broom         1 project_miceps/analysis.Rmd
2 dplyr         1 project_miceps/analysis.Rmd
3 ggplot2       1 project_miceps/analysis.Rmd
4 purrr         1 project_miceps/analysis.Rmd
5 readr         1 project_miceps/analysis.Rmd
6 rmarkdown     1 project_miceps/analysis.Rmd
7 skimr         1 project_miceps/analysis.Rmd
8 stargazer     1 project_miceps/analysis.Rmd
9 tidyr         1 project_miceps/analysis.Rmd
\end{verbatim}
\begin{verbatim}
--   Files present in directory ---------------------------------- fertile 0.0.0.9027 --
\end{verbatim}
\begin{verbatim}
# A tibble: 9 x 4
  file               ext        size mime                                       
  <fs::path>         <chr> <fs::byt> <chr>                                      
1 Estrogen_Receptor~ docx     10.97K application/vnd.openxmlformats-officedocum~
2 citrate_v_time.png png     187.46K image/png                                  
3 proteins_v_time.p~ png     378.17K image/png                                  
4 Blot_data_updated~ csv      14.43K text/csv                                   
5 CS_data_redone.csv csv       7.39K text/csv                                   
6 mice.csv           csv      14.33K text/csv                                   
7 README.md          md           39 text/markdown                              
8 miceps.Rproj       Rproj       204 text/rstudio                               
9 analysis.Rmd       Rmd       4.94K text/x-markdown                            
\end{verbatim}
\begin{verbatim}
--   Suggestions for moving files -------------------------------- fertile 0.0.0.9027 --
\end{verbatim}
\begin{verbatim}
# A tibble: 7 x 3
  path_rel           dir_rel    cmd                                             
  <fs::path>         <fs::path> <chr>                                           
1 Blot_data_updated~ data-raw   file_move('project_miceps/Blot_data_updated.csv~
2 CS_data_redone.csv data-raw   file_move('project_miceps/CS_data_redone.csv', ~
3 Estrogen_Receptor~ inst/other file_move('project_miceps/Estrogen_Receptors.do~
4 analysis.Rmd       vignettes  file_move('project_miceps/analysis.Rmd', fs::di~
5 citrate_v_time.png inst/image file_move('project_miceps/citrate_v_time.png', ~
6 mice.csv           data-raw   file_move('project_miceps/mice.csv', fs::dir_cr~
7 proteins_v_time.p~ inst/image file_move('project_miceps/proteins_v_time.png',~
\end{verbatim}
\begin{verbatim}
--   Problematic paths logged ------------------------------------ fertile 0.0.0.9027 --
\end{verbatim}
\begin{verbatim}
NULL
\end{verbatim}
\subsection{Logging}\label{logging}

\texttt{fertile} also contains logging functionality, which records
commands run in the console that have the potential to affect
reproducibility, enabling users to look at their past history at any
time. The package focuses mostly on package loading and file opening,
noting which function was used, the path or package it referenced, and
the timestamp at which that event happened. Users can access the log
recording their commands at any time via the \texttt{log\_report()}
function:
\begin{Shaded}
\begin{Highlighting}[]
\KeywordTok{log_report}\NormalTok{()}
\end{Highlighting}
\end{Shaded}
\begin{verbatim}
# A tibble: 6 x 4
  path          path_abs                           func      timestamp          
  <chr>         <chr>                              <chr>     <dttm>             
1 package:remo~ <NA>                               base::re~ 2020-09-14 15:26:52
2 package:thes~ <NA>                               base::re~ 2020-09-14 15:26:52
3 package:thes~ <NA>                               base::li~ 2020-09-14 15:26:52
4 package:purrr <NA>                               base::li~ 2020-09-14 15:27:13
5 package:forc~ <NA>                               base::li~ 2020-09-14 15:27:13
6 project_mice~ /Users/audreybertin/Documents/the~ readr::r~ 2020-09-14 15:27:13
\end{verbatim}
The log, if not managed, can grow very long over time. For users who do
not desire such functionality, \texttt{log\_clear()} provides a way to
erase the log and start over.

\subsection{Utility Functions}\label{utility-functions}

\texttt{fertile} also provides several useful utility functions that may
assist with the process of data analysis.

\subsection{File Path Management}\label{file-path-management}

The \texttt{check\_path()} function analyzes a vector of paths (or a
single path) to determine whether there are any absolute paths or paths
that lead outside the project directory.
\begin{Shaded}
\begin{Highlighting}[]
\CommentTok{# Path inside the directory}
\KeywordTok{check_path}\NormalTok{(}\StringTok{"project_miceps"}\NormalTok{)}
\end{Highlighting}
\end{Shaded}
\begin{verbatim}
# A tibble: 0 x 3
# ... with 3 variables: path <chr>, problem <chr>, solution <chr>
\end{verbatim}
\begin{Shaded}
\begin{Highlighting}[]
\CommentTok{# Absolute path (current working directory)}
\KeywordTok{check_path}\NormalTok{(}\KeywordTok{getwd}\NormalTok{())}
\end{Highlighting}
\end{Shaded}
\begin{verbatim}
Error: Detected absolute paths
\end{verbatim}
\begin{Shaded}
\begin{Highlighting}[]
\CommentTok{# Path outside the directory}
\KeywordTok{check_path}\NormalTok{(}\StringTok{"../fertile.Rmd"}\NormalTok{)}
\end{Highlighting}
\end{Shaded}
\begin{verbatim}
Error: Detected paths that lead outside the project directory
\end{verbatim}
\subsection{File Types}\label{file-types}

There are several functions that can be used to check the type of a
file:
\begin{Shaded}
\begin{Highlighting}[]
\KeywordTok{is_data_file}\NormalTok{(fs}\OperatorTok{::}\KeywordTok{path}\NormalTok{(proj_dir, }\StringTok{"mice.csv"}\NormalTok{))}
\end{Highlighting}
\end{Shaded}
\begin{verbatim}
[1] TRUE
\end{verbatim}
\begin{Shaded}
\begin{Highlighting}[]
\KeywordTok{is_image_file}\NormalTok{(fs}\OperatorTok{::}\KeywordTok{path}\NormalTok{(proj_dir, }\StringTok{"proteins_v_time.png"}\NormalTok{))}
\end{Highlighting}
\end{Shaded}
\begin{verbatim}
[1] TRUE
\end{verbatim}
\begin{Shaded}
\begin{Highlighting}[]
\KeywordTok{is_text_file}\NormalTok{(fs}\OperatorTok{::}\KeywordTok{path}\NormalTok{(proj_dir, }\StringTok{"README.md"}\NormalTok{))}
\end{Highlighting}
\end{Shaded}
\begin{verbatim}
[1] TRUE
\end{verbatim}
\begin{Shaded}
\begin{Highlighting}[]
\KeywordTok{is_r_file}\NormalTok{(fs}\OperatorTok{::}\KeywordTok{path}\NormalTok{(proj_dir, }\StringTok{"analysis.Rmd"}\NormalTok{))}
\end{Highlighting}
\end{Shaded}
\begin{verbatim}
[1] TRUE
\end{verbatim}
\subsection{Temporary Directories}\label{temporary-directories}

The \texttt{sandbox()} function allows the user to make a copy of their
project in a temporary directory. This can be useful for ensuring that
projects run properly when access to the local file system is removed.
\begin{Shaded}
\begin{Highlighting}[]
\NormalTok{proj_dir}
\end{Highlighting}
\end{Shaded}
\begin{verbatim}
[1] "project_miceps"
\end{verbatim}
\begin{Shaded}
\begin{Highlighting}[]
\NormalTok{fs}\OperatorTok{::}\KeywordTok{dir_ls}\NormalTok{(proj_dir) }\OperatorTok\StringTok{ }\KeywordTok{head}\NormalTok{(}\DecValTok{3}\NormalTok{)}
\end{Highlighting}
\end{Shaded}
\begin{verbatim}
project_miceps/Blot_data_updated.csv   project_miceps/CS_data_redone.csv      
project_miceps/Estrogen_Receptors.docx 
\end{verbatim}
\begin{Shaded}
\begin{Highlighting}[]
\NormalTok{temp_dir <-}\StringTok{ }\KeywordTok{sandbox}\NormalTok{(proj_dir)}
\NormalTok{temp_dir}
\end{Highlighting}
\end{Shaded}
\begin{verbatim}
/var/folders/v6/f62qz88s0sd5n3yqw9d8sb300000gn/T/RtmpEMcSA2/project_miceps
\end{verbatim}
\begin{Shaded}
\begin{Highlighting}[]
\NormalTok{fs}\OperatorTok{::}\KeywordTok{dir_ls}\NormalTok{(temp_dir) }\OperatorTok\StringTok{ }\KeywordTok{head}\NormalTok{(}\DecValTok{3}\NormalTok{)}
\end{Highlighting}
\end{Shaded}
\begin{verbatim}
/var/folders/v6/f62qz88s0sd5n3yqw9d8sb300000gn/T/RtmpEMcSA2/project_miceps/Blot_data_updated.csv
/var/folders/v6/f62qz88s0sd5n3yqw9d8sb300000gn/T/RtmpEMcSA2/project_miceps/CS_data_redone.csv
/var/folders/v6/f62qz88s0sd5n3yqw9d8sb300000gn/T/RtmpEMcSA2/project_miceps/Estrogen_Receptors.docx
\end{verbatim}
\subsection{Managing Project
Dependencies}\label{managing-project-dependencies}

One of the challenges with ensuring that work is reproducible is the
issue of dependencies. Many data analysis projects reference a variety
of \texttt{R} packages in their code. When such projects are shared with
other users who may not have the required packages downloaded, it can
cause errors that prevent the project from running properly.

The \texttt{proj\_pkg\_script()}\} function assists with this issue by
making it simple and fast to download dependencies. When run on an
\texttt{R} project directory, the function creates a \texttt{.R} script
file that contains the code needed to install all of the packages
referenced in the project, differentiating between packages located on
CRAN and those located on GitHub.
\begin{Shaded}
\begin{Highlighting}[]
\NormalTok{install_script <-}\StringTok{ }\KeywordTok{proj_pkg_script}\NormalTok{(proj_dir)}
\KeywordTok{cat}\NormalTok{(}\KeywordTok{readChar}\NormalTok{(install_script, }\FloatTok{1e5}\NormalTok{))}
\end{Highlighting}
\end{Shaded}
\begin{verbatim}
# Run this script to install the required packages for this
R project.
# Packages hosted on CRAN...
install.packages(c( 'broom', 'dplyr', 'ggplot2', 'purrr',
'readr', 'rmarkdown', 'skimr', 'stargazer', 'tidyr' ))
# Packages hosted on GitHub...
\end{verbatim}
\section{\texorpdfstring{How \texttt{fertile}
Works}{How fertile Works}}\label{how-fertile-works}

Much of the functionality in \texttt{fertile} is achieved by writing
\texttt{shims} \textbf{link to wikipedia page here}. \texttt{fertile}'s
shimmed functions intercept the user's commands and perform various
logging and checking tasks before executing the desired function. Our
process is:
\begin{enumerate}
\def\labelenumi{\arabic{enumi}.}
\item
  Identify an \texttt{R} function that is likely to be involved in
  operations that may break reproducibility. Popular functions
  associated with only one package (e.g., \texttt{read\_csv()} from
  \texttt{readr}) are ideal candidates.
\item
  Create a function in \texttt{fertile} with the same name that takes
  the same arguments (and always the dots \texttt{...}).
\item
  Write this new function so that it:
\end{enumerate}
\begin{enumerate}
\def\labelenumi{\alph{enumi})}
\tightlist
\item
  captures any arguments,
\item
  logs the name of the function called,
\item
  performs any checks on these arguments, and
\item
  calls the original function with the original arguments. Except where
  warranted, the execution looks the same to the user as if they were
  calling the original function.
\end{enumerate}
Most shims are quite simple and look something like what is shown below
for \texttt{read\_csv()}.
\begin{Shaded}
\begin{Highlighting}[]
\NormalTok{fertile}\OperatorTok{::}\NormalTok{read_csv}
\end{Highlighting}
\end{Shaded}
\begin{verbatim}
function(file, ...) {
  if (interactive_log_on()) {
    log_push(file, "readr::read_csv")
    check_path_safe(file)
    readr::read_csv(file, ...)
  }
}
<bytecode: 0x7f9e23fafa70>
<environment: namespace:fertile>
\end{verbatim}
\texttt{fertile} shims many common functions, including those that read
in a variety of data types, write data, and load packages. This works
both proactively and retroactively, as the shimmed functions written in
\texttt{fertile} are activated both when the user is coding
interactively and when a file containing code is rendered.

In order to ensure that the \texttt{fertile} versions of functions
(``shims'') always supersede (``mask'') their original namesakes when
called, \texttt{fertile} uses its own shims of the \texttt{library} and
\texttt{require} functions to manipulate the \texttt{R} \texttt{search}
path so that it is always located in the first position. In the
\texttt{fertile} version of \texttt{library()}, we detach
\texttt{fertile} from the search path, load the requested package, and
then re-attach \texttt{fertile}. This ensures that when a user executes
a command, \texttt{R} will check \texttt{fertile} for a matching
function before considering other packages. While it is possible that
this shifty behavior could lead to unintended consequences, our goal is
to catch a good deal of problems before they become problematic. Users
can easily disable \texttt{fertile} by detaching it, or not loading it
in the first place.

\section{\texorpdfstring{\texttt{fertile} in Practice: Experimental
Results From Smith College Student
Use}{fertile in Practice: Experimental Results From Smith College Student Use}}\label{fertile-in-practice-experimental-results-from-smith-college-student-use}

\texttt{fertile} is designed to: 1) be simple enough that users with
minimal \texttt{R} experience can use the package without issue, 2)
increase the reproducibility of work produced by its users, and 3)
educate its users on why their work is or is not reproducible and
provide guidance on how to address any problems.

To test \texttt{fertile}'s effectiveness, we began an initial randomized
control trial of the package on an introductory undergraduate data
science course at Smith College in Spring 2020 \textbf{ADD FOOTNOTE}
(This study was approved by Smith College IRB, Protocol \#19-032).

The experiment was structured as follows:

1.Students are given a form at the start of the semester asking whether
they consent to participate in a study on data science education. In
order to successfully consent, they must provide their system username,
collected through the command \texttt{Sys.getenv("LOGNAME")}. To
maintain privacy the results are then transformed into a hexadecimal
string via the \texttt{md5()} hashing function.
\begin{enumerate}
\def\labelenumi{\arabic{enumi}.}
\setcounter{enumi}{1}
\item
  These hexadecimal strings are then randomly assigned into equally
  sized groups, one experimental group that receives the features of
  \texttt{fertile} and one group that receives a control.
\item
  The students are then asked to download a package called
  \texttt{sds192} (the course number and prefix), which was created for
  the purpose of this trial. It leverages an \texttt{.onAttach()}
  function to scan the \texttt{R} environment and collect the username
  of the user who is loading the package and run it through the same
  hashing algorithm as used previously. It then identifies whether that
  user belongs to the experimental or the control group. Depending on
  the group they are in, they receive a different version of the
  package.
\item
  The experimental group receives the basic \texttt{sds192} package,
  which consists of some data sets and \texttt{R} Markdown templates
  necessary for completing homework assignments and projects in the
  class, but also has \texttt{fertile} installed and loaded silently in
  the background. The package's proactive features are enabled, and
  therefore users will receive warning messages when they use absolute
  or non-portable paths or attempt to change their working directory.
  The control group receives only the basic \texttt{sds192} package,
  including its data sets and \texttt{R} Markdown templates. All
  students from both groups then use their version of the package
  throughout the semester on a variety of projects.
\item
  Both groups are given a short quiz on different components of
  reproducibility that are intended to be taught by \texttt{fertile} at
  both the beginning and end of the semester. Their scores are then
  compared to see whether one group learned more than the other group or
  whether their scores were essentially equivalent. Additionally, for
  every homework assignment submitted, the professor takes note of
  whether or not the project compiles successfully.
\end{enumerate}
Based on the results, we hope to determine whether \texttt{fertile} was
successful at achieving its intended goals. A lack of notable difference
between the \emph{experimental} and \emph{control} groups in terms of
the number of code-related questions asked throughout the semester would
indicate that \texttt{fertile} achieved its goal of simplicity. A higher
average for the \emph{experimental} group in terms of the number of
homework assignments that compiled successfully would indicate that
\texttt{fertile} was successful in increasing reproducibility. A greater
increase over the semester in the reproducibility quiz scores for
students in the \emph{experimental} group compared with the
\emph{control} group would indicate that \texttt{fertile} achieved its
goal of educating users on reproducibility. Success according to these
metrics would provide evidence showing \texttt{fertile}'s benefit as
tool to help educators introduce reproducibility concepts in the
classroom.

\chapter{Incorporating Reproducibility Tools Into The Greater Data
Science Community}\label{applications}

\section{\texorpdfstring{Potential Applications of
\texttt{fertile}}{Potential Applications of fertile}}\label{potential-applications-of-fertile}

\subsection{In Journal Review}\label{in-journal-review}

\subsection{By Beginning Data
Scientists}\label{by-beginning-data-scientists}

\subsection{By Advanced Data
Scientists}\label{by-advanced-data-scientists}

\subsection{For Teaching
Reproducibility}\label{for-teaching-reproducibility}

Nicole Janz -- Brining the Gold Standard into the Classroom: Replication
in University Teaching

\section{\texorpdfstring{Integration Of \texttt{fertile} And Other
Reproducibility Tools in Data Science
Education}{Integration Of fertile And Other Reproducibility Tools in Data Science Education}}\label{integration-of-fertile-and-other-reproducibility-tools-in-data-science-education}

\chapter*{Conclusion}\label{conclusion}
\addcontentsline{toc}{chapter}{Conclusion}

\texttt{fertile} is an \texttt{R} package that lowers barriers to
reproducible data analysis projects in \texttt{R}, providing a wide
array of checks and suggestions addressing many different aspects of
project reproducibility, including file organization, file path usage,
documentation, and dependencies. \texttt{fertile} is meant to be
educational, providing informative error messages that indicate why
users' mistakes are problematic and sharing recommendations on how to
fix them. The package is designed in this way so as to promote a greater
understanding of reproducibility concepts in its users, with the goal of
increasing the overall awareness and understanding of reproducibility in
the \texttt{R} community.

The package has very low barriers to entry, making it accessible to
users with various levels of background knowledge. Unlike many other
\texttt{R} packages focused on reproducibility that are currently
available, the features of \texttt{fertile} can be accessed almost
effortlessly. Many of the retroactive features can be accessed in only
two lines of code requiring minimal arguments and some of the proactive
features can be accessed with no additional effort beyond loading the
package. This, in combination with the fact that \texttt{fertile} does
not focus on one specific area of reproducibility, instead covering
(albeit in less detail) a wide variety of topics, means that
\texttt{fertile} makes it easy for data analysts of all skill levels to
quickly gain a better understanding of the reproducibility of the work.

In the moment, it often feels easiest to take a shortcut---to use an
absolute path or change a working directory. However, when considering
the long term path of a project, spending the extra time to improve
reproducibility is worthwhile. \texttt{fertile}'s user-friendly features
can help data analysts avoid these harmful shortcuts with minimal
effort.

\appendix

\chapter{The First Appendix}\label{the-first-appendix}

This first appendix includes all of the R chunks of code that were
hidden throughout the document (using the \texttt{include\ =\ FALSE}
chunk tag) to help with readibility and/or setup.

\textbf{In the main Rmd file}

\chapter{The Second Appendix, for
Fun}\label{the-second-appendix-for-fun}

\chapter*{References}\label{references}
\addcontentsline{toc}{chapter}{References}

Placeholder

\hypertarget{refs}{}
\hypertarget{ref-R-tidyselect}{}
Henry, L., \& Wickham, H. (2020). Tidyselect: Select from a set of
strings. Retrieved from
\url{https://CRAN.R-project.org/package=tidyselect}

\hypertarget{ref-marwick2018packaging}{}
Marwick, B., Boettiger, C., \& Mullen, L. (2018). Packaging data
analytical work reproducibly using R (and friends). \emph{The American
Statistician}, \emph{72}(1), 80--88.
\url{http://doi.org/doi.org/10.1080/00031305.2017.1375986}

\hypertarget{ref-coreteam-extensions}{}
R-Core-Team. (2020). Writing r extensions. \emph{R Foundation for
Statistical Computing}. Retrieved from
\url{http://cran.stat.unipd.it/doc/manuals/r-release/R-exts.pdf}

\hypertarget{ref-hadley-packages}{}
Wickham, H. (2015). \emph{R packages} (1st ed.). Sebastopol, CA:
O'Reilly Media, Inc.


% Index?

\end{document}
